\documentclass{article}
\usepackage[margin=2cm, includefoot, footskip=30pt]{geometry}
\setlength\parindent{0pt}
\setlength{\parskip}{1em}
\usepackage{authblk}

\title{Cover Letter: A systematic literature review of the Prisoner's Dilemma; collaboration and influence.}

\author[1]{Nikoleta E. Glynatsi}
\author[1]{Vincent A. Knight}

\affil[1]{Cardiff University, School of Mathematics, Cardiff, United Kingdom}
\date{}
\setcounter{Maxaffil}{0}
\renewcommand\Affilfont{\itshape\small}

\begin{document}

\maketitle
\bigskip
\textbf{Declaration}: We confirm that neither the manuscript nor any parts of its
content are currently under consideration or published in another journal.

To the editors,

This paper presents a systematic literature review of the Iterated Prisoner's
Dilemma.

We feel that our paper is an excellent fit for Games because it provides a
concrete summary of existing literature on the prominent game theoretic field of
the Iterated Prisoner's Dilemma, which will be an excellent addition to the
advanced forum for studies related game theory that the journal is aiming to
build.

The paper does not only provide a literature review, but furthermore, it
provides an original quantitative analysis of collaborative behaviour of authors
in game theoretic subfields. More specifically, a bespoken piece of software
was used to automatically collect 3000 articles on the Prisoner's Dilemma, 3500
articles on Auction games and 750 articles on the Price of Anarchy. The data was
processed using modern data analysis techniques to compare the collaborative
behaviour of authors and their influence in their respective fields.

This paper offers a strong contribution on two fronts:

\begin{itemize}
    \item It presents a literature review which will allow the research
    community to understand overall trends on related research of the Prisoner's
    Dilemma and already existing results.
    \item It demonstrates the usage of modern data mining and data analysis
    techniques on understand the behaviour of authors in game theoretic
    subfields; which in a sense, understanding behaviour, is the aim of game
    theory itself.
\end{itemize}

This work has been carried out with the highest standard of reproducibility: all
scripts for collecting data and code for the analysis are not only well described but they
are also all open source, archived and made available online.


Thank you for taking the time to consider our work,

The Authors.
\end{document}