\documentclass{article}
\usepackage[margin=2cm, includefoot, footskip=30pt]{geometry}
\setlength\parindent{0pt}
\setlength{\parskip}{1em}
\usepackage{authblk}

\title{Cover Letter: A bibliometric study of research topics, collaboration and
influence in the field of the Iterated Prisoner’s Dilemma.}

\author[1]{Nikoleta E. Glynatsi}
\author[1]{Vincent A. Knight}

\affil[1]{Cardiff University, School of Mathematics, Cardiff, United Kingdom}
\date{}
\setcounter{Maxaffil}{0}
\renewcommand\Affilfont{\itshape\small}

\begin{document}

\maketitle

To the editors,

This paper presents a bibliometric study of the field the ``Prisoner’s
Dilemma’’. It identifies and presents five research topics in the field using a
topic modeling technique, and moreover, it explores the collaborative behaviour
of its authors using a graph theoretic analysis of the co-authorship network.
Our work is extending and improving on previous results which include
manuscripts appearing in Nature Communications.

The paper provides an original quantitative analysis of collaborative behaviour
of authors in the game theoretic subfield of the Prisoner’s Dilemma. A bespoke
piece of software is used to automatically collect metadata from 2470 articles
on the subfield from 4 prominent journals, which includes Nature, and the
preprint server arXiv. The Latent Dirichlet Allocation technique is  used to
identify five research topics in the field which are demonstrated to have been
relevant over the course of time.  The collaborative behaviour of the field is
analysed and compared to two other game theoretic subfields, ``Auction games’’
and the ``Price of Anarchy’’. Finally, it is demonstrated that authors in the
Prisoner’s Dilemma also do not influence or gain much information by their connections,
unless they are connected to a ``main” group of authors.

This paper offers a strong contribution on two fronts:

\begin{itemize}
    \item It demonstrates the usage of modern data mining and natural language
    process on identifying the topic structure of a subject area which has
    attracted the attetion of researchers across fields.
    \item It demonstrates the usage of modern data mining and data analysis
    techniques on understand the behaviour of authors in game theoretic
    subfields; which in a sense, understanding behaviour, is the aim of the game
    theory itself.
\end{itemize}

We feel that our paper is an excellent fit for Nature Communications as our
paper is within the scope of the journal, as demonstrated by a number of
publications in the field. One such example is the work of Li, Aste, Caccioli
and Livan:`` Early co-authorship with top scientists predicts success
in academic careers'' which in 2019 was published in Nature Communications and
demonstrated how a bibliometric analysis can be used to study the impact of
co-authorship with established, highly-cited scientists on the careers of junior
researchers.

This work has been carried out with the highest standard of reproducibility: all
scripts for collecting data and code for the analysis are not only well
described but they are also all open source, archived and made available online.

Thank you for taking the time to consider our work,

The Authors.

\end{document}