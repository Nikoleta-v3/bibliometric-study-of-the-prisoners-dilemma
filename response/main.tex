\documentclass[11pt]{article}
\usepackage[utf8]{inputenc}
\usepackage{fullpage}
\usepackage{hyperref}

% package needed for optional arguments
\usepackage{xifthen}
% define counters for reviewers and their points
\newcounter{reviewer}
\setcounter{reviewer}{0}
\newcounter{point}[reviewer]
\setcounter{point}{0}

% This refines the format of how the reviewer/point reference will appear.
\renewcommand{\thepoint}{P\,\thereviewer.\arabic{point}} 

% command declarations for reviewer points and our responses
\newcommand{\reviewersection}{\stepcounter{reviewer} \bigskip \hrule
				  \section*{Reviewer \thereviewer}}

\newenvironment{point}
   {\refstepcounter{point} \bigskip \noindent {\textbf{Reviewer~Point~\thepoint}
   } ---\ } {\par }

\newcommand{\shortpoint}[1]{\refstepcounter{point}  \bigskip \noindent
	{\textbf{Reviewer~Point~\thepoint} } ---~#1\par }

\newenvironment{reply}
   {\medskip \noindent \begin{sf}\textbf{Reply}:\  } {\medskip \end{sf}}

\newcommand{\shortreply}[2][]{\medskip \noindent \begin{sf}\textbf{Reply}:\  #2
	\ifthenelse{\equal{#1}{}}{}{ \hfill \footnotesize (#1)}%
	\medskip \end{sf}}

\title{A bibliometric study of research topics, collaboration and centrality in
the Iterated Prisoner's Dilemma.}

\begin{document}

\maketitle

\section*{Response to the reviewers}

We would like to open this response by thanking the reviewers for their
thoughtful comments and suggestions. We have fully taken their comments on board
and made modifications and additions to the manuscript. We feel this has greatly
improved the work.

%TODO do we need to address that the Editor suggested we rerun the data
%collection?

We will now take each comment of the reviewers in turn and highlight our efforts
to improve the work.

\reviewersection

\begin{point}
First, the authors used search terms ``prisoner's dilemma'', ``prisoners
dilemma'', ``prisoner dilemma'', ``prisoners evolution'', and ``prisoner game
theory'' to collect data. However, they did not justify the reasons for the use
of these terms. Why these terms, is there any support for the choice?
\end{point}

\begin{reply}
We have added a sentence justifying the use of the specific search terms.
\end{reply}

\begin{point}
Second, I am wondering whether you had conducted manual screening of collected
literature to ensure their close relation to the research target. Different from
systematic review aiming to provide in-depth analysis of representative
literature, bibliometric analysis is used for big data analysis to enable a
comprehensive and general view. Thus, sufficient coverage of relevant literature
are especially essential for a bibliometric study. In particular, you included
articles which had used any of the search terms in the title, the abstract or
the text. The use of title and abstract is understandable since they are
commonly used to search articles (you can refer to bibliometric studies such as,

\begin{itemize}
\item \url{https://www.sciencedirect.com/science/article/pii/S0360131520300555},

\item
\url{https://bera-journals.onlinelibrary.wiley.com/doi/full/10.1111/bjet.12907?casa_token=vCNbiEvDF5gAAAAA%3Ap9AeWKgtwePj-ABgaPIFecQrTObZdjh26sG49aeHs47Ms7zHROw8I_JhUpvQnOlZ0Vx5oCrzwmCN1kTsVQ},

\item
\url{https://idp.springer.com/authorize/casa?redirect_uri=https://link.springer.com/article/10.1007/s11042-020-09062-7&casa_token=z8E4b9PYoncAAAAA:9X2oCiTpIRj2V6xMkkLsjXvAyD2NSmMxVNYK_K-fLdb67SLc3zUFxoCmmniFJCNprc20XInB9KPum71LMCw)},
\end{itemize}

however, the inclusion of full text can be troublesome as it may result in too
many irrelevant articles being included. Thus, manual screening is needed and
should be conducted to ensure data quality.
\end{point}

\begin{reply}
We agree with the reviewer's point. We agree that articles which refer to the
prisoner's dilemma in the text do not necessarily study the topic. For that
reason we have conducted a manual check of all the articles included in our study
when the search terms existed only in the text. This was not clarified in our
original submission. We now clearly state that manual screening was conducted.
\end{reply}

\begin{point}
Third, you have to clearly justify bibliographically your paper selection, your
descriptors, inclusion and exclusion criteria you used to select paper, and the
aspects you analyze, in such a way that your viewpoint is rationalized.
\end{point}

\begin{reply}
We have included a sentence, in the Introduction section, to justify our selections.
\end{reply}

\begin{point}
Fourth, in your study, the appropriate number of topics was chosen based on the
coherence value. Coherence indicates the possibility terms within a topic
commonly appear together. However, there is also another commonly used indicator
called exclusivity, which indicates the possibility the terms are exclusively
related to the topic. Commonly, the two indicators will be used to select the
best fitted model together. Hence, suggest adding it in your study, so to choose
the appropriate number of topics by using both coherence and exclusivity.
\end{point}

\begin{reply}
We thank the reviewer for their suggestion to also include the exclusivity
score. We have included exclusivity in our analysis.
\end{reply}

\begin{point}
Fifth, in page 9, the authors mentioned “The three models are LDA models for the
entire data set for n equal to 5, 6 and the optimal number of topics over time,”
it is not clear to me what do you mean by “the optimal number of topics over
time.” How was it defined and calculated? Details should be provided.
\end{point}

\begin{reply}
We have included more details.
\end{reply}

\begin{point}
In addition, the authors should specify clearly the significance of their study,
what specific implications can be obtained from their analyses, particularly
those concerning future research directions As one of the typical aims for
review paper is to analyze the previous studies in order to provide helpful
suggestions or directions for researches to conduct future studies. This should
have been done in the Discussion and Conclusion sections. The authors need to
summarize their findings instead of presenting and discussing individual
findings. More importantly, they need to inspire the readers by providing a list
of suggestions/directions of future research based on the discussion. The
authors need to advise readers what to do next rather than ask the readers to
interpret the data on their own.
\end{point}

\begin{reply}
Thank you for this comment which encouraged us to reflect on the conclusions
of the paper. As a result we have re-written the conclusion section.
\end{reply}

\reviewersection

\begin{point}
\begin{itemize}
	\item The analysis of the co-authorship network is not convincing for me. The author
	created a cross-sectional map of the co-authorship network and argued that
	people in central positions benefit from their network. But it is not analyzed
	whether they benefit in anything only whether someone if centrally located or
	not and how dense is the network. The author did not analyze whether those who
	are more in a central position indeed get more collaboration subsequently or
	not. Further, the author said‘Nevertheless, there are authors that do benefit
	from their position, but these are only the authors connected to the main
	cluster.’ This is a tautology as the criteria for benefitting was based on their
	central position.
	\item I suggest only analyzing the centrality measures and structure of the co-author
	network (result section) and discussing what this could mean based on the
	previous literature (conclusion section). I would avoid claiming that benefit
	were tested in this paper.
\end{itemize}
\end{point}

\begin{reply}
	We agree with these two points made by the reviewer. We have removed the
	discussion regarding benefits and we just present the results on the centrality
	measures.
\end{reply}


\begin{point}
The author said that “The fact that most authors of the main cluster are
primarily publishing in evolutionary dynamics on networks indicates that
publishing in this specific topic differs from the other topics covered in this
manuscript.”. Couldn’t this mean rather that authors publishing in evolutionary
dynamics are more similar to other disciplines as they can collaborate with them
more?
\end{point}

\begin{reply}
We agree with the reviewer. The suggested interpretation was included in the paper.
\end{reply}

\begin{point}
	\begin{itemize}
		\item I was not convinced that the dataset covered all relevant article and only the
		relevant articles. It was not described at all what steps were taken to find the
		grey literature that was not covered by the database searches. Did you
		hand-searched the citation of eligible studies or consulted with experts? These
		are all very important steps to get a comprehensive review. I know that some
		steps were taken as the author mentions to include some extra paper. Further,
		what steps were taken to ensure that only relevant articles were included? An
		abstract can mention Prisoner’s Dilemma but not analyze this topic. Usually, it
		is good to scope the abstract (at least) to check whether studies were of
		interest. Further, the number of duplicates should be reported.

		\item I suggest mapping your data collection in a flow chart.
	\end{itemize}
\end{point}

\begin{reply}
We have re-written our data collection process to include more details.
We have also included a flow chart to visualize the process.
\end{reply}

\begin{point}
The strength of ties could be analyzed to map multiple collaborations between
two nodes. 
\end{point}

\begin{reply}
We agree with this point. We performed a preliminary assessment which showed
that our results remain the same. This is included in the conclusion.
\end{reply}

\begin{point}
The contribution of the paper should be named in the introduction. 
\end{point}

\begin{reply}
We have made this suggested change.
\end{reply}

\begin{point}
Disciplines differ a lot in the number of co-authors. For example, medical
papers often have more than 5 co-authors where sociology papers often have
single authors. Isn't the results merely coming from this pre-established
difference between disciplines?
\end{point}

\begin{reply}
We agree with the reviewer that there is a pre-established
difference between disciplines. The aim of the paper is to understand the
difference of the prisoner's dilemma compared to other game theoretic fields.
\end{reply}

\begin{point}
The structure of the text needs to be improved a lot before this work is
publishable. The methodology should be outlined in the Methodology section and
not in the Results section (e.g. fitting exponential lines to the number of
articles to see that the numbers are increasing). The contribution of the paper
needs to be in the introduction and not in the last paragraph of the Methodology
section.
\end{point}

\begin{reply}
We have made this suggested change.
\end{reply}

\begin{point}
The text contains some grammar mistakes (e.g. “The performance of the models
are”) typos (e.g. “Figure ??”) and unclear reference (e.g. “In comparison, 2b
gives the visualisation of LDA” where 2b could refer to a Table, Figure, or
topic).
\end{point}

\begin{reply}
We have gone over the manuscript and corrected any typos.
\end{reply}

\begin{point}
The abbreviation should be spelt out the first time they are used and then the
short version should be used consequently.
\end{point}

\begin{reply}
We have made this suggested change.
\end{reply}

\begin{point}
I like that Latent Dirichlet Allocation is defined in the introduction. But it
would enhance the readability if it was defined the first time it is used.
\end{point}

\begin{reply}
We have made this suggested change.
\end{reply}


\begin{point}
The author said that “The appropriate number of topics is chosen based on the
coherence value”. 6 topics provided the best coherence value but still 5 topics
were selected. The description of this conduct is a bit confusing.
\end{point}

\begin{reply}
We have made this suggested change.
\end{reply}

\begin{point}
Reference to software should go to the method section instead of the
Acknowledgment.
\end{point}

\begin{reply}
We have made this suggested change.
\end{reply}

\end{document}
