\documentclass{article}
% Package to manage page layout
\usepackage[margin=1.5cm, includefoot, footskip=30pt]{geometry}

\setlength\parindent{0pt}
\setlength{\parskip}{1em}

%%%%%%%PACKAGES HERE%%%%%%%
\usepackage{amsmath}
\usepackage{amsthm}
\usepackage{amssymb}
\usepackage{hyperref}
\usepackage{standalone}
\usepackage{subcaption}
\usepackage{adjustbox}
\usepackage{tikz}
\usepackage{booktabs}
\usepackage{minted}
\usepackage{multicol,multirow,array}
\usepackage{graphicx}
\usepackage{algorithm,algorithmic}
\usetikzlibrary{er,positioning, calc, patterns}
\usetikzlibrary{decorations.pathreplacing}

\definecolor{background}{RGB}{5, 66, 81}
\usemintedstyle{tango}

\setcounter{secnumdepth}{4}
\setcounter{tocdepth}{4}

\usepackage{kbordermatrix}
\theoremstyle{definition}
\newtheorem{definition}{Definition}[section]

%%%%%%%%%%%%%%%%%%%%%%%%%%%%%%%PARAMETERS%%%%%%%%%%%%%%%%%%%%%%%%%%%%%%%%%%%%%%%
\newcommand{\totalarticles}{\input{assets/total_articles.txt}}
\newcommand{\manual}{\input{assets/prov_manual.txt}}
\newcommand{\authors}{\input{assets/num_Nodes.txt}}
\newcommand{\edges}{\input{assets/num_Edges.txt}}
\newcommand{\isolated}{\input{assets/num_Isolated_nodes.txt}}
\newcommand{\isolatedpercentage}{\input{assets/perce_Isolated_nodes.txt}}
\newcommand{\connectedcomponents}{\input{assets/num_Connected_components.txt}}
\newcommand{\communities}{\input{assets/num_Communities.txt}}
\newcommand{\largestcc}{\input{assets/Size_of_largest_component.txt}}
\newcommand{\clustering}{\input{assets/Clustering_coeff.txt}}
\newcommand{\avdegree}{\input{assets/Av._degree.txt}}
% \newcommand{\prisonerscc}{\input{assets/prisoners_clustering.txt}}
% \newcommand{\pricecon}{\input{assets/price_connected_components.txt}}
% \newcommand{\pricecc}{\input{assets/price_clustering.txt}}
% \newcommand{\auctioncon}{\input{assets/auction_connected_components.txt}}
% \newcommand{\auctioncc}{\input{assets/auction_clustering.txt}}
% \newcommand{\prisonerisolated}{\input{assets/prisoners_isolated.txt}}
% \newcommand{\auctionisolated}{\input{assets/auction_isolated.txt}}
% \newcommand{\priceisolated}{\input{assets/price_isolated.txt}}
%%%%%%%%%%%%%%%%%%%%%%%%%%%%%%%%%%%%%%%%%%%%%%%%%%%%%%%%%%%%%%%%%%%%%%%%%%%%%%%%
%%%%%%%%%%%%%%%%%%%%%%%%%%%%%%%%%%%%%%%%%%%%%%%%%%%%%%%%%%%%%%%%%%%%%%%%%%%%%%%%
\title{A systematic literature review of the Prisoner's Dilemma; collaboration and influence.}
\author{Nikoleta E. Glynatsi, Vincent A. Knight}
\date{}

\begin{document}

\maketitle

\begin{abstract}
    The prisoner's dilemma is a well known game used since the 1950's as a framework
    for studying the emergence of cooperation; a topic of continuing interest
    for mathematical, social, biological and ecological sciences. The iterated version
    of the game attracted attention in the 1980's after
    the publication of the ``The Evolution of Cooperation'' and has been a topic
    of pioneering research ever since. This paper aims to provide a detailed
    literature review of the field. This is achieved by partitioning the timeline into six different
    sections. Furthermore, a comprehensive data set of literature is analysed
    using network theoretic approaches in order to explore the influence and the
    collaborative behaviour of the field itself.
\end{abstract}

\section{Introduction}\label{section:introduction}

In everyday life people choose between being selfish or being selfless when they
interact with others. The question that arises is: which behaviour is more beneficial?
There is a simple way of representing these behaviours/concepts. This is to use a
particular two player non-cooperative game called the prisoner's dilemma, originally
described in~\cite{Flood1958}.

Each player has two choices, to either be selfness and cooperate or to act in a
selfish manner and choose to defect. Each decision is made simultaneously and independently.
The fitness of each player is influenced by its own behaviour, and the behaviour
of the opponent. Both players do better if they choose to cooperate than if both
choose to defect. However, a player has the temptation to deviate as that player will
receive a higher payoff than that of a mutual cooperation.

A player's payoffs are generally represented by (\ref{eq:the_pd_payoffs}). Both
players receive a reward for mutual cooperation, \(R\), and a payoff \(P\) for
mutual defection. A player that defects while the other cooperates receives a payoff of
\(T\), whereas the cooperator receives \(S\). The dilemma exists due
to constrains (\ref{eq:constrain_one}) and (\ref{eq:constrain_two}).

\begin{equation} \label{eq:the_pd_payoffs}
    \begin{pmatrix}
    R & S \\ T & P
    \end{pmatrix}
\end{equation}

\begin{equation}\label{eq:constrain_one}
    T > R > P > S
\end{equation}

\begin{equation}\label{eq:constrain_two}
    2R > T + S
\end{equation}

Another common representation of the payoff matrix is given by~(\ref{eq:the_pd_payoffs_with_cost}),
where \(b\) is the benefit of the altruistic behaviour and \(c\) it's its cost
(constrains (\ref{eq:constrain_one}) - (\ref{eq:constrain_two}) still hold).

\begin{equation}\label{eq:the_pd_payoffs_with_cost}
    \begin{pmatrix}
        b - c & c \\ b & 0
    \end{pmatrix}
\end{equation}

Constrains (\ref{eq:constrain_one}) - (\ref{eq:constrain_two}) and rational behaviour
guarantee that it never benefits a player to cooperate. However,
when the game is studied in a manner where prior outcome matters,
defecting is no longer necessarily the dominant choice.

The repeated form of the game is called the iterated prisoner's dilemma and theoretical
works have shown that cooperation can emerge once players interact repeatedly.
Arguably, the most important of these works has been R. Axelrod's
``The Evolution of Cooperation''~\cite{Axelrod1984}.
In his book Axelrod reports on a series of computer tournaments he organised of
a finite turns games of the iterated prisoner's dilemma. Participants
had to choose between cooperation and defection again and again while having
memory of their previous encounters. Academics from several fields were invited to
design computer strategies to compete. The pioneering work of Axelrod
showed that greedy strategies did very poorly in the long run whereas altruistic
strategies did better.

``The Evolution of Cooperation'' is considered a milestone in the field but it
is not the only one. On the contrary, the prisoner's dilemma has attracted much
attention ever since the game's origins. In Section~\ref{section:analysis} a comprehensive
data set of literature regarding the prisoner's dilemma, and collected from
the following sources, is presented and analysed.

\begin{multicols}{2}
    \begin{itemize}
        \item arXiv~\cite{mckiernan2000}; a repository of electronic preprints.
        It consists of scientific
        papers in the fields of mathematics, physics, astronomy, electrical engineering,
        computer science, quantitative biology, statistics, and quantitative finance,
        which all can be accessed online.
        \item PLOS~\cite{plos}; a library of open access journals and other scientific literature
        under an open content license. It launched its first journal, PLOS Biology,
        in October 2003 and publishes seven journals, as of October 2015.
        \item IEEE Xplore Digital Library (IEEE)~\cite{ieee}; a research database for discovery
        and access to journal articles, conference proceedings, technical standards,
        and related materials on computer science, electrical engineering and electronics,
        and allied fields. It contains material published mainly by the Institute of
        Electrical and Electronics Engineers and other partner publishers. 
        \item Nature~\cite{nature}; a British multidisciplinary scientific journal,
        first published on 4 November 1869. It was ranked the world's most cited
        scientific journal by the Science Edition of the 2010 Journal Citation Reports
        and is ascribed an impact factor of 40.137, making it one of the world's
        top academic journals.
        \item Springer~\cite{springer}; a leading global scientific publisher of
        books and journals. It publishes close to 500 academic and professional
        society journals.
    \end{itemize}
\end{multicols}

The aim of the analysis is to review the amount of published academic articles
as well as to measure and explore the collaborations within the field.

\section{Timeline}\label{section:timeline}

In this section a large amount of literature regarding the prisoner's
dilemma is reviewed. The review starts from the year the game was formulated
and covers publications all the way to today.

\subsection{Origins of the prisoner's dilemma}\label{section:origin}

The origin of the prisoner's dilemma goes back to the 1950s in early experiments
conducted at RAND~\cite{Flood1958} to test the applicability of games
described in~\cite{VonNeumann1944}. The game received it's name later the same year.
According to~\cite{Tucker1983}, A. W. Tucker (the PhD supervisor of J. Nash~\cite{Nash1951}),
in an attempt to delivery the game with a story during a talk described the players
as prisoners and the game has been known as the prisoner's dilemma ever since.

The early research on the prisoner's dilemma had been very constrained. The only
source of experimental results was through groups of humans that simulated rounds
of the games and human groups came with disadvantages. Human can behave very
randomly and in several experiments both the size and the background of the
individuals were different. Thus comparing results of two or more studies
became tricky.

The main aim of these experiments had been to understand the conditions behind
the emergence of cooperation. Conditions such as the gender~\cite{Evans1966, Lutzker1961, Mack1971} of
individuals, the representation of the game~\cite{Evans1966}, the distance between
players~\cite{Sensenig1972}, the initial effects~\cite{Tedeschi1968} and whether
the experimenter was biased~\cite{Gallo1968} were being explored and still are
today.

An early figure that sought out to understand the conditions under which altruist
behaviour emerged was Prof A. Rapoport. A mathematical psychologist, whose work focused on
how to promote international and national cooperation. Rapoport tried to conceptualize
strategies that could promote international cooperation. In his teaching and research
he used the prisoner's dilemma. Rapoport offered the field many insights, he
is the creator of strategies such as Tit for Tat and Pavlov, which are going to be
discussed in later parts of this paper.

Decades later the political scientist R. Axelrod introduced the 
pioneer computer tournaments that have largely replaced human subjects in the study
of the iterated prisoner's dilemma ever since. In the next section  these
tournaments and several strategies that were design by researchers,
such as Rapoport, are introduced.

\subsection{Axelrod's tournaments and intelligent design of strategies}\label{subsection:intelligent_design}

As discussed in Section~\ref{section:origin}, before 1980 a great deal of
research was done in the field, however, as described in~\cite{Axelrod2012}, the
political scientist R. Axelrod believed that there was no clear answer to the question
of how to avoid conflict, or even how an individual should play the game. Combining
his interest in artificial intelligence and political sciences Axelrod created a
framework for exploring these questions using computer tournaments.

Axelrod's tournaments made the study of cooperation of critical interest
once again, academic articles were being published reproducing Axelord's work,
accessing and further developing his results. As described in~\cite{Rapoport2015},
``Axelrod's “new approach” has been extremely
successful and immensely influential in casting light on the conflict between an
individual and the collective rationality reflected in the choices of a population
whose members are unknown and its size unspecified, thereby opening a new avenue
of research''.
In a collaboration with a colleague, Douglas Dion, Axelrod in~\cite{Axelrod1988}
summarized a number of works that were immediately inspired from the ``Evolution of Cooperation''
and in 2012,~\cite{Jurisic2012} wrote a review on iterated prisoner's
dilemma strategies and big competitions that had occurred since the originals.

In essence, Axelrod asked researchers to design a strategy,
set a number of rules, with the purpose of wining an iterated prisoner's dilemma
tournament. These strategies were constructed by an intelligent cause and not an
undirected process, and here there are refereed to as strategies of intelligent
design. This section covers Axelrod's original tournaments as well as 
research that introduced new strategies of intelligent design.

The first reported computer tournament took place in 1980~\cite{Axelrod1980a}.
Several scientists were invited to submit their strategies, written in the
programming languages Fortran or Basic. There was a total of 13 submissions
made by the following researchers,

\begin{multicols}{2}
    \begin{enumerate}
        \item T Nicolaus Tideman and Paula Chieruzz;
        \item Rudy Nydegger;
        \item Bernard Grofman;
        \item Martin Shubik;
        \item Stein and Anatol Rapoport;
        \item James W Friedman;
        \item Morton Davis;
        \item Jim Graaskamp;
        \item Leslie Downing;
        \item Scott Feld;
        \item Johann Joss;
        \item Gordon Tullock;
    \end{enumerate}
\end{multicols}

and a \(13^{th}\) who remained anonymous.

Each competed in a 200 turn match against all 12 opponents, itself and a player
that played randomly (called \textbf{Random}). This type of tournament is referred to as a round robin.
The tournament was run only once, each participant knew
the exact length of the matches and had access to the full history of each match.
Furthermore, Axelrod performed a preliminary tournament and the results were known
to the participants. The payoff values used for equation~(\ref{eq:the_pd_payoffs}) where
\(R=3, P=1, T=5\) and \(S=0\). These values are commonly used in the literature
and unless specified will be the values used in the rest of the work described here.

The winner of the tournament was determined by the total average score and not by
the number of matches won. The strategy that was announced the winner was
submitted by Rapoport and was called \textbf{Tit For Tat}. Tit for Tat, is a
strategy that always cooperates on the first round and then mimics the opponent's
previous move.
The success of Tit for Tat came as a surprise. It was not only the simplest submitted
strategy but it had also won the tournament even though it could never do better
than any player it was interacting with.

In order to further test the results Axelrod performed a second
tournament later in 1980~\cite{Axelrod1980b}. The results of the first tournament
had been publicised and the second tournament received much more attention, with 62 entries
made by the following people,

\begin{multicols}{3}
    \begin{enumerate}
        \item Gail Grisell;
        \item Harold Rabbie;
        \item James W Friedman;
        \item Abraham Getzler;
        \item Roger Hotz;
        \item George Lefevre;
        \item Nelson Weiderman;
        \item Tom Almy;
        \item Robert Adams;
        \item Herb Weiner;
        \item Otto Borufsen;
        \item R D Anderson;
        \item William Adams;
        \item Michael F McGurrin;
        \item Graham J Eatherley;
        \item Richard Hufford;
        \item George Hufford;
        \item Rob Cave;
        \item Rik Smoody;
        \item John Willaim Colbert;
        \item David A Smith;
        \item Henry Nussbacher;
        \item William H Robertson;
        \item Steve Newman;
        \item Stanley F Quayle;
        \item Rudy Nydegger;
        \item Glen Rowsam;
        \item Leslie Downing;
        \item Jim Graaskamp and Ken Katzen;
        \item Danny C Champion;
        \item Howard R Hollander;
        \item George Duisman;
        \item Brian Yamachi;
        \item Mark F Batell;
        \item Ray Mikkelson;
        \item Craig Feathers;
        \item Fransois Leyvraz;
        \item Johann Joss;
        \item Robert Pebly;
        \item James E Hall;
        \item Edward C White Jr;
        \item George Zimmerman;
        \item Edward Friedland;
        \item X	Edward Friedland;
        \item Paul D Harrington;
        \item David Gladstein;
        \item Scott Feld;
        \item Fred Mauk;
        \item Dennis Ambuehl and Kevin Hickey;
        \item Robyn M Dawes and Mark Batell;
        \item Martyn Jones;
        \item Robert A Leyland;
        \item Paul E Black;
        \item T Nicolaus Tideman and Paula Chieruzz;
        \item Robert B Falk and James M Langsted;
        \item Bernard Grofman;
        \item E E H Schurmann;
        \item Scott Appold;
        \item Gene Snodgrass;
        \item John Maynard Smith;
        \item Jonathan Pinkley;
        \item Anatol Rapoport.
    \end{enumerate}
\end{multicols}

The new participants knew the results of the previous tournament. The rules
were similar with only a few alternations. The tournament was repeated 5 times
and the length of each match was not known to the participants. Axelrod intended
to use a fixed probability (refereed to as `shadow of the future'~\cite{Axelrod1988})
of the game ending on the next move. However, 5 different length matches were
selected for each match 63, 77, 151, 308 and 401, such that the average length
would be around 200 turns.

Several entries of the second tournament tended to be variants of Tit for Tat, such as
\textbf{Tit for Two Tats} submitted by John Maynard Smith. Tit for Two
Tats defects only when the opponent has defected twice in a row. Another well
known entry was \textbf{Grudger}.
Grudger was originally submitted by James W. Friedman. Grudger is a strategy that
will cooperate as long as the opponent does not defect. The name Grudger was give
to the strategy in~\cite{Li2014}. Though the strategy goes by many names in the
literature such as, Spite~\cite{Beaufils1997}, Grim Trigger~\cite{Banks1990} and
Grim~\cite{Van2015}.

Despite the larger size of the second tournament, none of the variants and new strategies managed
to outperform the simple designed strategy. The winner was once again Tit for Tat.
The conclusions made from the first two tournaments were that the strong performance
of the strategy was due to:

\begin{itemize}
    \item The strategy would start of by cooperating.
    \item It would forgive it's opponent after a defection.
    \item It would always be provoked by a defection no matter the history.
    \item As soon as the opponents identified that they were playing Tit for Tat,
    they would choose to cooperate for the rest of the game.
\end{itemize}

However, the success of Tit for Tat was not unquestionable. Several papers
showed that  stochastic uncertainty severely undercut the effectiveness
of reciprocating strategies. Though such stochastic uncertainty
are unlikely to occur in a computer tournament, but have to be expected in real
life situations~\cite{Milinski1987}.

In~\cite{Molander1985} it was proved that
in an environment where \textbf{noise} is introduced two strategies playing Tit
for Tat receive the same average payoff as two Random players. Noise is a probability
that a player's move will be flipped. In 1986,~\cite{Donninger1986} ran a computer
tournament with a 10 percent chance of noise and Tit for Tat finished sixth out of
21 strategies. Bendor in~\cite{Bendor1991} performed tournament similar to Axelrod's
with noise and a probability of 0.0067 of ending in the next turn.
His results demonstrated the poor performance of Tit for Tat once again and showed
that the highest ranked strategies were more generous ones.
His top ranked strategy was called \textbf{Nice and Forgiving}.
Nice and Forgiving, differs in significant ways
from Tit for Tat. Initially, Nice's generosity takes the form of a benign indifference.
It will continue to play cooperation as long as its rival's cooperation level
exceeded 80\%. Secondly, although it will retaliate if its rival's observed cooperation fell
below 80, it is willing to revert to full cooperation before its partner does,
so long as the partner satisfies a certain thresholds of acceptable behaviour.

Hammerstein~\cite{Hammerstein1984}, pointed out another weakness of Tit
for Tat in noisy environments. If by mistake, one of two Tit for Tat players makes a wrong move,
this locks the two opponents into a hopeless sequence of alternating defections
and cooperations. To overcome this error~\cite{Wolfgang2006} introduced another more
generous variant of Tit for Tat, \textbf{OmegaTFT}. They also altered their strategy
so that it had the ability
to recognize and exploit the Random strategy in a way that after an opponent
strategy crosses a certain randomness threshold they conclude that the opponent is
a Random strategy and change the behaviour to act as a \textbf{Defector}, a strategy that
always choose to defect.

A second type of stochastic uncertainty is misperception, where a player's action is made correctly
but it's recorder incorrectly by the opponent. In 1986,~\cite{Sugden2004} introduced
a strategy called~\textbf{Contrite Tit for Tat} that was more successful than Tit
for Tat in such environments. The difference between the strategies was that
Contrite Tit for Tat was not so fast to retaliate a defection. This hints, alongside
more generous versions of Tit for Tat as discussed above, that the counter attack
to stochastic uncertainties is a strategy's readiness to defect after a defection.

Another protagonist in the literature and better perform strategy than Tit for Tat
came along in 1993. The strategy was \textbf{Pavlov} and though the name was
formally given  by Nowak~\cite{Nowak1993} the strategy had been around since 1965
known as Simpleton~\cite{rapoport1965}, introduced by Rapoport himself.
The strategy is based on the fundamental
behavioural mechanism win-stay, lose-shift. It starts off with a cooperation and
then repeats it's previous move only if it was awarder with a payoff of \(R\) or
\(T\). Pavlov is heavily studied in the literature and similarly to Tit for Tat
it's used in tournaments perform until today and has had many variants trying
to build upon it's success. \textbf{PavlovD}, just a Pavlov which starts the game
with a defection and \textbf{Adaptive Pavlov}. Adaptive Pavlov tried to
classify it's opponents as as one of the following strategies, \textbf{Cooperator}, Defector,
Pavlov, Random or Pavlov and chooses a strategy that maximise it's payoff against
the now `known' opponent.
Cooperator is a deterministic strategy that conditionally cooperates.

Several researchers, and this will discussed in later sections as well, argued
with Axelrod's result on simplicity. The advantages of complexity were shown
by~\cite{Beaufils1997} in 1997 where they introduced another well known strategy
\textbf{Gradual}. Gradual starts off by cooperating,
then after the first defection of the other player, it defects one time and cooperates
twice. After the second defection of the opponent, it defects two times and cooperates
twice. After the \(n^{th}\) defection it reacts with \(n\) consecutive defections 
and then two cooperations. In a tournament of 12 strategies~\cite{Beaufils1997},
Gradual had managed to outperform strategies such as Tit for Tat and Pavlov.
Gradual was surpassed by yet another complex and intelligent designed strategy
\textbf{Adaptive Tit for Tat}. The authors of~\cite{tzafestas-2000a} conducted
the exact same tournament as~\cite{Beaufils1997} with now 13 strategies and their
strategy ranked first.

Another interesting research on intelligent design strategies is on teams~\cite{J.P.Delahaye1993Lp,
J.P.Delahaye1995LIeP, A.Rogers2007Ctpw}. The strategies which make the team 
have been programmed with a recognition mechanism
by default. Once the strategies recognise one another, one would act as leader
and the other as a follower. The follower then plays as a Cooperator, cooperates
unconditionally and the leader would play as a Defector gaining the highest achievable
score. Followers would behave as Defector towards other strategies to defect
their score and help the leader. In~\cite{A.Rogers2007Ctpw}, a team for the University
of Southampton used teams and recognition patterns and managed to win the 2004
Anniversary Iterated Prisoner's Dilemma Tournament.

The next section focuses on a different aspect of research which is evolutionary
dynamics.

\subsection{Evolutionary Dynamics}\label{subsection:evolutionary_dynamics}

Following Axelrod's tournaments it was proven that direct reciprocity can
make cooperation successful is an round robin setting. A long standing question
has been to understand the conditions required for the emergence and maintenance
of cooperation in evolving populations. Is reciprocity the key here as well?
and what other mechanisms could favour cooperation? Due to the complex nature of
the iterated prisoner's dilemma strategies makes their evolutionary stability
more complex to study and though the question and these still remain open questions.
Even so, several insights have been published over the years and in this section
several remarks that have been made on the evolutionary dynamics are discussed.

In the later sections of~\cite{Axelrod1980b}, Axelrod discusses about
an ecological tournament he performed using the 62 strategy of the second tournament
An ecological approach is a simulation of theoretical
future rounds of the game where strategies that do better are more likely to be
included in future rounds than others.
The simulation of the process, as described in~\cite{Axelrod1980b}, is straightforward.
Let us consider an example with four strategies Tit for Tat, Tit for Two Tat,
Cooperator and Defector compete in an ecological tournament.
The expected payoff matrix, when these four strategies interact, is give by,

\[\begin{bmatrix}
    3.0,  & 3.0,  & 3.0, & 0.99 \\
    3.0,  & 3.0,  & 3.0, & 0.99 \\
    3.0,  & 3.0,  & 3.0, & 0.0  \\
    1.02, & 1.039,& 5.0, & 1.0 \\
\end{bmatrix}\]

Starting with proportions of each type in a given generation, their proportions
for the next generation needs to be calculated. This is achieved by calculating
the weighted average of the scores of a given strategy with all other players.

\begin{itemize}
    \item The weights are the numbers of the other strategies which
    exist in the current generation.
    \item The numbers of a given strategy in the next generation is then taken to
    be proportional to the product of its numbers in the current generation and
    its score in the current generation.
\end{itemize}

\begin{figure}[!hbtp]
    \centering
    \includegraphics[width=.6\textwidth]{./assets/images/ecological_tournament.png}
    \caption{Results on an ecological tournament with Tit for Tat, Tit for Two Tats,
    Cooperator and Defector.}
    \label{fig:ecological_tournament}
\end{figure}

The process is then repeated for a given number of future tournaments.
Figure~\ref{fig:ecological_tournament} illustrates a simulation of our
ecological tournament, as shown strategies that cooperate quickly kill off the Defector.
A similar result was presented by Axelrod. In his ecological tournament cooperative
strategies managed to take over the population over time. On the other hand exploitative
strategies started to die off as weaker strategies were becoming extinct. In other
words they were dying because there was fewer and fewer prey for them to exploit.
In 1981, Axelrod also studied the prisoner's dilemma in an evolutionary context based
on the evolutionary approaches of John Maynard Smith~\cite{Smith1973,
Smith1974, Smith1979}. Smith is a well know evolutionary biologist as well
as an attendant of Axelord's second tournament. John Maynard Smith alongside George Price
are considered fundamental figures of evolutionary game theory. In~\cite{Smith1973}
they introduced the definition of an evolutionary stable strategy (ESS).

Imagine a population made up of individuals where everyone follows the
same strategy \(B\) and a single individual adopts a mutant strategy \(A\).
Strategy \(A\) is said to invade strategy \(B\) if the payoff of \(A\) against \(B\)
is greater than the expected payoff received by \(B\) against itself.
Since strategy \(B\) is in a population that interacts only with itself,
the concept of invasion is equivalent to a single mutant being able to outperform
the average population. Thus for a strategy to be ESS it must be able to resists
any invasion.

The work described in~\cite{axelrod1981}, studied the evolutionary stability of
Tit for Tat and although the strategy was likely to take over the population, its
stability was conditional on the importance of the future of the game. This is
represented by a discounting factor denoted as \(w\). Axelrod showed that if \(w\)
was sufficiently large, Tit for Tat could resist invasion by any other strategy.
Moreover, he showed how a small cluster of Tit for Tat players could invade a extortionate
environment.
Alongside the biologist William Donald Hamilton they wrote about the biological applications
of the evolutionary dynamics of the iterated prisoner's dilemma~\cite{Axelrod1984}
and won the Newcomb-Cleveland prize of the American Association for the Advancement
of Science. Arguing with Axelrod's results.
In~\cite{Boyd1987} Boyd and Lorderbaum show that if \(w\), the importance of the
future of the game, is large enough then no deterministic strategy is ESS because
it can always be invaded by any pair of other strategies. This was also independently
proven by~\cite{Pudaite1987}, so is reciprocity the answer to the emergence of
cooperation?

Cooperation can be favoured in several other settings, in~\cite{nowak1989},
Nowak and Sigmund studied the dynamics of the evolutionary
iterated prisoner's dilemma with a spectrum of stochastic strategies that only
remember their opponent's last move, not their own. They found that there can be
multiple fixed points that there can be an evolutionary stable coexistence among
multiple such strategies and in  1992,~\cite{Nowak1992b} explored how local
interaction alone can facilitate population wide
cooperation in a one shot prisoner's dilemma.

The two deterministic strategies Defector and Cooperator, were placed onto a two
dimensional square array where the individuals could interact only with the immediate
neighbours. The number of immediate neighbours could be either, fourth, six or eight.
As shown in Figure~\ref{fig:topologies}, where each node represents a player and
the edges denote whether two players will interact, this topology is refereed to
as spatial topology. Thus each cell of the lattice is occupied by a Cooperator
or a Defector.

\begin{itemize}
    \item At each generation step each cell owner interacts with its immediate neighbours.
    \item The score of each player is calculated as the sum of all the scores the player
    achieved at each generation.
    \item At the start of the next generation, each lattice
    cell is occupied by the player with the highest score among the previous owner
    and the immediate neighbours.
\end{itemize}

This topology is refereed to as spatial topology. The population dynamics of these
experiments were studied as a function of the temptation (\(T\)) payoff.
More specifically the following payoff matrix was used, which is equivalent
to equation~(\ref{eq:the_pd_payoffs}):

\begin{equation}
    \begin{pmatrix}
    R & S \\ T & P
    \end{pmatrix}
    =
    \begin{pmatrix}
        1 & 0 \\ b & 0
    \end{pmatrix}
\end{equation}

where \((b>1)\). Note that this representation fails constraint (\ref{eq:constrain_one}).

Local interactions proved that as long as small clusters of cooperators form and
they can benefit from interactions with other cooperators and while avoiding
interactions with defectors, global cooperation will continue.
Cooperation was also proven to be evolutionary beneficial under conditions were
the benefit to cost ratio \(\frac{b}{c}\) is higher to the number of neighbours
composed of unconditional cooperators and defectors~\cite{Ohtsuki2006} and
~\cite{MASUDA2003} showed that cooperation is more likely to emerge in a
small world topology. Moreover,~\cite{Perc2011}
studied graphs were a probability of rewiring ones connections was in place,
however, the rewire could be with any given node in the graphs and not just
with imitate neighbours. Perc etc all showed that ``making of new friends'' may
be an important activity for the successful evolution of cooperation,
but also that partners must be selected carefully and one should keep their number
limited.

\begin{figure}[!hbtp]
\centering
    \begin{subfigure}{.25\textwidth}
        \includestandalone[width=\textwidth]{assets/tex/square_lattice}
    \end{subfigure}
    \begin{subfigure}{.25\textwidth}\centering
        \includestandalone[width=\textwidth]{assets/tex/square_lattice_eight}
     \end{subfigure}
     \begin{subfigure}{.25\textwidth}\centering
        \includestandalone[width=\textwidth]{assets/tex/hexagonal_lattice}
     \end{subfigure}
     \caption{Spatial neighbourhoods}\label{fig:topologies}
    \end{figure}

Other mechanisms that have studied how cooperators may interact more so that
cooperation can emerge include reputation~\cite{janssen2006, nowak1998, suzuki2005}
and communication tokens~\cite{miller2002} which are partner identification methods.
Another related approach is using a tag based partner identification~\cite{choi2006,
hales2000, miller2002, riolo2001}. Tags allow cooperative strategies to distinguish
`them' from the rest.

This section has focused on evolutionary dynamics and how different evolutionary
settings can used to shed some light to how cooperation emerges in populations.
In the following section evolutionary settings are used to train strategies.

\subsection{Structured strategies and training}
\label{section:structured_strategies}

In Section~\ref{subsection:evolutionary_dynamics} several evolutionary
dynamic approaches used in the iterated prisoner's dilemma research were covered.
All of these approaches have a limitation, and that is their inability to develop
new strategies. In evolutionary settings strategies can learn to adapt their
actions over time based upon what has been effective and what has not. In
essence evolutionary reinforcement learning techniques can be used to train strategies for playing
the iterated prisoner's dilemma. A strategy must be represented in a generic strategy
archetype so it can be trained. Strategies that are discovered via strategy archetypes
are refereed to as structured strategies. This does not include only strategies
that have been trained, but any strategy that can emerge for any given structure
that covers a range of iterated prisoner's dilemma strategies. They are the opposite
to that of intelligent designed discussed in Section~\ref{subsection:intelligent_design}.
This section covers papers that proposed new structured strategies as well as
papers that explore new training algorithms are covered.

In Axelrod,~\cite{Axelrod1987} having realised the limitations of his evolutionary
work, decided to use reinforcement learning to demonstrated how to fine tune the
responses of an iterated prisoner's dilemma strategy and obtain a trained strategy.
The algorithm used in~\cite{Axelrod1987} was the genetic algorithm~\cite{Hoffmann1998}
and the structure was that of a lookup table. 
Axelrod decided to considered deterministic strategies that took into account
the last 3 turns of the game. For each turn there are 4 possible outcomes
\((CC, CD, DC, DD)\), thus for 3 turns there are a total of \(4\times4\times4=64\)
possible combinations. Therefore, the strategy can be defined by a series of
64 C's/D's, corresponding to each history; a lookup table.
A more generic lookup definition is a deterministic strategy which take into account
the last \(m\) actions.

In 1989~\cite{nowak1989}, Nowak and Sigmund proposed a structure for studying
sophisticated strategies instead of deterministic ones. They studied
a set of very simple strategies that remember only the previous
turn, and moreover, only record the move of the opponent. They are called reactive
strategies and they can be represented by using three parameters \((y, p_1, p_2)\), where \(y\) is the
probability to cooperate in the first move, and \(p_1\) and \(p_2\) the conditional probabilities
to cooperate, given that the opponent's last move was a cooperation or a defection.
Using the above notation a strategy can now be defined by a triple. For example,
Defector: (0, 0, 0), Cooperator: (1, 1, 1), Tit for Tat: (1, 1, 0) and Pavlov: (0, 1, 1, 0).

This framework was used by the authors to study the evolution of a population
composed of 99 reactive strategies. The strategy that managed to take
over was \textbf{Generous Tit for Tat} which is given by the triplet \((1, 0, \frac{2}{3})\).
In~\cite{Nowak1990} the reactive strategies structure was expanded and a formal definition
of a \textbf{memory one strategy} was given for the first time.
Memory one strategies consider the entire history of
the previous turn to make a decision (thus reactive strategies are a subset of
memory one).

If only a single turn of the game is taken into account and depending on the
simultaneous moves of two players there are only four possible states that
players could possibly be in. These are \(CC, CD, DC\) and \(DD\). A memory one
strategy is denoted by the probabilities of cooperating after each of these states,
\( p = (p_1, p_2, p_3, p_4) \in\mathbb{R}_{[0,1]}^{4} \).
A match between two memory one players \(p\) and \(q\) can be modelled as a
stochastic process, where the players move from state to state. More specifically,
it can be modelled by the use of a Markov chain~\cite{gamerman2006markov},
which is described by a matrix \(M\).

\begin{equation}\label{eq:markov_matrix}
    M =
\begin{bmatrix}
    p_{1} q_{1} & p_{1} (- q_{1} + 1) & q_{1} (- p_{1} + 1) & (- p_{1} + 1) (- q_{1} + 1)
    \\
    p_{2} q_{3} & p_{2} (- q_{3} + 1) & q_{3} (- p_{2} + 1) & (- p_{2} + 1) (- q_{3} + 1)
    \\
    p_{3} q_{2} & p_{3} (- q_{2} + 1) & q_{2} (- p_{3} + 1) & (- p_{3} + 1) (- q_{2} + 1)
    \\
    p_{4} q_{4} & p_{4} (- q_{4} + 1) & q_{4} (- p_{4} + 1) & (- p_{4} + 1) (- q_{4} + 1)
    \\
\end{bmatrix}
\end{equation}

The players are assumed to move from each state until the system reaches a state
steady, let the steady states vector be denoted as \(\bar{v}\).
The utility of a player can be given by multiplying the steady states of
\(M\) by the payoffs of equation~(\ref{eq:the_pd_payoffs}). Thus~\cite{Nowak1990}
offered a mathematical framework to calculate the utility of two players without
actually simulating the game. The payoff of a player \(p\) can be obtained by,

\[s_p = \bar{v} \times \begin{pmatrix} R \\ S \\ T \\ P \end{pmatrix}\]

The family of memory one strategies have been proven rather useful in the terms
of exploring strategies. The most famous work of memory one strategies is that
of Press and Dyson~\cite{Press2012}.
In 2012,~\cite{Press2012} presented
a new set of strategies called \textbf{zero determinant (ZD)}. The ZD strategies
are memory one strategies that manage to force a linear relationship between their
score and that of the opponent. The payoffs of players \(p\) and \(q\) are denoted
as:

\begin{align*}
    s_p = v S_p \\
    s_q = v S_q
\end{align*}

where \(v\) is a vector of the steady states of matrix \(M\) and \(S_p\), \(S_q\)
are the equivalent payoff values of the players for each state \(CC, CD, DC, DD\).
Using linear algebra, Press and Dyson showed that the dot product of the stationary
distribution of \(v\) with any vector \(f\) can be expressed as a \(4\times 4\)
determinant. In which one column is \(f\), one column is entirely under the control
of player \(p\) and another column is entirely under the control of player \(q\).
This meant that either \(p\) or \(q\) could independently force the dot product
of \(v\) with some other chosen vector \(f\) to be zero by choosing their
strategy so as to make the column they control be proportional to \(f\).
In particular, by

\begin{equation}\label{eq:exortion}
    f = \alpha S_p + \beta S_q + \gamma
\end{equation}

any player can force a given linear relation to hold between the long-run scores
of both players. Press and Dyson's suggested that these extortionate strategies
are the dominant family of strategies and that memory does not benefit them,
thus it would not benefit any strategy.

The ZD strategies have attracted a lot of attention. It was stated that
``Press and Dyson have fundamentally changed the viewpoint on the Prisoner's
Dilemma''~\cite{Stewart2012} and as stated in~\cite{hilbe2015} the American
Mathematical Society's news section said that ``the world of game theory is currently on fire''.
In~\cite{Adami2013, Hilbe2013b, Hilbe2013, hilbe2015, KnightHGC17,
Lee2015, Stewart2012} they question the effectiveness of ZD strategies. In~\cite{Stewart2012},
they revealed a more generous set of ZDs the~\textbf{Generous ZD},~\cite{Adami2013}
showed that ZD strategies were not evolutionarily stable and
in~\cite{Lee2015}, the `memory of a strategy does not matter' statement was
questioned. A set of more complex strategies, strategies that take in account
the entire history set of the game, were trained and proven to be more stable than
ZD strategies.

Complex strategy can be explored by using different structures. In 1996,
two were introduced by~\cite{Harrald1996} and ~\cite{Miller1996}.
Harrald and Fogel~\cite{Harrald1996}, considered a neural network structure.
Their neural network used a memory length of 3 and the actions were encoded as
continuous values in \([-1, 1]\),
where 1 meant complete cooperation. The input nodes represented 3 previous steps
of the player and the opponent and there was a single hidden layer of \(N\) fully
connected nodes and an output node that produced values from the range \([-1, 1]\).
Same year~\cite{Miller1996} considered finite state automata.
The specific type of finite automata that were used were Moore machines~\cite{moore1956}.
Finite state machine consist of a set of internal states. One of these states
is the initial state of the machine. A machine also consists of transitions
arrows associated with the states. Each arrow is labelled with \(A/R\) where
\(A\) is the opponent's last action and \(R\) is the player's response.
\cite{Miller1996} used a genetic algorithm to train finite state machines in
environments with noise. His results showed that even a small difference in noise
(from 1\% to 3\%) significantly changed the characteristics of the
evolving strategies. Three machines described in his paper were \textbf{Punish Twice},
\textbf{Punish Once for Two Tats} and \textbf{Punish Twice and Wait}.


Arguably the most complex strategies that have been trained are these of~\cite{Knight2017, KnightHGC17}.
In~\cite{Knight2017, KnightHGC17} present several powerful strategies
created using training. In these
papers the authors used genetic algorithms and particle swarm optimisation
algorithms~\cite{suganthan1999}. Their selected structures included, lookerup tables,
finite state machines, artificial neural networks~\cite{yegnanarayana2009} and
hidden Markov models~\cite{eddy1996}.
Hidden Markov models, are a variant of a finite state machine that use probabilistic
transitions based on the prior round of play to other states and cooperate or
defect with various probabilities at each state.
Additionally a variant of a look up table is also presented called the lookerup
archetype. The lookerup archetype responses based on the opponent's first \(n_1\)
moves, the opponent's last \(m_1\) moves, and the players last \(m_2\) moves.
Taking into account the initial move of the opponent can give many insights.
For it is the only move a strategy is truly itself without being affected by
the other player. Finally, a new structure called the Gambler was also introduced,
which is a stochastic variant of the lookerup archetype.

The structured strategies were put up against a large number of strategies in
two following settings:

\begin{itemize}
    \item A Moran process, which is an evolutionary model of invasion and resistance across
    time during which high performing individuals are more likely to be replicated.
    \item A round robin tournament.
\end{itemize}

The authors made use of an open source package called Axelrod-Python
( paper describing it and it's capabilities was published in 2016~\cite{Knight2016}).
The package includes more than 200 implemented strategies, this includes most
the strategies covered in this review.
These experiments are, to the authors knowledge, the biggest ones done in the
field in terms of different strategies.
In~\cite{Knight2017}, they performed a standard tournament and a noisy tournament.
For the standard tournament the newly introduced trained
strategies outperform the strategies outperform all the strategies of intelligent
design.In the case of noise there is one particular strategy that has not seen much
attention in the literature called ``Desired Belief Strategy''~\cite{Au2006}.

The result of~\cite{KnightHGC17} show that the trained strategies evolve an ability
to recognise themselves by using a handshake. This characteristic of the strategies
was an important one because in a Moran process this recognition mechanism allowed these
strategies to resist invasion.

Training can return a series of strategies. Differentiating between strategies
is not always an easy task. It is not obvious looking at a finite state diagram
how a machine will behave, and many different machines, or neural networks can
represent the same strategy. For example Figure~\ref{fig:machine_tft} shows
two finite automata and both are a representation of Tit for Tat.

\begin{figure}[!hbtp]
    \centering
    \includestandalone[height=.15\textheight]{./assets/tex/tit_for_tat_fsm}
    \caption{Finite state machine representations of Tit for Tat.}\label{fig:machine_tft}
\end{figure}

In order to distinguish the strategies and assuring that they are indeed
different~\cite{Ashlock2005} introduced a method called fingerprinting.
The method of fingerprinting is a technique for generating a functional signature for a
strategy~\cite{Ashlock2008}. This is achieved by computing the score of a strategy
against a spectrum of opponents. The basic method is to play the strategy
against a probe strategy with varying noise parameters. In~\cite{Ashlock2005}
Tit for Tat is used as the probe strategy. Fingerprint functions
can then be compared to allow for easier identification of similar strategies.
In Figure~\ref{fig:fingerprinting} an example of Pavlov's fingerprint is given.
Fingerprinting has been studied in depth in~\cite{Ashlock2008, Ashlock2009,
Ashlock2010, Ashlock2006a}.

\begin{figure}[!hbtp]
    \centering
    \includegraphics[height=.3\textheight]{./assets/images/Win-Stay_Lose-Shift.png}
    \caption{Pavlov fingerprinting with Tit for Tat used as the probe strategy.
    Figure was generated using~\cite{axelrodproject}.}
    \label{fig:fingerprinting}
\end{figure}

This section covered structured strategies and training methods. In the following
section software that has been developed with main aim simulating the iterated
prisoner's dilemma is presented.

\subsection{Software}\label{section:software}

The research of the iterated prisoner's dilemma heavily relays on software.
This is to be expected as the pioneer computer tournaments have become the main
mean of simulating the interactions in an iterated prisoner's dilemma game.
Many academics fields suffer from the lack of source code availability and the prisoner's dilemma
is not an exception. Though several of the tournaments that have been discussed so far were generated
using computer code not all of the source code was made available by the authors.
The code for Axelrod's original tournament is known to be lost and
moreover for the second tournament the only source code available is the code
for the 62 strategies (found on Axelrod's personal website~\cite{fortan_code}).

Several projects, however, are open, available and have been used as research
tools or educational platforms over the years. Two research tools are briefly mentioned
here~\cite{prison, axelrodproject} and two educational tools~\cite{pd_trust, pd_game}.
Both~\cite{prison, axelrodproject} are open source projects used as research
tools. PRISON is written in the programming language Java and preliminary version
was launched on 1998. It was used by it's authors in several publications, such
as~\cite{Beaufils1997} which introduced Gradual and~\cite{Beaufils1988}. The project
includes a good number of strategies from the literature but unfortunately the
last update of the project dates back in 2004.
Axelrod-Python is a package used by several papers covered in Section~\ref{section:structured_strategies}.
It is written in the programming language Python following best practice
approaches and contains the larger to date data set of strategies, known to the
author. The strategy
list of the project has been cited by publications~\cite{Anastassacos2018,
Hayes2017, Neumann2018} and the package has been for several manuscripts
such as~\cite{Goodman2018, Wang2017}.

The `Game of Trust'~\cite{pd_trust} is an on-line, graphical user interface
educational platform for learning the basics of game theory, the iterated prisoner's
dilemma and the notion of strategies. It attracted a lot of attention due to
being ``well-presented with scribble-y hand drawn characters''~\cite{trust_blogb}
and ``a whole heap of fun''~\cite{trust_bloga}. Finally~\cite{pd_game} is a
personal project written in PHP. It's graphical user interface platform that offers
a big collection of strategies and allows the user to try several matches and
tournament configurations.

\subsection{Conclusion and Contemporary period}\label{section:contemporary_period}

This section of the paper served as a review of publications on the prisoner's
dilemma. This review has partitioned the literature into five different sections,
focusing on different aspects of the research. Section~\ref{section:origin}
covered the early years of research. This was when simulating turns of the game
was only possible with human subject research.

Following the early years, the pioneer tournaments of Axelord were introduced in
Section~\ref{subsection:intelligent_design}. Axelrod's work offered the field an
agent based game theoretic framework to study the iterated prisoner's dilemma.
In his original papers he asked researchers to design strategies to test their
performance with the new framework. The winning strategy of both his tournaments
was Tit for Tat. The strategy however came with limitations which were explored
by other researchers and new strategies of intelligent design were introduced in
order to surpass Tit for Tat with some contributions such as Pavlov and Gradual.

Soon researchers came to realise that strategies should not just do well in a tournament setting
but should also be evolutionary robust. Evolutionary dynamics methods were
applied to many works in the fields, and factors under which cooperation
emerges were explored, as described in Section~\ref{subsection:evolutionary_dynamics}.
This was not done only for unstructured populations, where all strategies
in the population interacted with each other, but also in population where
interactions were limited to only strategies that were close to each other.
In such topologies it was proven that even in the one shot game cooperation can
indeed emerge.

Evolutionary approaches can offer many insights in the study of the prisoner's
dilemma. In evolutionary settings strategies can learn to adapt and take over
population by adjusting their actions;
Such algorithms can be applied so that evolutionary robust strategies can emerge.
Algorithms and structures used to train strategies in the literature were covered in
Section~\ref{section:structured_strategies}. Several strategies can emerge from
such processes, and to be able to differentiate between strategies fingerprinting
was introduced. The research of best play and cooperation has been going on
since the 1950s, and software has been developed along the way.
Few have been briefly discussed in Section~\ref{section:software}.

The study of the prisoner's dilemma is still an ongoing field of pioneer and
innovating research, where new variants and new structures of strategies are
continuously being explored~\cite{Ohtsuki2018}. The game now serves as a model
in a wide range of applications, for example in medicine and the study of cancer
cells~\cite{archetti2018, Kaznatchee2017}, as well as in social situations and
how they can be driven by rewards~\cite{Dridi2018}. A lot of work is still being
published on  evolutionarily dynamics on graphs~\cite{Allen2017, hathcock2018,
Liu2017}.

A large scale of articles has been covered in each of the corresponding sections
of this review. This literature doe not pretend to have covered all the publications
in the field. It will soon in the following section that the field has hab many
publications, exceeding 3000 articles. However, several important milestones
of the field have been presented here.

\section{Analysing a large corpus of articles}\label{section:analysis}

The focus of Section~\ref{section:timeline} has been the academic publications on
the topic of the iterated prisoner's dilemma. Whilst in Section~\ref{section:timeline}
several publications of specific interest were covered and the literature was manually
partitioned in different sections, in the second part of this paper the publications
are analysed using a large dataset of articles. In Section~\ref{section:background}
some background
research on bibliometrics is covered. The data collection process is covered in
Section~\ref{section:data_collection} and a preliminary analysis of the data is
conducted in Section~\ref{section:preliminary_analysis}. In Section~\ref{section:methodology},
the methodology which will be used to analyse the author relationships is presented.
In summary, graph theoretical methods will be used to ascertain the level of
collaborative nature of the field and identify influence.
This type of analysis has been carried out in~\cite{Liu2015}. The novelty here
is to consider approaches not considered in~\cite{Liu2015} and new origins of
publications. A further comparison of the results are made, relative to
two other sub fields of game theory: auction games~\cite{menezes2005} and
the price of anarchy~\cite{roughgarden2005} and a temporal analysis.
Finally in Section~\ref{section:results}, the results of the analysis are presented.

\subsection{Background}\label{section:background}

As discussed in~\cite{youngblood2018}, bibliometrics or the statistical analysis
of published works (originally described by~\cite{pritchard1969}) 
have been used to support historical assumptions about the development of fields
\cite{raina1998}, identify connections between scientific growth and policy changes 
\cite{das2016}, develop a quantitative understanding of author order~\cite{sekara2018},
and investigate the collaborative structure of an interdisciplinary field~\cite{Liu2015}.
Most academic research is undertaken in the form of collaborative effort and as~\cite{Kyvik2017}
points out, it is rationale that two or more people have the potential to do better
as a group than individually. Collaboration in groups has a long tradition in experimental 
sciences and it has be proven to be productive according to~\cite{Etzkowitz1992}.
The number of collaborations can be very different between research fields and
understanding how collaborative a field is, is not always an easy task.
Several studies tend to consider academic citations as a measure for these things.
A blog post published in Nature~\cite{nature_blog} argues that depending on citations
can often be misleading because the true number of citations can not be
known. Citations can be missed due to data entry errors, academics are influenced
by many more papers than they actually cite and several of the citations are
superficial.

A more recent approach to measure collaborative behaviour is to use the co
authorship network, as described in~\cite{Liu2015}. Using this approach has many
advantages as several graph theoretic measures can be used as proxies to explain
authors relationship. In~\cite{Liu2015}, they analyse the development of the field
``evolution of cooperation'' using this approach. The topic ``evolution of cooperation''
is a multidisciplinary field which also includes a large number of publications
on the prisoner's dilemma. This paper builds upon the work done by~\cite{Liu2015}
and extends their methodology. Though in
\cite{Liu2015}, they considered a data set from a single source, Web of Science,
the data set described here has been collected from five different sources. Moreover, the collaborative
results of the analysis are compared to those of two different sub fields.
Co authorship networks have also been used in~\cite{youngblood2018} for classifying
topics of an interdisciplinary field. This was done using centrality measures,
which will be covered below, here centrality measures are used in order to understand
the influence an author can have and can receive by being part of an academic group.
Furthermore, in~\cite{alshebli2018} they look at the relationship between research
impact and five classes of diversity: ethnicity, discipline, gender, affiliation,
and academic age. These characteristics of the authors are not being captured here.
In future work these characteristics would be included in the analysis.

\subsection{Data Collection}\label{section:data_collection}

Academic articles are accessible through scholarly databases.
Several databases and collections today offer access through
an open application protocol interface (API). An API allows users to query
directly a journal's database and bypass the user interface side of the journal.
Interacting with an API has two phases: requesting and receiving.
The request phase includes composing a url with the details of the request. For
example, \url{http://export.arxiv.org/api/query?search_query=abs:prisoner's dilemma&max_results=1}
represents a request message. The first part of the request is the address
of the API.
In this example the address corresponds to the API of arXiv.
The second part of the request contains the search arguments. In this example 
it is requested that the word `prisoners dilemma' exists within the article's title.
The format of the request message is different from API to API.
The receive phase includes receiving a number of raw metadata of articles that
satisfies the request message. The raw metadata are commonly received in extensive markup
language (xml) or Javascript object notation (json) formats~\cite{nurseitov2009}.
Similarly to the request message, the structure of the received data differs from journal
to journal.

The data collection is crucial to this study. To ensure that this study can be
reproduced all code used to query the different APIs has been packaged as a Python library and is
available online~\cite{nikoleta_2017}. The software could be used for any type of
projects similar to the one described here, documentation for it is available at:
\url{http://arcas.readthedocs.io/en/latest/}.
Project~\cite{nikoleta_2017} allow users to collect articles from a list of APIs by
specifying just a single keyword. Four prominent journals in the field and a
pre print server were used as sources to collect data for this analysis:
PLOS, Nature, IEEE, Springer and arXiv.

A series of search terms were used to identify relevant articles:

\begin{itemize}
    \item ``prisoner's dilemma'',
    \item ``prisoners dilemma'',
    \item ``prisoner dilemma'',
    \item ``prisoners evolution'',
    \item ``prisoner game theory''
\end{itemize}

and articles for which any of these terms existed within the title, the abstract
or the text are included in the analysis. More specifically, 23\% of article
considered here were included because any of the above terms existed within
the abstract, 50\% within the main text and 27\% within the title.
As will be described in Section~\ref{section:preliminary_analysis}, two other
game theoretic sub fields were also considered in this work, auction games and the
price of anarchy. For collecting data on these sub fields the search terms used
were ``auction game theory'' and ``price of anarchy''. The three data sets
are archived and available at. % TODO: archive data.
Note that the latest data collection was perform on November
2018.% TODO Ensure this stay accurate

\subsection{Preliminary Analysis}\label{section:preliminary_analysis}

A summary of each of the three data sets used is presented in this section.
The three data sets are:

\begin{itemize}
    \item The main data set which contains articles on the prisoner's dilemma.
    \item A data set which contains article on auction games.
    \item A data set which contains articles on the price of anarchy.
\end{itemize}
% TODO archive all data sets

The main data set is archived at~[ref]. It
consists of \totalarticles articles with unique titles. In case of duplicates
the preprint version of an article (collected from arXiv) was dropped.
%TODO add reference to archived dataset
Of these \totalarticles article, \manual have not been collected from the
aforementioned APIs. These articles were of specific interest and manually added
to the dataset throughout the writing of Section~\ref{section:timeline}. A
similar approach was used in~\cite{Liu2015} where a number of articles of interest
where manually added to the data set. A more
detailed summary of the articles' provenance is given by Table~\ref{table:provenance}.
Only 3\% of the data set consists of articles that were manually added and 33\% of the
articles were collected from arXiv.

\begin{table}[!hbtp]
    \begin{center}
    \begin{tabular}{lrr}
\toprule
{} &  \# of Articles &  Percentage \\
provenance &                &             \\
\midrule
Manual     &             89 &        2.88 \\
IEEE       &            295 &        9.55 \\
PLOS       &            482 &       15.60 \\
Springer   &            572 &       18.52 \\
Nature     &            673 &       21.79 \\
arXiv      &           1056 &       34.19 \\
\bottomrule
\end{tabular}

    \end{center}
    \caption{Articles' provenance for the main data set.} % Use citation when archived
    \label{table:provenance}
\end{table}

The average number of publications was calculated for the entire dataset and for
each provenance. The average number of publications is denoted as,
\(\mu_P = \frac{N_A}{N_Y},\) where \(N_A\) is the total number of articles and
\(N_Y\) is the years of publication. The years of publication is calculated as
the range between the collection date and the first published article, for each provenance,
within the data.
These averages are summarised in Table~\ref{table:publication_rates}.
Overall an average of 49 articles are published per year on the topic. The most
significant contribution to this appears to be from arXiv with 16 articles per year,
followed by Nature with 10 and Springer with 9.

\begin{table}[!hbtp]
    \begin{center}
    \begin{tabular}{lr}
\toprule
{} &  Average Yearly publication \\
\midrule
IEEE     &                         5.0 \\
PLOS     &                         8.0 \\
Springer &                         9.0 \\
Nature   &                        11.0 \\
arXiv    &                        16.0 \\
Overall  &                        49.0 \\
\bottomrule
\end{tabular}

    \end{center}
    \caption{Average publication for main data set.} % Use citation when archived
    \label{table:publication_rates}
\end{table}

Though the average publication offers insights about the publications of the
fields, it remains a constant number. The data handled here is a time
series between 1950, when the game was introduced, and 2018 (Figure~\ref{fig:timeseries}). 
Two observations can be made from Figure~\ref{fig:timeseries}.

\begin{enumerate}
    \item There is a steady increase to the number of publications since the 1980s
    and the introduction of computer tournaments.
    \item There is a decrease in 2017-2018. This is due to our data set being incomplete.
    Articles that have been written in 2017-2018 have either not being published
    or have are not retrievable by the APIs yet.
\end{enumerate}

These observations can be confirmed by studying the time series.
Using~\cite{scipy}, an exponential distribution is fitted to
the data from 1980-2016. The exponential fitting proves that since 1980 there has
been an increase in the number of publications till 2016 (Figure~\ref{fig:fitting}).
The fitted model can also be used to project the behaviour of the field for the next 
5 years. The forecasted periods are plotted in Figure~\ref{fig:forecasting} and
their exact values are given by Table~\ref{table:forecast}. The time series
has indicated a slight decrease however the model forecasts that the number
of publications will keep increasing, thus indicating that the field of the iterated
prisoner's dilemma still attracts academic attention.

\begin{figure}[!hbtp]
\begin{minipage}{.45\textwidth}
    \centering
    \includegraphics[width=.9\textwidth]{./assets/images/timeline.pdf}
    \caption{Line plot; \# of articles published on the PD 1950-2019.}\label{fig:timeseries}
\end{minipage}
\begin{minipage}{.45\textwidth}
    \centering
    \includegraphics[width=.9\textwidth]{./assets/images/fitting.pdf}
    \caption{Scatter plot; \# of articles published on the PD 1980-2019.}\label{fig:fitting}
\end{minipage}
\end{figure}

\begin{figure}[!hbtp]
    \centering
    \includegraphics[width=.5\textwidth]{./assets/images/forecasting.pdf}
    \caption{Forecast for 2017-2022.}\label{fig:forecasting}
\end{figure}

\begin{table}[!hbtp]
    \begin{center}
    \begin{tabular}{lr}
\toprule
{} &  Forecast \\
\midrule
2017 &     371.0 \\
2018 &     423.0 \\
2019 &     483.0 \\
2020 &     550.0 \\
2021 &     627.0 \\
\bottomrule
\end{tabular}

    \end{center}
    \caption{Forecasting the number of publications over the next 10 years.}
    \label{table:forecast}
\end{table}

To allow for a comparative analysis two sub fields of game theory have been chosen
for this work; auction games and the price of anarchy.

\begin{itemize}
    \item Auction theory is a branch of game theory which researches the properties
    of auction markets.
    Game theory is used for years to study auctions and the behaviour of
    bidders~\cite{Shubik1971}. The earliest entry in our data set~[ref] goes
    back to 1974 (Figure~\ref{fig:timeseries_ag}). Note that no articles
    have been added manually for auction games.
    \item Price of Anarchy is a concept in game theory which
    measures how the efficiency of a system degrades due to selfish behaviour of
    it's agents. There is a variety of such measures however the price of anarchy
    has attracted a lot of attention since it's informal introduction in 1999
    by~\cite{Koutsoupias1999}. Note that~\cite{Koutsoupias1999} has been manually
    added to the date set and it's its first entry (Figure~\ref{fig:timeseries_pa}).
\end{itemize}

A summary of both data sets collected on both topics, in comparison to that of
[ref], is given by Table~\ref{table:summary_other_topics}.

\begin{table}[!hbtp]
    \centering
    \resizebox{\textwidth}{!}{
    \begin{tabular}{lrrllrrrrr}
\toprule
{} &  Num. Articles &  Num. Authors & Manual (\%) & PLOS (\%) &  Nature (\%) &  Springer (\%) &  IEEE (\%) &  arXiv (\%) &  Av. Yearly Publication \\
\midrule
Prisoner's Dilemma &           3077 &          5772 &        2.5 &    15.66 &       21.87 &         18.59 &      9.59 &      34.32 &                    49.0 \\
Auction Games      &           3444 &          5362 &          - &        - &        5.89 &         37.63 &      7.46 &      51.36 &                    93.0 \\
Price of Anarchy   &            748 &          1316 &       0.13 &     1.74 &       24.73 &         37.97 &     30.61 &       8.82 &                    39.0 \\
\bottomrule
\end{tabular}
}
    \caption{Measures of all three data sets.}\label{table:summary_other_topics}
\end{table}

The iterated prisoner's dilemma and auction theory are very well studied topics
that have been publicising for decades. A large number of articles have
been collected for both topics, \totalarticles and 34444 respectively. Though, auction
games have a larger number of articles, the iterated prisoner's dilemma
has almost 300 more authors.

Auction games have an overall average yearly publication
of 93 articles per year compared to the prisoner's dilemma with 49 per year. 50\% of articles
for [ref] have been collected from the pre print server arXiv and no articles have
been published in PLOS.

Compared to these two topics the price of anarchy is a fairly recent one. Only a
total of 747 articles have been collected, however it has a large number
of 1229 authors. On average each paper had had at least two authors.
It has an overall average publication of 39 articles and the biggest contribution
has been made to Springer.

\begin{figure}[!hbtp]
    \begin{minipage}{.45\textwidth}
        \centering
        \includegraphics[width=\textwidth]{./assets/images/Auction_Games.pdf}
        \caption{Line plot; \# articles published on auction games 1974-2018.}\label{fig:timeseries_ag}
    \end{minipage}%
    \begin{minipage}{.45\textwidth}
        \centering
        \includegraphics[width=\textwidth]{./assets/images/Price_of_Anarchy.pdf}
        \caption{Line plot; \# articles published on the price of anarchy.}\label{fig:timeseries_pa}
    \end{minipage}
    \end{figure}

\subsection{Methodology}\label{section:methodology}

The relationship between the authors within a field will be modelled as a graph \(G\) with
a set \(V_G\) of nodes and \(E_G\) of edges. The set \(V_G\) represents the authors
and an edge connects two authors if and only if those authors have written together.
The co authorship network is constructed using the main data set described in
Section~\ref{section:preliminary_analysis} and the open source package
\cite{networkx}. The prisoner's dilemma network is denoted as \(G_1\) where the
number of unique authors \(|V(G_1)|\) is \authors and \(|E(G_1)|\) is \edges.
All authors' names were formatted as their last name and
first initial (i.e. Martin A. Nowak to Martin Nowak). This was done to avoid errors
such as Martin A. Nowak and Martin Nowak, being treated as a different person.

Collaborativeness, will be analysed using measures such as, isolated nodes,
connected components, clustering coefficient, communities, modularity and average degree.
These measures show the number of connections authors can have
and how strongly connected these people are. The number of isolated nodes is the
number of nodes that are not connected to another node, thus the
number of authors that have published alone. The average degree denotes the average
number of neighbours for each nodes, i.e. the average number of collaborations
between the authors.

A connected component is a maximal set of nodes such that each pair of nodes is
connected by a path. The number of connected components as well as the size of the
largest connected component in the network are reported.
The size of the largest connected component represents the scale of the central cluster
of the entire network, as it will discussed in the analysis section.
Clustering coefficient and modularity and are also calculated. The clustering
coefficient, defined as 3 times the number of triangle on the graph divided
by the number of connected triples of nodes, is a local measure of the degree to
which nodes in a graph tend to cluster together
in a clique. It is precisely the probability that the collaborators
of an author also write together.

In comparison, modularity is a global measure designed to measure the strength of
division of a network into communities. The number of communities will be reported
using the Clauset-Newman-Moore method~\cite{clauset2004}. Also the modularity index
is calculated using the Louvain method described in~\cite{Blondel2008}. The value
of the modularity index can vary between \([-1, 1]\), a high value of modularity
corresponds to a structure where there are dense connections between the nodes within
communities but sparse connections between nodes in different communities.
That means that authors in the network are mainly connected co-authors that they
all have written together, and not to several different collaborators.

Networks are commonly dominated by one person who controls information flow and
people that receive a great amount of information due to their position.
Two further points are aimed to be explored in this work, (1) which people control the flow;
as in which people influence the field the most and (2) which are the authors that
gain the most from the influence of the field. To measure these concepts graph
theoretic metrics, more specifically centrality measures are going to be used.
Centrality measures are often used to understand different
aspects fo social networks~\cite{Landherr2010}. Two centrality measures have been
chosen for this paper and these are closeness and betweenness centrality.

\begin{enumerate}
    \item In networks some nodes have a short distance to other nodes and consequently
    are able to spread information on the network very effectively.
    A representative of this idea is \textbf{closeness centrality}, where a person is seen
    as centrally involved in the network if it requires only few intermediaries
    to contact others and thus is structurally relatively independent. Here,
    this is defined as influence. Authors with a high value of closeness centrality,
    are the authors that spread scientific knowledge easier on the network
    and they have high influence.
    \item Another centrality measure is the \textbf{betweenness centrality},
    where the determination of an author's centrality is based on the quotient of
    the number of all shortest paths between nodes in the network that include
    the regarded node and the number of all shortest paths in the network.
    In betweenness centrality the position of the node matters. Nodes
    with a higher value of betweenness centrality are located in positions that
    a lot of information pass through them, this is defined as the gain from
    the influence, thus these authors gain the most from their networks.
\end{enumerate}

In the next section all the metrics discussed here are calculated for the data
sets in order to provide insights into the field.

\subsection{Analysis of co authorship network}\label{section:results}

As mentioned previously, \(G_1\) denotes the co authorship network of the iterated
prisoner's dilemma. It's graphical representation is given by Figure~\ref{fig:g_one_network}.
It is evident that the network is disjoint, which is only natural as many authors
write academic articles on their own. More specifically, a total of \isolated authors,
have had single author publications, which corresponds to the \isolatedpercentage (\%)
of authors in \(G_1\).

There are a total of \connectedcomponents connected components and the largest one
has a size of \largestcc nodes. The largest connected component is shown in Figure~\ref{fig:g_one_cluster}
and is going to be refereed to as the main cluster of the network.
There are total of \communities communities in \(G_1\). The network has a clustering
coefficient of \clustering,
thus authors are 70\% likely to write with a collaborator's co author and the degree
distribution, Figure~\ref{fig:degree_distrs}, shows that the average degree is \(\approx4\).
Thus authors are on average connected to 4 other authors, however there are authors
with far more connections, the largest one being 58.

In~\cite{Liu2015} the collaborative metrics for the ``evolution of cooperation''
co authorship network were reported. Though their network is of smaller size
(number of nodes 3670 \(<\) 5394), the collaborative metrics are fairly similar
between the two graphs (clustering coeff. \(0.632\approx0.708\) and modularity
\(0.950\approx0.977\)), indicating that for the same multidisciplinary field the same
remarks can be made from a different co authors network. But how does these compare to other fields
and more specifically to other fields of game theory? The representation of the
two graphs,

\begin{itemize}
    \item \(G_2\) for auction games and
    \item \(G_3\) for the price of anarchy,
\end{itemize}

are given by Figures~\ref{fig:g_two} and~\ref{fig:g_three} and their respective
clusters in Figures~\ref{fig:g_two_cluster} and \ref{fig:g_three_cluster}.
As stated before \(G_3\) is the smallest network of all three, this is also clear
from it's graphical representation. The \(G_2\) network appears to be
very similar to \(G_1\), however it's main cluster is larger in size.

\begin{figure}[!hbtp]
    \begin{subfigure}{.45\textwidth}\centering
        \includegraphics[width=.54\textwidth]{./assets/pd_network.pdf}
        \caption{\(G_1\) network.}\label{fig:g_one_network}
    \end{subfigure}
    \begin{subfigure}{.45\textwidth}\centering
        \includegraphics[width=.54\textwidth]{./assets/pd_network_cluster.pdf}
        \caption{\(G_1\) larger connected component.}\label{fig:g_one_cluster}
     \end{subfigure}
     
     \begin{subfigure}{.45\textwidth}\centering
        \includegraphics[width=.54\textwidth]{./assets/auction.pdf}
        \caption{\(G_2\) network.}\label{fig:g_two}
     \end{subfigure}
    \begin{subfigure}{.45\textwidth}\centering
        \includegraphics[width=.54\textwidth]{./assets/auction_network_cluster.pdf}
        \caption{\(G_2\) larger connected component.}\label{fig:g_two_cluster}
    \end{subfigure}

    \begin{subfigure}{.45\textwidth}\centering
        \includegraphics[width=.54\textwidth]{./assets/anarchy_network.pdf}
        \caption{\(G_3\) network.}\label{fig:g_three}
     \end{subfigure}
     \begin{subfigure}{.45\textwidth}\centering
        \includegraphics[width=.54\textwidth]{./assets/anarchy_network_cluster.pdf}
        \caption{\(G_3\) larger connected component.}\label{fig:g_three_cluster}
     \end{subfigure}
     \caption{Graphical representations of \(G_1, G_2, G_3\) and their respective
     main clusters.}
\end{figure}

A summary of the collaborative metrics for all three co authorship networks is given by
Table~\ref{table:summary_other_networks} and shown in Figure~\ref{fig:degree_distrs}
are the degree distributions of all three networks. In \(G_1\) and \(G_2\) there
are cases of high degree (\(> 20\)) but this could be an affect of the size of
the data, networks and subsequently the size of the main clusters.

\begin{table}[!hbtp]
    \centering
    \resizebox{\textwidth}{!}{
    \begin{tabular}{lrrrrrrrrrr}
\toprule
{} &  \# Nodes &  \# Edges &  \# Isolated nodes &  \% Isolated nodes &  \# Connected components &  Size of largest component &  Av. degree &  \# Communities &  Modularity &  Clustering coeff \\
\midrule
Prisoner's Dilemma &     5394 &    10397 &               176 &               3.3 &                    1356 &                        815 &       3.855 &           1369 &       0.977 &             0.708 \\
Auction Games      &     5165 &     7861 &               256 &               5.0 &                    1272 &                       1348 &       3.044 &           1294 &       0.958 &             0.622 \\
Price of Anarchy   &     1155 &     1953 &                 4 &               0.3 &                     245 &                        222 &       3.382 &            253 &       0.965 &             0.712 \\
\bottomrule
\end{tabular}
}
    \caption{Network metrics for \(G_1, G_2, G_3\).}\label{table:summary_other_networks}
\end{table}

\begin{figure}[!hbtp]
    \centering
    \includegraphics[width=\textwidth]{./assets/images/networks_ditributions.pdf}
    \caption{Degree distribution for networks \(G_1, G_2\) and \(G_3\). The descriptive
    statistics for each of the distribution are for \(G_1\): mean \(=3.85\),
    median \(=3\), std \(=4.25\). For \(G_2\): mean \(=3.04\), median \(=2\),
    std \(=3.01\) and for \(G_3\): mean \(=3.38\), median \(=3\), std \(=2.90\).
    It is clear that these distributions are not normally distributed, which has
    also been verified using a statistical test. Moreover, the statistical difference
    of the medians has been tested using a Kruskal Wallis test. The medians
    of \(G_1\) and \(G_3\) are not significantly different, however they are
    are significantly large than that of \(G_2\).}\label{fig:degree_distrs}
\end{figure}

Regarding the three sub fields of game theory and using Table~\ref{table:summary_other_networks}
and Figure~\ref{fig:degree_distrs} the following remarks can be made:

\begin{itemize}
    \item All three networks have similar values of modularity index, and they are
    all very high (Table~\ref{table:summary_other_networks}), indicating that the
    networks are partitioned in many communities.
    Note that the number of
    communities is very much similar to the number of connected components.
    This is all to expected. Due to the nature of our network, most connected
    components represent a single publication written by all the authors in the
    component (corresponding to a fully connected graph), and due to that density
    they are also a community on their own.
    \item Comparing to another well studied topic, auction games, the field of the
    iterated prisoner's dilemma appear to be more collaborative. Due to the
    value of the average degree, authors in \(G_1\) are known to have on average almost
    one more collaboration than \(G_2\). A slightly lower cluster coefficient ($ .622 < .702$)
    of auction games indicate that is less likely for authors in \(G_2\) to collaborate
    with a collaborators co author.
    \item Regarding the price of anarchy, the measures indicate that the field
    is not as mature as the other two sub fields. There are no isolated authors,
    which is more of an indication of the time the field has been active. As a
    more recent field there had been better communication tools that enable more
    collaborations between researches. The average degree as well as the clustering
    coefficient (clustering coeff.$=0.713$) of \(G_3\) is comparable to those
    of the iterated prisoner's dilemma.
\end{itemize}

These results can be extend to the main clusters of each network, Table~\ref{table:summary_clusters}.
The metrics's values are fairly similar and the size of \(G_2\)'s main cluster
does not appear to gave any significant effect; all the same conclusions are made.
Compared to auction games the iterated prisoner's
dilemma is a more collaborative field, and it is fairly similar to the price of
anarchy. However, our analysis suggests that the price of anarchy is still a
maturing field.

\begin{table}[!hbtp]
    \centering
    \resizebox{\textwidth}{!}{
    \begin{tabular}{lrrrrrrrrr}
\toprule
{} &  \# Nodes &  \# Edges &  \# Isolated nodes &  \% Isolated nodes &  \# Connected components &  Size of largest component &  Av. degree &  Modularity &  Clustering coeff \\
\midrule
Prisoner's Dilemma &      815 &     2300 &                 0 &               0.0 &                       1 &                        815 &       5.644 &       0.857 &             0.775 \\
Auction Games      &     1348 &     3158 &                 0 &               0.0 &                       1 &                       1348 &       4.685 &       0.858 &             0.699 \\
Price of Anarchy   &      221 &      520 &                 0 &               0.0 &                       1 &                        221 &       4.706 &       0.818 &             0.714 \\
\bottomrule
\end{tabular}
}
    \caption{Network metrics for \(G_1, G_2, G_3\).}\label{table:summary_clusters}
\end{table}

The change of the networks over time is also studied by constructing the network
cumulatively with a year interval. A total of 64 sub graphs
over 64 periods, starting in 1950, were created and all the collaborative metrics
for each sub graph have been calculated. Note that years 1952 and 1953 have no
publications in our data set. The metrics of each network for each period are given by
Table~\ref{table:coll_cumulative}.
Similar to the results of~\cite{Liu2015}, it can been observed that the network \(G_1\)
grows over time and that the network always had a high value of modularity.

\begin{table}[!hbtp]
    \centering
    \begin{adjustbox}{totalheight=.8\textheight-2\baselineskip, width=\textwidth}
    \begin{tabular}{lrrrrrrl}
\toprule
{} &  \# Connected Components &  \# Isolated &  \% Isolated &  Av. Degree &  Clustering &  Largest cc & Modularity \\
\midrule
Period 0  &                       3 &           3 &        1.00 &        0.00 &        0.00 &           1 &          - \\
Period 1  &                       2 &           2 &        1.00 &        0.00 &        0.00 &           1 &          - \\
Period 2  &                       3 &           3 &        1.00 &        0.00 &        0.00 &           1 &          - \\
Period 3  &                       4 &           4 &        1.00 &        0.00 &        0.00 &           1 &          - \\
Period 4  &                       6 &           6 &        1.00 &        0.00 &        0.00 &           1 &          - \\
Period 5  &                       7 &           7 &        1.00 &        0.00 &        0.00 &           1 &          - \\
Period 6  &                       7 &           7 &        1.00 &        0.00 &        0.00 &           1 &          - \\
Period 7  &                       8 &           8 &        1.00 &        0.00 &        0.00 &           1 &          - \\
Period 8  &                       9 &           9 &        1.00 &        0.00 &        0.00 &           1 &          - \\
Period 9  &                      10 &          10 &        1.00 &        0.00 &        0.00 &           1 &          - \\
Period 10 &                      12 &          10 &        0.71 &        0.29 &        0.00 &           2 &        0.5 \\
Period 11 &                      15 &          12 &        0.67 &        0.33 &        0.00 &           2 &   0.666667 \\
Period 12 &                      19 &          13 &        0.46 &        0.93 &        0.16 &           5 &   0.591716 \\
Period 13 &                      21 &          15 &        0.48 &        0.97 &        0.16 &           6 &   0.533333 \\
Period 14 &                      23 &          16 &        0.47 &        0.94 &        0.15 &           6 &   0.585938 \\
Period 15 &                      26 &          16 &        0.38 &        1.10 &        0.26 &           6 &   0.763705 \\
Period 16 &                      27 &          16 &        0.36 &        1.16 &        0.31 &           6 &   0.801775 \\
Period 17 &                      29 &          17 &        0.35 &        1.12 &        0.29 &           6 &   0.814815 \\
Period 18 &                      29 &          17 &        0.35 &        1.12 &        0.29 &           6 &   0.814815 \\
Period 19 &                      30 &          18 &        0.37 &        1.10 &        0.29 &           6 &   0.814815 \\
Period 20 &                      30 &          18 &        0.37 &        1.10 &        0.29 &           6 &   0.814815 \\
Period 21 &                      33 &          19 &        0.35 &        1.07 &        0.26 &           6 &   0.837099 \\
Period 22 &                      34 &          19 &        0.34 &        1.07 &        0.25 &           6 &   0.846667 \\
Period 23 &                      36 &          21 &        0.34 &        1.11 &        0.25 &           6 &   0.854671 \\
Period 24 &                      37 &          22 &        0.34 &        1.16 &        0.28 &           6 &   0.866326 \\
Period 25 &                      37 &          22 &        0.33 &        1.27 &        0.29 &           6 &    0.85941 \\
Period 26 &                      40 &          24 &        0.34 &        1.27 &        0.32 &           6 &   0.873086 \\
Period 27 &                      43 &          26 &        0.35 &        1.23 &        0.30 &           6 &   0.878072 \\
Period 28 &                      46 &          28 &        0.35 &        1.19 &        0.28 &           6 &   0.882752 \\
Period 29 &                      46 &          28 &        0.35 &        1.19 &        0.28 &           6 &   0.882752 \\
Period 30 &                      54 &          35 &        0.40 &        1.09 &        0.26 &           6 &   0.887153 \\
Period 31 &                      58 &          38 &        0.41 &        1.05 &        0.24 &           6 &   0.891295 \\
Period 32 &                      61 &          39 &        0.39 &        1.07 &        0.26 &           6 &   0.903524 \\
Period 33 &                      70 &          43 &        0.37 &        1.10 &        0.26 &           6 &   0.921643 \\
Period 34 &                      75 &          45 &        0.36 &        1.10 &        0.26 &           6 &   0.930363 \\
Period 35 &                      84 &          50 &        0.36 &        1.07 &        0.26 &           6 &   0.939007 \\
Period 36 &                      87 &          52 &        0.36 &        1.07 &        0.25 &           6 &   0.942486 \\
Period 37 &                      95 &          54 &        0.34 &        1.12 &        0.26 &           6 &   0.951852 \\
Period 38 &                     110 &          62 &        0.32 &        1.24 &        0.33 &           6 &    0.95996 \\
Period 39 &                     118 &          64 &        0.31 &        1.23 &        0.31 &           6 &   0.964111 \\
Period 40 &                     127 &          68 &        0.30 &        1.30 &        0.32 &           6 &   0.966357 \\
Period 41 &                     137 &          71 &        0.29 &        1.31 &        0.34 &           6 &   0.970812 \\
Period 42 &                     138 &          55 &        0.20 &        1.42 &        0.36 &           6 &   0.976778 \\
Period 43 &                     139 &          49 &        0.17 &        1.50 &        0.38 &           6 &   0.979126 \\
Period 44 &                     156 &          54 &        0.16 &        1.53 &        0.40 &           6 &   0.981765 \\
Period 45 &                     157 &          42 &        0.12 &        1.67 &        0.42 &           9 &   0.980204 \\
Period 46 &                     178 &          48 &        0.12 &        1.70 &        0.43 &           9 &   0.982271 \\
Period 47 &                     208 &          54 &        0.11 &        1.66 &        0.43 &           9 &   0.985315 \\
Period 48 &                     230 &          54 &        0.10 &        1.73 &        0.44 &          10 &   0.986425 \\
Period 49 &                     256 &          54 &        0.09 &        1.82 &        0.47 &          20 &   0.985488 \\
Period 50 &                     309 &          62 &        0.08 &        1.94 &        0.50 &          22 &   0.988459 \\
Period 51 &                     361 &          72 &        0.07 &        2.18 &        0.53 &          26 &   0.986283 \\
Period 52 &                     404 &          82 &        0.07 &        2.40 &        0.55 &          40 &   0.984604 \\
Period 53 &                     451 &          94 &        0.07 &        2.50 &        0.55 &          70 &   0.979801 \\
Period 54 &                     503 &         106 &        0.07 &        2.81 &        0.56 &         195 &   0.965391 \\
Period 55 &                     570 &         116 &        0.06 &        3.00 &        0.59 &         259 &   0.964951 \\
Period 56 &                     637 &         120 &        0.05 &        3.16 &        0.62 &         377 &   0.960718 \\
Period 57 &                     700 &         131 &        0.05 &        3.30 &        0.63 &         498 &   0.956249 \\
Period 58 &                     769 &         139 &        0.05 &        3.35 &        0.64 &         651 &   0.948894 \\
Period 59 &                     857 &         148 &        0.04 &        3.52 &        0.65 &         845 &   0.943237 \\
Period 60 &                     935 &         155 &        0.04 &        3.98 &        0.67 &        1116 &   0.939336 \\
Period 61 &                     978 &         157 &        0.04 &        4.04 &        0.68 &        1253 &   0.938227 \\
Period 62 &                    1029 &         157 &        0.03 &        4.19 &        0.68 &        1456 &   0.929681 \\
\bottomrule
\end{tabular}
}
    \caption{Collaborativeness metrics for cumulative graphs.}\label{table:coll_cumulative}
\end{adjustbox}
\end{table}

\begin{table}[!hbtp]
    \centering
    \begin{adjustbox}{totalheight=\textheight-2\baselineskip, width=\textwidth}
    \begin{tabular}{lrrrrrrrlr}
\toprule
{} &  \# Nodes &  \# Edges &  \# Isolated nodes &  \% Isolated nodes &  \# Connected components &  Size of largest component &  Av. degree & Modularity &  Clustering coeff \\
\midrule
1954 - 1950 &        1 &        0 &                 1 &             100.0 &                       1 &                          1 &       0.000 &          - &             0.000 \\
1954 - 1955 &        1 &        0 &                 1 &             100.0 &                       1 &                          1 &       0.000 &          - &             0.000 \\
1955 - 1956 &        1 &        0 &                 1 &             100.0 &                       1 &                          1 &       0.000 &          - &             0.000 \\
1956 - 1957 &        1 &        0 &                 1 &             100.0 &                       1 &                          1 &       0.000 &          - &             0.000 \\
1957 - 1958 &        1 &        0 &                 1 &             100.0 &                       1 &                          1 &       0.000 &          - &             0.000 \\
1958 - 1959 &        1 &        0 &                 1 &             100.0 &                       1 &                          1 &       0.000 &          - &             0.000 \\
1959 - 1961 &        1 &        0 &                 1 &             100.0 &                       1 &                          1 &       0.000 &          - &             0.000 \\
1961 - 1962 &        1 &        0 &                 1 &             100.0 &                       1 &                          1 &       0.000 &          - &             0.000 \\
1962 - 1964 &        1 &        0 &                 1 &             100.0 &                       1 &                          1 &       0.000 &          - &             0.000 \\
1964 - 1965 &        1 &        0 &                 1 &             100.0 &                       1 &                          1 &       0.000 &          - &             0.000 \\
1965 - 1966 &        2 &        1 &                 0 &               0.0 &                       1 &                          2 &       1.000 &          0 &             0.000 \\
1966 - 1967 &        2 &        1 &                 0 &               0.0 &                       1 &                          2 &       1.000 &          0 &             0.000 \\
1967 - 1968 &        5 &        8 &                 0 &               0.0 &                       1 &                          5 &       3.200 &          0 &             0.867 \\
1968 - 1969 &        6 &       10 &                 0 &               0.0 &                       1 &                          6 &       3.333 &       0.02 &             0.833 \\
1969 - 1970 &        6 &       10 &                 0 &               0.0 &                       1 &                          6 &       3.333 &       0.02 &             0.833 \\
1970 - 1971 &        6 &       10 &                 0 &               0.0 &                       1 &                          6 &       3.333 &       0.02 &             0.833 \\
1971 - 1972 &        6 &       10 &                 0 &               0.0 &                       1 &                          6 &       3.333 &       0.02 &             0.833 \\
1972 - 1973 &        6 &       10 &                 0 &               0.0 &                       1 &                          6 &       3.333 &       0.02 &             0.833 \\
1973 - 1974 &        6 &       10 &                 0 &               0.0 &                       1 &                          6 &       3.333 &       0.02 &             0.833 \\
1974 - 1975 &        6 &       10 &                 0 &               0.0 &                       1 &                          6 &       3.333 &       0.02 &             0.833 \\
1975 - 1976 &        6 &       10 &                 0 &               0.0 &                       1 &                          6 &       3.333 &       0.02 &             0.833 \\
1976 - 1977 &        6 &       10 &                 0 &               0.0 &                       1 &                          6 &       3.333 &       0.02 &             0.833 \\
1977 - 1978 &        6 &       10 &                 0 &               0.0 &                       1 &                          6 &       3.333 &       0.02 &             0.833 \\
1978 - 1979 &        6 &       10 &                 0 &               0.0 &                       1 &                          6 &       3.333 &       0.02 &             0.833 \\
1979 - 1980 &        6 &       10 &                 0 &               0.0 &                       1 &                          6 &       3.333 &       0.02 &             0.833 \\
1980 - 1981 &        6 &        9 &                 0 &               0.0 &                       1 &                          6 &       3.000 &  0.0493827 &             0.678 \\
1981 - 1982 &        6 &        9 &                 0 &               0.0 &                       1 &                          6 &       3.000 &  0.0493827 &             0.678 \\
1982 - 1983 &        6 &       10 &                 0 &               0.0 &                       1 &                          6 &       3.333 &       0.02 &             0.833 \\
1983 - 1984 &        6 &        9 &                 0 &               0.0 &                       1 &                          6 &       3.000 &  0.0493827 &             0.678 \\
1984 - 1985 &        6 &        9 &                 0 &               0.0 &                       1 &                          6 &       3.000 &  0.0493827 &             0.678 \\
1985 - 1986 &        6 &       10 &                 0 &               0.0 &                       1 &                          6 &       3.333 &       0.02 &             0.833 \\
1986 - 1987 &        6 &        9 &                 0 &               0.0 &                       1 &                          6 &       3.000 &  0.0493827 &             0.678 \\
1987 - 1988 &        6 &       10 &                 0 &               0.0 &                       1 &                          6 &       3.333 &       0.02 &             0.833 \\
1988 - 1989 &        6 &       10 &                 0 &               0.0 &                       1 &                          6 &       3.333 &       0.02 &             0.833 \\
1989 - 1990 &        6 &        9 &                 0 &               0.0 &                       1 &                          6 &       3.000 &  0.0493827 &             0.678 \\
1990 - 1991 &        6 &        9 &                 0 &               0.0 &                       1 &                          6 &       3.000 &  0.0493827 &             0.678 \\
1991 - 1992 &        6 &        9 &                 0 &               0.0 &                       1 &                          6 &       3.000 &  0.0493827 &             0.678 \\
1992 - 1993 &        6 &        9 &                 0 &               0.0 &                       1 &                          6 &       3.000 &  0.0493827 &             0.678 \\
1993 - 1994 &        6 &        9 &                 0 &               0.0 &                       1 &                          6 &       3.000 &  0.0493827 &             0.678 \\
1994 - 1995 &        6 &        9 &                 0 &               0.0 &                       1 &                          6 &       3.000 &  0.0493827 &             0.678 \\
1995 - 1996 &        6 &        9 &                 0 &               0.0 &                       1 &                          6 &       3.000 &  0.0493827 &             0.678 \\
1996 - 1997 &        6 &        9 &                 0 &               0.0 &                       1 &                          6 &       3.000 &  0.0493827 &             0.678 \\
1997 - 1998 &        6 &        9 &                 0 &               0.0 &                       1 &                          6 &       3.000 &  0.0493827 &             0.678 \\
1998 - 1999 &        6 &        9 &                 0 &               0.0 &                       1 &                          6 &       3.000 &  0.0493827 &             0.678 \\
1999 - 2000 &        6 &       10 &                 0 &               0.0 &                       1 &                          6 &       3.333 &       0.02 &             0.833 \\
2000 - 2001 &        7 &       21 &                 0 &               0.0 &                       1 &                          7 &       6.000 &          0 &             1.000 \\
2001 - 2002 &        7 &       21 &                 0 &               0.0 &                       1 &                          7 &       6.000 &          0 &             1.000 \\
2002 - 2003 &        7 &       21 &                 0 &               0.0 &                       1 &                          7 &       6.000 &          0 &             1.000 \\
2003 - 2004 &       10 &       13 &                 0 &               0.0 &                       1 &                         10 &       2.600 &    0.37574 &             0.553 \\
2004 - 2005 &       19 &       28 &                 0 &               0.0 &                       1 &                         19 &       2.947 &   0.544005 &             0.730 \\
2005 - 2006 &       21 &       32 &                 0 &               0.0 &                       1 &                         21 &       3.048 &   0.530273 &             0.713 \\
2006 - 2007 &       24 &       36 &                 0 &               0.0 &                       1 &                         24 &       3.000 &   0.533179 &             0.678 \\
2007 - 2008 &       32 &       59 &                 0 &               0.0 &                       1 &                         32 &       3.688 &   0.627837 &             0.732 \\
2008 - 2009 &       56 &      102 &                 0 &               0.0 &                       1 &                         56 &       3.643 &   0.716792 &             0.699 \\
2009 - 2010 &       99 &      238 &                 0 &               0.0 &                       1 &                         99 &       4.808 &   0.781539 &             0.734 \\
2010 - 2011 &      121 &      288 &                 0 &               0.0 &                       1 &                        121 &       4.760 &   0.776054 &             0.713 \\
2011 - 2012 &      210 &      610 &                 0 &               0.0 &                       1 &                        210 &       5.810 &     0.7809 &             0.747 \\
2012 - 2013 &      330 &      908 &                 0 &               0.0 &                       1 &                        330 &       5.503 &    0.81959 &             0.753 \\
2013 - 2014 &      406 &     1125 &                 0 &               0.0 &                       1 &                        406 &       5.542 &    0.81494 &             0.749 \\
2014 - 2015 &      514 &     1390 &                 0 &               0.0 &                       1 &                        514 &       5.409 &   0.827358 &             0.757 \\
2015 - 2016 &      614 &     1682 &                 0 &               0.0 &                       1 &                        614 &       5.479 &   0.831149 &             0.765 \\
2016 - 2017 &      703 &     1925 &                 0 &               0.0 &                       1 &                        703 &       5.477 &   0.839165 &             0.774 \\
2017 - 2018 &      815 &     2300 &                 0 &               0.0 &                       1 &                        815 &       5.644 &   0.857237 &             0.775 \\
\bottomrule
\end{tabular}
}
    \caption{Collaborativeness metrics for cumulative graphs' main clusters.}\label{table:clusters_cumulative}
\end{adjustbox}
\end{table}

To better assess the change over time for each metric they have been plotted in
Figure~\ref{fig:cumulative_networks}. The number of nodes, connected components
and the size of largest component have been normalised such that the trend between
the three networks can be compared.

\begin{itemize}
    \item In Figure~\ref{fig:normalised_number_nodes} the normalised number of nodes,
    which is calculated by dividing by the total number of nodes in each respective network,
    is shown. A steep increase in the size of all three networks is spotted soon after
    2000. This could indicate that more data have been available in the sources
    used in this work following the year 2000. It is however, definitely not a effect
    of a single field, as it is true for all three sub fields considered here.
    The sudden increase following the year 2000, is also reported by the
    number of connected components and the size of the main cluster, Figures
    \ref{fig:normalised_number_connected_components}, \ref{fig:normalised_size_of_cc}.
    A connected components represents at least one publication which means that indeed
    more articles are being gathered from 2000 onwards.
    \item Auction games have been through out time less collaborative compared
    to the iterated prisoner's dilemma. The average degree (Figure~\ref{fig:average_degree})
    and the clustering coefficient (Figure~\ref{fig:clustering_coefficient}) of 
    the cumulative sub graphs have been lower than that of \(G_1\).
    The only exception is during years 2001-2008. For these
    year auction games appear to have had a more collaborative environment.
    \item In the price of anarchy cumulative graphs a sharp increase since the 
    beginning of the field can be observed for all metrics. There are not
    many data points due to the recent development of the field, however
    these steep trends could be an indication that game theoretic and potentially
    all scientific research has over time been more collaborative. This could
    be due to logistic and techical solutions.
    \item The high values of modularity through out time is not true only for the
    network reported in~\cite{Liu2015} but also for all three networks of this
    field. This could indicate a limitation to the co authorship network. The
    measure is likely to be skewed, as each paper is more likely a connected
    component on it's own.
\end{itemize}

\begin{figure}[!hbtp]
    \centering
    \begin{subfigure}{.45\textwidth}\centering
        \includegraphics[width=1.1\textwidth]{./assets/images/percentage_networks_nodes.pdf}
        \caption{\% Nodes.}\label{fig:normalised_number_nodes}
    \end{subfigure}
    \begin{subfigure}{.45\textwidth}\centering
        \includegraphics[width=1.1\textwidth]{./assets/images/degrees_over_time.pdf}
        \caption{Average Degree.}\label{fig:average_degree}
     \end{subfigure}

     \begin{subfigure}{.45\textwidth}\centering
        \includegraphics[width=1.1\textwidth]{./assets/images/connected_components_over_time.pdf}
        \caption{\% Connected components.}\label{fig:normalised_number_connected_components}
     \end{subfigure}
    \begin{subfigure}{.45\textwidth}\centering
        \includegraphics[width=1.1\textwidth]{./assets/images/size_of_largest_cc_over_time.pdf}
        \caption{\% Size of largest connected component.}\label{fig:normalised_size_of_cc}
    \end{subfigure}

    \begin{subfigure}{.45\textwidth}\centering
        \includegraphics[width=1.1\textwidth]{./assets/images/clustering_coeff_over_time.pdf}
        \caption{Clustering coefficient.}\label{fig:clustering_coefficient}
     \end{subfigure}
     \begin{subfigure}{.45\textwidth}\centering
        \includegraphics[width=1.1\textwidth]{./assets/images/modularity_over_time.pdf}
        \caption{Modularity.}\label{fig:modularity}
     \end{subfigure}
    \caption{Collaborative metrics over time for cumulative networks for \(G_1\),
    \(G_2\) and \(G_3\).}\label{fig:cumulative_networks}
\end{figure}

The cumulative collaborative metrics have also been calculated for each main
cluster, given in Table~\ref{table:clusters_cumulative}. Summarily, the results
do not appear to change over the main cluster.

\newpage

The next results discussed here are on centrality measures. As a reminder,
two centrality measures are reported here, these are the closeness centrality
and the betweenness centrality. Closeness centrality is a measure of how
easy is for an author to contact others, and consequently affect them; influence them.
Thus closeness centrality here is a measure of influence. Betweenness centrality
is a measure of how many paths pass through a specific node, thus the amount
of information this person has access to. Betweenness centrality is used here
as a measure of how much an author gain from the field. All centrality measure
can have values ranging from \([0, 1]\).

For \(G_1\)
the most central author based on closeness and betweenness are given by Tables
\ref{table:central_authors_cc} and \ref{table:central_authors} respectively.
The betweenness centrality of the
most central authors in \(G_1\) are rather low with the highest ranked author
being Matjaz Perc with a between centrality of 0.008, Table~\ref{table:central_authors}.
A publication of Perc's work has been briefly discussed in
Section, and the centrality measure suggest that the network is very influenced by him.
He is connected to a total of 58 nodes and he has published to all five of the different
sources considered in the study. Though he also gains from his position
in the network, the gain is minor. An author who is not in the top influencers but
does indeed gain from his position in the network is Martin Nowak. An author
that his work has been discussed in Section~\ref{section:timeline}.

\begin{figure}[!hbtp]
    \centering
    \begin{minipage}{.45\textwidth}
        \centering
        \begin{tabular}{llr}
\toprule
{} &      Name &  Closeness \\
\midrule
1  &   L. Wang &   0.096421 \\
2  &   M. Perc &   0.095338 \\
3  &  Y. Zhang &   0.094736 \\
4  &   Z. Wang &   0.094260 \\
5  &   Y. Chen &   0.090542 \\
6  &   J. Wang &   0.089248 \\
7  &   X. Wang &   0.088720 \\
8  &    Y. Liu &   0.088546 \\
9  &  J. Zhang &   0.088181 \\
10 &  L. Zhang &   0.087923 \\
\bottomrule
\end{tabular}

        \caption{Ten most influenced authors in \(G_1\).}\label{table:central_authors_cc}
    \end{minipage}%
    \begin{minipage}{.45\textwidth}
        \centering
        \begin{tabular}{llr}
\toprule
{} &             Name &  Betweeness \\
\midrule
1  &      Matjaz Perc &    0.008331 \\
2  &        Zhen Wang &    0.006356 \\
3  &     Yamir Moreno &    0.004806 \\
4  &        Long Wang &    0.003538 \\
5  &     Martin Nowak &    0.003230 \\
6  &  Valerio Capraro &    0.002739 \\
7  &    Arne Traulsen &    0.002479 \\
8  &    Angel Sanchez &    0.002319 \\
9  &       Jianye Hao &    0.002188 \\
10 &   Franz Weissing &    0.002186 \\
\bottomrule
\end{tabular}

        \caption{Authors that gain the most influence in \(G_1\).}\label{table:central_authors}
    \end{minipage}
\end{figure}


From Tables~\ref{table:central_authors_cc} and \ref{table:central_authors} it
can be seen that authors in \(G_1\) are more likely to affect their field instead
of gaining from it. This can be better explored by considering the distributions
of the centralities and by comparing them to other fields.
The distributions for both centralities are
plotted in Figures~\ref{fig:betweenness_dist} and~\ref{fig:closeness_dist}, and
in Figures~\ref{fig:betweenness_dist_cluster} and~\ref{fig:closeness_dist_cluster}
for their respective main clusters.

Regarding gaining from your network. An author is more likely to gain more
from the influence of the field if they were authors in auction games or the
price of anarchy. Though if it were an author in the main cluster it would make
to statistical difference in which field they were to published. Overall,
all the betweenness values are rather small and the distributions skewed to the
left. This could imply that in all three networks, authors do not gain much from
the influence of their fields.

\begin{figure}[!hbtp]
    \centering
    \begin{subfigure}{\textwidth}\centering
        \includegraphics[width=\textwidth]{./assets/images/betweeness_distributions.pdf}
        \caption{Betweenness centrality distributions \(G_1, G_2, G_3\). The descriptive
        statistics for each of the distribution are for \(G_1\): mean\(=0.000019\),
        median\(=0.0\), std\(=0.000207\). For \(G_2\): mean\(=0.000086\), median\(=0.0\),
        std\(=0.000693\) and for \(G_3\): mean\(=0.000151\), median\(=0.0\), std\(=0.000931\).
        None of the three distributions is normally distributed and there is
        significant difference between the means (these have been tested using
        appropriate statistical difference). According to a Mann Whitney both
        \(G_2\) and \(G_3\) medians are significant larger than that of \(G_1\) however there
        is not statistical difference between those two medians.}\label{fig:betweenness_dist}
    \end{subfigure}
    \begin{subfigure}{\textwidth}\centering
        \includegraphics[width=\textwidth]{./assets/images/betweeness_distributions_clusters.pdf}
        \caption{Betweenness centrality distributions for \(G_1, G_2, G_3\) respective
        main clusters.The descriptive
        statistics for each of the distribution are for \(G_1\): mean\(=0.0054\),
        median\(=0.0\), std\(=0.022\). For \(G_2\): mean\(=0.0048\), median\(=0.0\),
        std\(=0.019\) and for \(G_3\): mean\(=0.02\), median\(=0.0\), std\(=0.055\).
        None of the three distributions is normally distributed. There is no
        statistical difference between the medians of \(G_1\) and \(G_3\). There
        is however, statistical difference between the median of \(G_2\).
        These have been tested using a Kruskal Wallis test.}\label{fig:betweenness_dist_cluster}
    \end{subfigure}%
\end{figure}

In relation to influencing your field. An author is most likely to influence their
field if the write for the price of anarchy and authors that publish on auctions
game are more likely to influence compared to authors in the iterated prisoner's
dilemma. Though if an author was to be placed in the main cluster of the respective
field they would chose to be in either \(G_1\) or \(G_3\).
In conclusion, authors regarding both influence metrics that have been defined here
are more likely to gain more if they were to published on either topics of auction
games or the price of anarchy. Though the value of gaining is actual very small,
you are more likely to influence you field more in another compared to that
of the prisoner's dilemma.

\begin{figure}[!hbtp]
    \centering
    \begin{subfigure}{\textwidth}\centering
    \includegraphics[width=\textwidth]{./assets/images/closeness_distributions.pdf}
    \caption{Closeness centrality distributions \(G_1, G_2, G_3\). The descriptive
        statistics for each of the distribution are for \(G_1\): mean\(=0.0050\),
        median\(=0.00056\), std\(=0.010\). For \(G_2\): mean\(=0.000086\), median\(=0.00058\),
        std\(=0.000693\) and for \(G_3\): mean\(=0.000151\), median\(=0\), std\(=0.000931\).
        None of the three distributions is normally distributed and the median
        of \(G_3\) is statistically larger than that of \(G_2\), which is
        larger than that of \(G_1\).}\label{fig:closeness_dist}
\end{subfigure}
\begin{subfigure}{\textwidth}\centering
    \centering
    \includegraphics[width=\textwidth]{./assets/images/closeness_distributions_clusters.pdf}
    \caption{Closeness centrality distributions for \(G_1, G_2, G_3\) respective
    main clusters. The descriptive statistics for each of the distribution are
    for \(G_1\): mean\(=0.19\),
    median\(=0.19\), std\(=0.035\). For \(G_2\): mean\(=0.14\), median\(=0.14\),
    std\(=0.026\) and for \(G_3\): mean\(=0.19\), median\(=0.19\), std\(=0.035\).
    None of the three distributions is normally distributed. All medians
    are statistically different. The medians of \(G_1\) and \(G_3\) are greater
    than that of \(G_2\).}\label{fig:closeness_dist_cluster}
    \end{subfigure}
\end{figure}

\section{Acknowledges}

A variety of software libraries have been used in this work:

\begin{itemize}
    \item Networkx~\cite{networkx}, library for analysing networks.
    \item Gephi~\cite{ICWSM09154} open source package for visualising networks.
    \item louvain, library for calculating the networks modularity.
\end{itemize}

\newpage
\bibliographystyle{plain}
\bibliography{bibliography.bib}
\end{document}