\documentclass{article}
% Package to manage page layout
\usepackage[margin=1.5cm, includefoot, footskip=30pt]{geometry}

\setlength\parindent{0pt}
\setlength{\parskip}{1em}

%%%%%%%PACKAGES HERE%%%%%%%
\usepackage{amsmath}
\usepackage{amsthm}
\usepackage{amssymb}
\usepackage{hyperref}
\usepackage{standalone}
\usepackage{subcaption}
\usepackage{adjustbox}
\usepackage{tikz}
\usepackage{booktabs}
\usepackage{minted}
\usepackage{multicol}
\usepackage{graphicx}
\usepackage{algorithm,algorithmic}
\usetikzlibrary{er,positioning, calc}

\definecolor{background}{RGB}{5, 66, 81}
\usemintedstyle{tango}

\setcounter{secnumdepth}{4}
\setcounter{tocdepth}{4}

\usepackage{kbordermatrix}
\theoremstyle{definition}
\newtheorem{definition}{Definition}[section]

%%%%%%%%%%%%%%%%%%%%%%%%%%%%%%%PARAMETERS%%%%%%%%%%%%%%%%%%%%%%%%%%%%%%%%%%%%%%%
\newcommand{\totalarticles}{\input{assets/total_articles.txt}}
\newcommand{\manual}{\input{assets/prov_manual.txt}}
\newcommand{\authors}{\input{assets/num_Nodes.txt}}
\newcommand{\edges}{\input{assets/num_Edges.txt}}
\newcommand{\isolated}{\input{assets/num_Isolated_nodes.txt}}
\newcommand{\isolatedpercentage}{\input{assets/perce_Isolated_nodes.txt}}
\newcommand{\connectedcomponents}{\input{assets/num_Connected_components.txt}}
\newcommand{\communities}{\input{assets/num_Communities.txt}}
\newcommand{\largestcc}{\input{assets/Size_of_largest_component.txt}}
\newcommand{\clustering}{\input{assets/Clustering_coeff.txt}}
\newcommand{\avdegree}{\input{assets/Av._degree.txt}}
% \newcommand{\prisonerscc}{\input{assets/prisoners_clustering.txt}}
% \newcommand{\pricecon}{\input{assets/price_connected_components.txt}}
% \newcommand{\pricecc}{\input{assets/price_clustering.txt}}
% \newcommand{\auctioncon}{\input{assets/auction_connected_components.txt}}
% \newcommand{\auctioncc}{\input{assets/auction_clustering.txt}}
% \newcommand{\prisonerisolated}{\input{assets/prisoners_isolated.txt}}
% \newcommand{\auctionisolated}{\input{assets/auction_isolated.txt}}
% \newcommand{\priceisolated}{\input{assets/price_isolated.txt}}
%%%%%%%%%%%%%%%%%%%%%%%%%%%%%%%%%%%%%%%%%%%%%%%%%%%%%%%%%%%%%%%%%%%%%%%%%%%%%%%%
%%%%%%%%%%%%%%%%%%%%%%%%%%%%%%%%%%%%%%%%%%%%%%%%%%%%%%%%%%%%%%%%%%%%%%%%%%%%%%%%
\title{A systematic literature review of the Prisoner's Dilemma; collaboration and influence.}
\author{Nikoleta E. Glynatsi, Vincent A. Knight}
\date{}

\begin{document}

\maketitle

\begin{abstract}

\end{abstract}

\section{Introduction}\label{section:introduction}

\section{Timeline}\label{section:timeline}


\section{Analysing a large corpus of articles}\label{section:analysis}

The focus of Section~\ref{section:timeline} has been the academic publications on
the topic of the iterated prisoner's dilemma. Whilst in Section~\ref{section:timeline}
several publications of specific interest were covered and the literature was manually
partitioned in different sections, in the second part of this paper the publications
are analysed using a large dataset of articles. In Section~\ref{section:background}
some background
research on bibliometrics is covered. The data collection process is covered in
Section~\ref{section:data_collection} and a preliminary analysis of the data is
conducted in Section~\ref{section:preliminary_analysis}. In Section~\ref{section:methodology},
the methodology which will be used to analyse the author relationships is presented.
In summary, graph theoretical methods will be used to ascertain the level of
collaborative nature of the field and identify influence.
This type of analysis has been carried out in~\cite{Liu2015}. The novelty here
is to consider approaches not considered in~\cite{Liu2015} and new origins of
publications. A further comparison of the results are made, relative to
two other sub fields of game theory: auction games~\cite{menezes2005} and
the price of anarchy~\cite{roughgarden2005} and a temporal analysis.
Finally in Section~\ref{section:results}, the results of the analysis are presented.

\subsection{Background}\label{section:background}

As discussed in~\cite{youngblood2018}, bibliometrics or the statistical analysis
of published works (originally described by~\cite{pritchard1969}) 
have been used to support historical assumptions about the development of fields
\cite{raina1998}, identify connections between scientific growth and policy changes 
\cite{das2016}, to develop a quantitative understanding of author order~\cite{sekara2018},
and investigate the collaborative structure of an interdisciplinary field~\cite{Liu2015}.
Most academic research is undertaken in the form of collaborative effort and as~\cite{Kyvik2017}
points out, it is rationale that two or more people have the potential to do better
as a group than individually. Collaboration in groups has a long tradition in experimental 
sciences and it has be proven to be productive according to~\cite{Etzkowitz1992}.
The number of collaborations can be very different between research fields and
understanding how collaborative a field is, is not always an easy task.
Several studies tend to consider academic citations as a measure for these things.
A blog post published in Nature~\cite{nature_blog} argues that depending on citations
can often be misleading because the true number of citations can not be
known. Citations can be missed due to data entry errors, academics are influenced
by many more papers than they actually cite and several of the citations are
superficial.

A more recent approach to measure collaborative behaviour is to use the co
authorship network, as described in~\cite{Liu2015}. Using this approach has many
advantages as several graph theoretic measures can be used as proxies to explain
authors relationship. In~\cite{Liu2015}, they analyse the development of the field
``evolution of cooperation'' using this approach. The topic ``evolution of cooperation''
is a multidisciplinary field which also includes a large number of publications
on the prisoner's dilemma. This paper builds upon the work done by~\cite{Liu2015}
and extends their methodology. Though in
\cite{Liu2015}, they considered a data set from a single source, Web of Science,
the data set described here has been collected from five different sources. Moreover, the collaborative
results of the analysis are compared to those of two different sub fields.
Co authorship networks have also been used in~\cite{youngblood2018} for classifying
topics of an interdisciplinary field. This was done using centrality measures,
which will be covered below, here centrality measures are used in order to understand
the influence an author can have and can receive by being part of an academic group.
Furthermore, in~\cite{alshebli2018} they study the relationship between research
impact and five classes of diversity: ethnicity, discipline, gender, affiliation,
and academic age. These characteristics of the authors described here are not
being captured. This is considered to be a limitation to this work which will
be will explored in future work.

\subsection{Data Collection}\label{section:data_collection}

Academic articles are accessible through scholarly databases.
Several databases and collections today offer access through
an open application protocol interface (API). An API allows users to query
directly a journal's database and bypass the user interface side of the journal.
Interacting with an API has two phases: requesting and receiving.
The request phase includes composing a url with the details of the request. For
example, \url{http://export.arxiv.org/api/query?search_query=abs:prisoner's dilemma&max_results=1}
represents a request message. The first part of the request is the address
of the API we are querying.
In this example the address corresponds to the API of arXiv.
The second part of the request contains the search arguments. In this example 
it is requested that the word `prisoners dilemma' exists within the article's title.
The format of the request message is different from API to API.
The receive phase includes receiving a number of raw metadata of articles that
satisfies the request message. The raw metadata are commonly received in extensive markup
language (xml) or Javascript object notation (json) formats~\cite{nurseitov2009}.
Similarly to the request message, the structure of the received data differs from journal
to journal.

The data collection is crucial to this study. To ensure that this study can be
reproduced all code used to query the different APIs has been packaged as a Python library and is
available online~\cite{nikoleta_2017}. The software could be used for any type of
projects similar to the one described here, documentation for it is available at:
\url{http://arcas.readthedocs.io/en/latest/}.
Project~\cite{nikoleta_2017} allow users to collect articles from a list of APIs by
specifying just a single keyword. Four prominent journals in the field and a
pre print server were used as sources to collect data for this analysis:
PLOS, Nature, IEEE, Springer and arXiv.

A series of search terms were used to identify relevant articles:

\begin{itemize}
    \item ``prisoner's dilemma'',
    \item ``prisoners dilemma'',
    \item ``prisoner dilemma'',
    \item ``prisoners evolution'',
    \item ``prisoner game theory''
\end{itemize}

and articles for which any of these terms existed within the title, the abstract
or the text are included in the analysis. More specifically, 23\% of article
considered here were included because any of the above terms existed within
the abstract, 50\% within the main text and 27\% within the title.
As will be described in Section~\ref{section:preliminary_analysis}, two other
game theoretic sub fields were also considered in this work, auction games and the
price of anarchy. For collecting data on these sub fields the search terms used
were ``auction game theory'' and ``price of anarchy''. The three data sets
are archived and available at. % TODO: archive data.
Note that the latest data collection was perform on November
2018.% TODO Ensure this stay accurate

\subsection{Preliminary Analysis}\label{section:preliminary_analysis}

A summary of each of the three data sets used is presented in this section.
The three data sets are:

\begin{itemize}
    \item The main data set which contains articles on the prisoner's dilemma.
    \item A data set which contains article on auction games.
    \item A data set which contains articles on the price of anarchy.
\end{itemize}
% TODO archive all data sets

The main data set is archived at~[ref]. It
consists of \totalarticles articles with unique titles. In case of duplicates
the preprint version of an article (collected from arXiv) was dropped.
%TODO add reference to archived dataset
Of these \totalarticles article, \manual have not been collected from the
aforementioned APIs. These articles were of specific interest and manually added
to the dataset throughout the writing of Section~\ref{section:timeline}. A
similar approach was used in~\cite{Liu2015} where a number of articles of interest
where manually added to the data set. A more
detailed summary of the articles' provenance is given by Table~\ref{table:provenance}.
Only 3\% of the data set consists of articles that were manually added and 33\% of the
articles were collected from arXiv.

\begin{table}[!hbtp]
    \begin{center}
    \begin{tabular}{lrr}
\toprule
{} &  \# of Articles &  Percentage \\
provenance &                &             \\
\midrule
Manual     &             89 &        2.88 \\
IEEE       &            295 &        9.55 \\
PLOS       &            482 &       15.60 \\
Springer   &            572 &       18.52 \\
Nature     &            673 &       21.79 \\
arXiv      &           1056 &       34.19 \\
\bottomrule
\end{tabular}

    \end{center}
    \caption{Articles' provenance for the main data set.} % Use citation when archived
    \label{table:provenance}
\end{table}

The average number of publications was calculated for the entire dataset and for
each provenance. The average number of publications is denoted as,
\(\mu_P = \frac{N_A}{N_Y},\) where \(N_A\) is the total number of articles and
\(N_Y\) is the years of publication. The years of publication is calculated as
the range between the collection date and the first published article, for each provenance,
within the data.
These averages are summarised in Table~\ref{table:publication_rates}.
Overall an average of 49 articles are published per year on the topic. The most
significant contribution to this appears to be from arXiv with 16 articles per year,
followed by Nature with 10 and Springer with 9.

\begin{table}[!hbtp]
    \begin{center}
    \begin{tabular}{lr}
\toprule
{} &  Average Yearly publication \\
\midrule
IEEE     &                         5.0 \\
PLOS     &                         8.0 \\
Springer &                         9.0 \\
Nature   &                        11.0 \\
arXiv    &                        16.0 \\
Overall  &                        49.0 \\
\bottomrule
\end{tabular}

    \end{center}
    \caption{Average publication for main data set.} % Use citation when archived
    \label{table:publication_rates}
\end{table}

Though the average publication offers insights about the publications of the
fields, it remains a constant number. The data handled here is a time
series between 1950, when the game was introduced, and 2018 (Figure~\ref{fig:timeseries}). 
Two observations can be made from Figure~\ref{fig:timeseries}.

\begin{enumerate}
    \item A steady increase to the number of publications since the 1980s
    and the introduction of computer tournaments.
    \item A decrease in 2017-2018. This is due to our data set being incomplete.
    Articles that have been written in 2017-2018 have either not being published
    or have are not retrievable by the APIs yet.
\end{enumerate}

These observations can be confirmed by studying the time series.
Using~\cite{scipy}, an exponential distribution is fitted to
the data from 1980-2016. The exponential fitting proves that since 1980 there has
been an increase in the number of publications till 2016 (Figure~\ref{fig:fitting}).
The fitted model can also be used to project the behaviour of the field for the next 
5 years. The forecasted periods are plotted in Figure~\ref{fig:forecasting} and
their exact values are given by Table~\ref{table:forecast}. The time series
has indicated a slight decrease however we can see that the model forecasts that the number
of publications will keep increasing, thus indicating that the field of the iterated
prisoner's dilemma still attracts academic attention.

\begin{figure}[!hbtp]
\begin{minipage}{.45\textwidth}
    \centering
    \includegraphics[width=.9\textwidth]{./assets/images/timeline.pdf}
    \caption{Line plot; \# of articles published on the PD 1950-2019.}\label{fig:timeseries}
\end{minipage}
\begin{minipage}{.45\textwidth}
    \centering
    \includegraphics[width=.9\textwidth]{./assets/images/fitting.pdf}
    \caption{Scatter plot; \# of articles published on the PD 1980-2019.}\label{fig:fitting}
\end{minipage}
\end{figure}

\begin{figure}[!hbtp]
    \centering
    \includegraphics[width=.5\textwidth]{./assets/images/forecasting.pdf}
    \caption{Forecast for 2017-2022.}\label{fig:forecasting}
\end{figure}

\begin{table}[!hbtp]
    \begin{center}
    \begin{tabular}{lr}
\toprule
{} &  Forecast \\
\midrule
2017 &     371.0 \\
2018 &     423.0 \\
2019 &     483.0 \\
2020 &     550.0 \\
2021 &     627.0 \\
\bottomrule
\end{tabular}

    \end{center}
    \caption{Forecasting the number of publications over the next 10 years.}
    \label{table:forecast}
\end{table}

To allow for a comparative analysis two sub fields of game theory have been chosen
for this work; auction games and the price of anarchy.

\begin{itemize}
    \item Auction theory is a branch of game theory which researches the properties
    of auction markets.
    Game theory is used for years to study auctions and the behaviour of
    bidders~\cite{Shubik1971}. The earliest entry in our data set~[ref] goes
    back to 1974 (Figure~\ref{fig:timeseries_ag}). Note that no articles
    have been added manually for auction games.
    \item Price of Anarchy is a concept in game theory which
    measures how the efficiency of a system degrades due to selfish behaviour of
    it's agents. There is a variety of such measures however the price of anarchy
    has attracted a lot of attention since it's informal introduction in 1999
    by~\cite{Koutsoupias1999}. Note that~\cite{Koutsoupias1999} has been manually
    added to the date set and it's the first entry (Figure~\ref{fig:timeseries_pa}).
\end{itemize}

A summary of both data sets collected on both topics, in comparison to that of
[ref], is given by Table~\ref{table:summary_other_topics}.

\begin{table}[!hbtp]
    \centering
    \resizebox{\textwidth}{!}{
    \begin{tabular}{lrrllrrrrr}
\toprule
{} &  Num. Articles &  Num. Authors & Manual (\%) & PLOS (\%) &  Nature (\%) &  Springer (\%) &  IEEE (\%) &  arXiv (\%) &  Av. Yearly Publication \\
\midrule
Prisoner's Dilemma &           3077 &          5772 &        2.5 &    15.66 &       21.87 &         18.59 &      9.59 &      34.32 &                    49.0 \\
Auction Games      &           3444 &          5362 &          - &        - &        5.89 &         37.63 &      7.46 &      51.36 &                    93.0 \\
Price of Anarchy   &            748 &          1316 &       0.13 &     1.74 &       24.73 &         37.97 &     30.61 &       8.82 &                    39.0 \\
\bottomrule
\end{tabular}
}
    \caption{Measures of all three data sets.}\label{table:summary_other_topics}
\end{table}

The iterated prisoner's dilemma and auction theory are very well studied topics
that have been publicising for decades. A large number of articles have
been collected for both topics, \totalarticles and 34444 respectively. Though, auction
games have a larger number of articles, the iterated prisoner's dilemma
has almost 300 more authors.

Auction games have an overall average yearly publication
of 93 articles per year compared to the prisoner's dilemma with 49 per year. 50\% of articles
for [ref] have been collected from the pre print server arXiv and no articles have
been published in PLOS.

Compared to these two topics the price of anarchy is a fairly recent one. Only a
total of 747 articles have been collected, however it has a large number
of 1229 authors. On average each paper had had at least two authors.
It has an overall average publication of 39 articles and the biggest contribution
has been made to Springer.

\begin{figure}[!hbtp]
    \begin{minipage}{.45\textwidth}
        \centering
        \includegraphics[width=\textwidth]{./assets/images/Auction_Games.pdf}
        \caption{Line plot; \# articles published on auction games 1974-2018.}\label{fig:timeseries_ag}
    \end{minipage}%
    \begin{minipage}{.45\textwidth}
        \centering
        \includegraphics[width=\textwidth]{./assets/images/Price_of_Anarchy.pdf}
        \caption{Line plot; \# articles published on the price of anarchy.}\label{fig:timeseries_pa}
    \end{minipage}
    \end{figure}

\subsection{Methodology}\label{section:methodology}

The relationship between the authors within a field will be modelled as a graph \(G\) with
a set \(V_G\) of nodes and \(E_G\) of edges. The set \(V_G\) represents the authors
and an edge connects two authors if and only if those authors have written together.
The co authorship network is constructed using the main data set described in
Section~\ref{section:preliminary_analysis} and the open source package
\cite{networkx}. The prisoner's dilemma network is denoted as \(G_1\) where the
number of unique authors \(|V(G_1)|\) is \authors and \(|E(G_1)|=\) \edges.
Note that the names of all authors names were formatted as their last name and
first initial (i.e. Martin A. Nowak to Martin Nowak). This was done to avoid errors
such as Martin A. Nowak and Martin Nowak, being treated as a different person.

Collaborativeness, will be analysed using measures such as, isolated nodes,
connected components, clustering coefficient, communities, modularity and average degree.
These measures show the number of connections authors can have
and how strongly connected these people are. The number of isolated nodes is the
number of nodes that are not connected to another node, thus the
number of authors that have published alone. The average degree denotes the average
number of neighbours for each nodes, i.e. the average number of collaborations
between the authors.

A connected component is a maximal set of nodes such that each pair of nodes is
connected by a path. The number of connected components as well as the size of the
largest connected component in the network are reported.
The size of largest connected component represents the scale of the central cluster
of the entire network, as it will discussed in the analysis section.
Clustering coefficient and modularity and are also calculated. The clustering
coefficient, defined as 3 times the number of triangle on the graph divided
by the number of connected triples of nodes, is a local measure of the degree to
which nodes in a graph tend to cluster together
in a clique. It is precisely the probability that the collaborators
of an author also write together.

In comparison, modularity is a global measure designed to measure the strength of
division of a network into communities. The number of communities will be reported
using the Clauset-Newman-Moore method~\cite{clauset2004}. Also the modularity index
is calculated using the Louvain method described in~\cite{Blondel2008}. The value
of the modularity index can vary between \([-1, 1]\), a high value of modularity
corresponds to a structure where have dense connections between the nodes within
communities but sparse connections between nodes in different communities.
That means that authors in the network are mainly connected co-authors that they
all have written together, and not to several different collaborators.

Networks are commonly dominated by one person who controls information flow and
people that receive a great amount of information due to their position.
Two further points are aimed to be explored in this work, (1) which people control the flow;
as in which people influence the field the most and (2) which are the authors that
gain the most from the influence of the field. To measure these concepts graph
theoretic metrics, more specifically centrality measures are going to be used.
Centrality measures are often used to understand different
aspects fo social networks~\cite{Landherr2010}. In order to achieve that two
centrality measures that have been chosen are closeness and betweenness
centrality.

\begin{enumerate}
    \item In networks some nodes have a short distance to other nodes and consequently
    are able to spread information on the network very effectively.
    A representative of this idea is \textbf{closeness centrality}, where a person is seen
    as centrally involved in the network if it requires only few intermediaries
    to contact others and thus is structurally relatively independent. Here,
    this is defined as influence. Authors with a high value of closeness centrality,
    are the authors that spread scientific knowledge easier on the network
    and they have high influence.
    \item Another centrality measure is the \textbf{betweenness centrality},
    where the determination of an author's centrality is based on the quotient of
    the number of all shortest paths between nodes in the network that include
    the regarded node and the number of all shortest paths in the network.
    In betweenness centrality the position of the node matters. Nodes
    with a higher value of betweenness centrality are located in positions that
    a lot of information pass through them, this is defined as the gain from
    the influence, thus these authors gain the most from their networks.
\end{enumerate}

In the next section all the metrics discussed here are calculated for the data
sets in order to provide insights into the field.

\subsection{Analysis of co authorship network}\label{section:results}

As mentioned previously, \(G_1\) denotes the co authorship network of the iterated
prisoner's dilemma. It's graphical representation is given by Figure~\ref{fig:g_one_network}.
It is evident that the network is disjoint, which is only natural as many authors
write academic articles on their own. More specifically, a total of \isolated authors,
have had single author publications, which corresponds to the \isolatedpercentage (\%)
of authors in \(G_1\).

There are a total of \connectedcomponents connected components and the largest one
has a size of \largestcc nodes. The largest connected component is shown in Figure~\ref{fig:g_one_cluster}
and is going to be refereed to as the main cluster of the network.
There are total of \communities communities in \(G_1\). The network has a clustering
coefficient of \clustering,
thus authors are 70\% likely to write with a collaborator's co author and the degree
distribution, Figure~\ref{fig:degree_distrs}, shows that the average degree is \(\approx4\).
Thus authors are on average connected to 4 other authors, however there are authors
with far more connections, the largest one being 58.

In~\cite{Liu2015} the collaborative metrics for the ``evolution of cooperation''
co authorship network were reported. Though their network is of smaller size
(number of nodes 3670 \(<\) 5394), the collaborative metrics are fairly similar
between the two graphs (clustering coeff. \(0.632\approx0.708\) and modularity
\(0.950\approx0.977\)), indicating that for the same multidisciplinary field the same
remarks can be made from a different co authors network. But how does these compare to other fields
and more specifically to other fields of game theory? The representation of the
two graphs,

\begin{itemize}
    \item \(G_2\) for auction games and
    \item \(G_3\) for the price of anarchy,
\end{itemize}

are given by Figures~\ref{fig:g_two} and~\ref{fig:g_three} and their respective
clusters in Figures~\ref{fig:g_two_cluster} and \ref{fig:g_three_cluster}.
As stated before \(G_3\) is the smallest network of all three, this is also clearly
seen from it's graphical representation. The \(G_2\) network  appears to be
very similar to \(G_1\), however it's main cluster is larger in size.

\begin{figure}[!hbtp]
    \begin{subfigure}{.45\textwidth}\centering
        \includegraphics[width=.54\textwidth]{./assets/pd_network.pdf}
        \caption{\(G_1\) network.}\label{fig:g_one_network}
    \end{subfigure}
    \begin{subfigure}{.45\textwidth}\centering
        \includegraphics[width=.54\textwidth]{./assets/pd_network_cluster.pdf}
        \caption{\(G_1\) larger connected component.}\label{fig:g_one_cluster}
     \end{subfigure}
     
     \begin{subfigure}{.45\textwidth}\centering
        \includegraphics[width=.54\textwidth]{./assets/auction.pdf}
        \caption{\(G_2\) network.}\label{fig:g_two}
     \end{subfigure}
    \begin{subfigure}{.45\textwidth}\centering
        \includegraphics[width=.54\textwidth]{./assets/auction_network_cluster.pdf}
        \caption{\(G_2\) larger connected component.}\label{fig:g_two_cluster}
    \end{subfigure}

    \begin{subfigure}{.45\textwidth}\centering
        \includegraphics[width=.54\textwidth]{./assets/anarchy_network.pdf}
        \caption{\(G_3\) network.}\label{fig:g_three}
     \end{subfigure}
     \begin{subfigure}{.45\textwidth}\centering
        \includegraphics[width=.54\textwidth]{./assets/anarchy_network_cluster.pdf}
        \caption{\(G_3\) larger connected component.}\label{fig:g_three_cluster}
     \end{subfigure}
     \caption{Graphical representations of \(G_1, G_2, G_3\) and their respective
     main clusters.}
\end{figure}

A summary of the collaborative metrics for all three co authorship networks is given by
Table~\ref{table:summary_other_networks} and shown in Figure~\ref{fig:degree_distrs}
are the degree distributions of all three networks. In \(G_1\) and \(G_2\) there
are cases of high degree (\(> 20\)) but this could be an affect of the size of
the data, networks and subsequently the size of the main clusters.

\begin{table}[!hbtp]
    \centering
    \resizebox{\textwidth}{!}{
    \begin{tabular}{lrrrrrrrrrr}
\toprule
{} &  \# Nodes &  \# Edges &  \# Isolated nodes &  \% Isolated nodes &  \# Connected components &  Size of largest component &  Av. degree &  \# Communities &  Modularity &  Clustering coeff \\
\midrule
Prisoner's Dilemma &     5394 &    10397 &               176 &               3.3 &                    1356 &                        815 &       3.855 &           1369 &       0.977 &             0.708 \\
Auction Games      &     5165 &     7861 &               256 &               5.0 &                    1272 &                       1348 &       3.044 &           1294 &       0.958 &             0.622 \\
Price of Anarchy   &     1155 &     1953 &                 4 &               0.3 &                     245 &                        222 &       3.382 &            253 &       0.965 &             0.712 \\
\bottomrule
\end{tabular}
}
    \caption{Network metrics for \(G_1, G_2, G_3\).}\label{table:summary_other_networks}
\end{table}

\begin{figure}[!hbtp]
    \centering
    \includegraphics[width=\textwidth]{./assets/images/networks_ditributions.pdf}
    \caption{Degree distribution for networks \(G_1, G_2\) and \(G_3\). The descriptive
    statistics for each of the distribution are for \(G_1\): mean \(=3.85\),
    median \(=3\), std \(=4.25\). For \(G_2\): mean \(=3.04\), median \(=2\),
    std \(=3.01\) and for \(G_3\): mean \(=3.38\), median \(=3\), std \(=2.90\).
    It is clear that these distributions are not normally distributed, which has
    also been verified using a statistical test. Moreover, the statistical difference
    of the medians has been tested using a Kruskal Wallis test. The medians
    of \(G_1\) and \(G_3\) are not significantly different, however they are
    are significantly large than that of \(G_2\).}\label{fig:degree_distrs}
\end{figure}

Regarding the three sub fields of game theory and using Table~\ref{table:summary_other_networks}
and Figure~\ref{fig:degree_distrs} the following remarks can be made:

\begin{itemize}
    \item All three networks have similar values of modularity index, and they are
    all very high (Table~\ref{table:summary_other_networks}), indicating that the
    networks are partitioned in many communities.
    Note that the number of
    communities is very much similar to the number of connected components.
    This is all to expected.
    Due to the nature of our network,
    the number of connected components and the number of communities are very close.
    Most connected components represent a single publication written by all the 
    authors in the component (corresponding to a fully connected graph), and due to that density they are
    also a community on their own.
    \item Comparing to another well studied topic, auction games, the field of the
    iterated prisoner's dilemma appear to be more collaborative. Due to the
    value of the average degree, authors in \(G_1\) are known to have on average almost
    one more collaboration than \(G_2\). A slightly lower cluster coefficient ($ .622 < .702$)
    of auction games indicate that is less likely for authors in \(G_2\) to collaborate
    with a collaborators co author.
    \item Regarding the price of anarchy, the measures indicate that the field
    is not as mature as the other two sub fields. There are no isolated authors,
    which is more of an indication of the time the field has been active. As a
    more recent field there had been better communication tools that enable more
    collaborations between researches. The average degree as well as the clustering
    coefficient (clustering coeff.$=0.713$) of \(G_3\) is comparable to those
    of the iterated prisoner's dilemma.
\end{itemize}

These results can be extend to the main clusters of each network, Table~\ref{table:summary_clusters}.
The metrics's values are fairly similar and the size of \(G_2\)'s main cluster
does not appear to gave any significant effect; all the same conclusions are made.
Compared to auction games the iterated prisoner's
dilemma is a more collaborative field, and it is fairly similar to the price of
anarchy. However, our analysis suggests that the price of anarchy is still a
maturing field.

\begin{table}[!hbtp]
    \centering
    \resizebox{\textwidth}{!}{
    \begin{tabular}{lrrrrrrrrr}
\toprule
{} &  \# Nodes &  \# Edges &  \# Isolated nodes &  \% Isolated nodes &  \# Connected components &  Size of largest component &  Av. degree &  Modularity &  Clustering coeff \\
\midrule
Prisoner's Dilemma &      815 &     2300 &                 0 &               0.0 &                       1 &                        815 &       5.644 &       0.857 &             0.775 \\
Auction Games      &     1348 &     3158 &                 0 &               0.0 &                       1 &                       1348 &       4.685 &       0.858 &             0.699 \\
Price of Anarchy   &      221 &      520 &                 0 &               0.0 &                       1 &                        221 &       4.706 &       0.818 &             0.714 \\
\bottomrule
\end{tabular}
}
    \caption{Network metrics for \(G_1, G_2, G_3\).}\label{table:summary_clusters}
\end{table}

The change of the networks over time is also studied by constructing the network
cumulatively with a year interval. A total of 64 sub graphs
over 64 periods, starting in 1950, were created and all the collaborative metrics
for each sub graph have been calculated. Note that years 1952 and 1953 have no
publications in our data set. The metrics of each network for each period are given by
Table~\ref{table:coll_cumulative} and for each main cluster by Table~\ref{table:clusters_cumulative}.
Similar to the results of~\cite{Liu2015}, it can been observed that the network \(G_1\)
grows over time and that the network always had a high value of modularity.

\begin{table}[!hbtp]
    \centering
    \begin{adjustbox}{totalheight=.8\textheight-2\baselineskip, width=\textwidth}
    \begin{tabular}{lrrrrrrl}
\toprule
{} &  \# Connected Components &  \# Isolated &  \% Isolated &  Av. Degree &  Clustering &  Largest cc & Modularity \\
\midrule
Period 0  &                       3 &           3 &        1.00 &        0.00 &        0.00 &           1 &          - \\
Period 1  &                       2 &           2 &        1.00 &        0.00 &        0.00 &           1 &          - \\
Period 2  &                       3 &           3 &        1.00 &        0.00 &        0.00 &           1 &          - \\
Period 3  &                       4 &           4 &        1.00 &        0.00 &        0.00 &           1 &          - \\
Period 4  &                       6 &           6 &        1.00 &        0.00 &        0.00 &           1 &          - \\
Period 5  &                       7 &           7 &        1.00 &        0.00 &        0.00 &           1 &          - \\
Period 6  &                       7 &           7 &        1.00 &        0.00 &        0.00 &           1 &          - \\
Period 7  &                       8 &           8 &        1.00 &        0.00 &        0.00 &           1 &          - \\
Period 8  &                       9 &           9 &        1.00 &        0.00 &        0.00 &           1 &          - \\
Period 9  &                      10 &          10 &        1.00 &        0.00 &        0.00 &           1 &          - \\
Period 10 &                      12 &          10 &        0.71 &        0.29 &        0.00 &           2 &        0.5 \\
Period 11 &                      15 &          12 &        0.67 &        0.33 &        0.00 &           2 &   0.666667 \\
Period 12 &                      19 &          13 &        0.46 &        0.93 &        0.16 &           5 &   0.591716 \\
Period 13 &                      21 &          15 &        0.48 &        0.97 &        0.16 &           6 &   0.533333 \\
Period 14 &                      23 &          16 &        0.47 &        0.94 &        0.15 &           6 &   0.585938 \\
Period 15 &                      26 &          16 &        0.38 &        1.10 &        0.26 &           6 &   0.763705 \\
Period 16 &                      27 &          16 &        0.36 &        1.16 &        0.31 &           6 &   0.801775 \\
Period 17 &                      29 &          17 &        0.35 &        1.12 &        0.29 &           6 &   0.814815 \\
Period 18 &                      29 &          17 &        0.35 &        1.12 &        0.29 &           6 &   0.814815 \\
Period 19 &                      30 &          18 &        0.37 &        1.10 &        0.29 &           6 &   0.814815 \\
Period 20 &                      30 &          18 &        0.37 &        1.10 &        0.29 &           6 &   0.814815 \\
Period 21 &                      33 &          19 &        0.35 &        1.07 &        0.26 &           6 &   0.837099 \\
Period 22 &                      34 &          19 &        0.34 &        1.07 &        0.25 &           6 &   0.846667 \\
Period 23 &                      36 &          21 &        0.34 &        1.11 &        0.25 &           6 &   0.854671 \\
Period 24 &                      37 &          22 &        0.34 &        1.16 &        0.28 &           6 &   0.866326 \\
Period 25 &                      37 &          22 &        0.33 &        1.27 &        0.29 &           6 &    0.85941 \\
Period 26 &                      40 &          24 &        0.34 &        1.27 &        0.32 &           6 &   0.873086 \\
Period 27 &                      43 &          26 &        0.35 &        1.23 &        0.30 &           6 &   0.878072 \\
Period 28 &                      46 &          28 &        0.35 &        1.19 &        0.28 &           6 &   0.882752 \\
Period 29 &                      46 &          28 &        0.35 &        1.19 &        0.28 &           6 &   0.882752 \\
Period 30 &                      54 &          35 &        0.40 &        1.09 &        0.26 &           6 &   0.887153 \\
Period 31 &                      58 &          38 &        0.41 &        1.05 &        0.24 &           6 &   0.891295 \\
Period 32 &                      61 &          39 &        0.39 &        1.07 &        0.26 &           6 &   0.903524 \\
Period 33 &                      70 &          43 &        0.37 &        1.10 &        0.26 &           6 &   0.921643 \\
Period 34 &                      75 &          45 &        0.36 &        1.10 &        0.26 &           6 &   0.930363 \\
Period 35 &                      84 &          50 &        0.36 &        1.07 &        0.26 &           6 &   0.939007 \\
Period 36 &                      87 &          52 &        0.36 &        1.07 &        0.25 &           6 &   0.942486 \\
Period 37 &                      95 &          54 &        0.34 &        1.12 &        0.26 &           6 &   0.951852 \\
Period 38 &                     110 &          62 &        0.32 &        1.24 &        0.33 &           6 &    0.95996 \\
Period 39 &                     118 &          64 &        0.31 &        1.23 &        0.31 &           6 &   0.964111 \\
Period 40 &                     127 &          68 &        0.30 &        1.30 &        0.32 &           6 &   0.966357 \\
Period 41 &                     137 &          71 &        0.29 &        1.31 &        0.34 &           6 &   0.970812 \\
Period 42 &                     138 &          55 &        0.20 &        1.42 &        0.36 &           6 &   0.976778 \\
Period 43 &                     139 &          49 &        0.17 &        1.50 &        0.38 &           6 &   0.979126 \\
Period 44 &                     156 &          54 &        0.16 &        1.53 &        0.40 &           6 &   0.981765 \\
Period 45 &                     157 &          42 &        0.12 &        1.67 &        0.42 &           9 &   0.980204 \\
Period 46 &                     178 &          48 &        0.12 &        1.70 &        0.43 &           9 &   0.982271 \\
Period 47 &                     208 &          54 &        0.11 &        1.66 &        0.43 &           9 &   0.985315 \\
Period 48 &                     230 &          54 &        0.10 &        1.73 &        0.44 &          10 &   0.986425 \\
Period 49 &                     256 &          54 &        0.09 &        1.82 &        0.47 &          20 &   0.985488 \\
Period 50 &                     309 &          62 &        0.08 &        1.94 &        0.50 &          22 &   0.988459 \\
Period 51 &                     361 &          72 &        0.07 &        2.18 &        0.53 &          26 &   0.986283 \\
Period 52 &                     404 &          82 &        0.07 &        2.40 &        0.55 &          40 &   0.984604 \\
Period 53 &                     451 &          94 &        0.07 &        2.50 &        0.55 &          70 &   0.979801 \\
Period 54 &                     503 &         106 &        0.07 &        2.81 &        0.56 &         195 &   0.965391 \\
Period 55 &                     570 &         116 &        0.06 &        3.00 &        0.59 &         259 &   0.964951 \\
Period 56 &                     637 &         120 &        0.05 &        3.16 &        0.62 &         377 &   0.960718 \\
Period 57 &                     700 &         131 &        0.05 &        3.30 &        0.63 &         498 &   0.956249 \\
Period 58 &                     769 &         139 &        0.05 &        3.35 &        0.64 &         651 &   0.948894 \\
Period 59 &                     857 &         148 &        0.04 &        3.52 &        0.65 &         845 &   0.943237 \\
Period 60 &                     935 &         155 &        0.04 &        3.98 &        0.67 &        1116 &   0.939336 \\
Period 61 &                     978 &         157 &        0.04 &        4.04 &        0.68 &        1253 &   0.938227 \\
Period 62 &                    1029 &         157 &        0.03 &        4.19 &        0.68 &        1456 &   0.929681 \\
\bottomrule
\end{tabular}
}
    \caption{Collaborativeness metrics for cumulative graphs.}\label{table:coll_cumulative}
\end{adjustbox}
\end{table}

\begin{table}[!hbtp]
    \centering
    \begin{adjustbox}{totalheight=\textheight-2\baselineskip, width=\textwidth}
    \begin{tabular}{lrrrrrrrlr}
\toprule
{} &  \# Nodes &  \# Edges &  \# Isolated nodes &  \% Isolated nodes &  \# Connected components &  Size of largest component &  Av. degree & Modularity &  Clustering coeff \\
\midrule
1954 - 1950 &        1 &        0 &                 1 &             100.0 &                       1 &                          1 &       0.000 &          - &             0.000 \\
1954 - 1955 &        1 &        0 &                 1 &             100.0 &                       1 &                          1 &       0.000 &          - &             0.000 \\
1955 - 1956 &        1 &        0 &                 1 &             100.0 &                       1 &                          1 &       0.000 &          - &             0.000 \\
1956 - 1957 &        1 &        0 &                 1 &             100.0 &                       1 &                          1 &       0.000 &          - &             0.000 \\
1957 - 1958 &        1 &        0 &                 1 &             100.0 &                       1 &                          1 &       0.000 &          - &             0.000 \\
1958 - 1959 &        1 &        0 &                 1 &             100.0 &                       1 &                          1 &       0.000 &          - &             0.000 \\
1959 - 1961 &        1 &        0 &                 1 &             100.0 &                       1 &                          1 &       0.000 &          - &             0.000 \\
1961 - 1962 &        1 &        0 &                 1 &             100.0 &                       1 &                          1 &       0.000 &          - &             0.000 \\
1962 - 1964 &        1 &        0 &                 1 &             100.0 &                       1 &                          1 &       0.000 &          - &             0.000 \\
1964 - 1965 &        1 &        0 &                 1 &             100.0 &                       1 &                          1 &       0.000 &          - &             0.000 \\
1965 - 1966 &        2 &        1 &                 0 &               0.0 &                       1 &                          2 &       1.000 &          0 &             0.000 \\
1966 - 1967 &        2 &        1 &                 0 &               0.0 &                       1 &                          2 &       1.000 &          0 &             0.000 \\
1967 - 1968 &        5 &        8 &                 0 &               0.0 &                       1 &                          5 &       3.200 &          0 &             0.867 \\
1968 - 1969 &        6 &       10 &                 0 &               0.0 &                       1 &                          6 &       3.333 &       0.02 &             0.833 \\
1969 - 1970 &        6 &       10 &                 0 &               0.0 &                       1 &                          6 &       3.333 &       0.02 &             0.833 \\
1970 - 1971 &        6 &       10 &                 0 &               0.0 &                       1 &                          6 &       3.333 &       0.02 &             0.833 \\
1971 - 1972 &        6 &       10 &                 0 &               0.0 &                       1 &                          6 &       3.333 &       0.02 &             0.833 \\
1972 - 1973 &        6 &       10 &                 0 &               0.0 &                       1 &                          6 &       3.333 &       0.02 &             0.833 \\
1973 - 1974 &        6 &       10 &                 0 &               0.0 &                       1 &                          6 &       3.333 &       0.02 &             0.833 \\
1974 - 1975 &        6 &       10 &                 0 &               0.0 &                       1 &                          6 &       3.333 &       0.02 &             0.833 \\
1975 - 1976 &        6 &       10 &                 0 &               0.0 &                       1 &                          6 &       3.333 &       0.02 &             0.833 \\
1976 - 1977 &        6 &       10 &                 0 &               0.0 &                       1 &                          6 &       3.333 &       0.02 &             0.833 \\
1977 - 1978 &        6 &       10 &                 0 &               0.0 &                       1 &                          6 &       3.333 &       0.02 &             0.833 \\
1978 - 1979 &        6 &       10 &                 0 &               0.0 &                       1 &                          6 &       3.333 &       0.02 &             0.833 \\
1979 - 1980 &        6 &       10 &                 0 &               0.0 &                       1 &                          6 &       3.333 &       0.02 &             0.833 \\
1980 - 1981 &        6 &        9 &                 0 &               0.0 &                       1 &                          6 &       3.000 &  0.0493827 &             0.678 \\
1981 - 1982 &        6 &        9 &                 0 &               0.0 &                       1 &                          6 &       3.000 &  0.0493827 &             0.678 \\
1982 - 1983 &        6 &       10 &                 0 &               0.0 &                       1 &                          6 &       3.333 &       0.02 &             0.833 \\
1983 - 1984 &        6 &        9 &                 0 &               0.0 &                       1 &                          6 &       3.000 &  0.0493827 &             0.678 \\
1984 - 1985 &        6 &        9 &                 0 &               0.0 &                       1 &                          6 &       3.000 &  0.0493827 &             0.678 \\
1985 - 1986 &        6 &       10 &                 0 &               0.0 &                       1 &                          6 &       3.333 &       0.02 &             0.833 \\
1986 - 1987 &        6 &        9 &                 0 &               0.0 &                       1 &                          6 &       3.000 &  0.0493827 &             0.678 \\
1987 - 1988 &        6 &       10 &                 0 &               0.0 &                       1 &                          6 &       3.333 &       0.02 &             0.833 \\
1988 - 1989 &        6 &       10 &                 0 &               0.0 &                       1 &                          6 &       3.333 &       0.02 &             0.833 \\
1989 - 1990 &        6 &        9 &                 0 &               0.0 &                       1 &                          6 &       3.000 &  0.0493827 &             0.678 \\
1990 - 1991 &        6 &        9 &                 0 &               0.0 &                       1 &                          6 &       3.000 &  0.0493827 &             0.678 \\
1991 - 1992 &        6 &        9 &                 0 &               0.0 &                       1 &                          6 &       3.000 &  0.0493827 &             0.678 \\
1992 - 1993 &        6 &        9 &                 0 &               0.0 &                       1 &                          6 &       3.000 &  0.0493827 &             0.678 \\
1993 - 1994 &        6 &        9 &                 0 &               0.0 &                       1 &                          6 &       3.000 &  0.0493827 &             0.678 \\
1994 - 1995 &        6 &        9 &                 0 &               0.0 &                       1 &                          6 &       3.000 &  0.0493827 &             0.678 \\
1995 - 1996 &        6 &        9 &                 0 &               0.0 &                       1 &                          6 &       3.000 &  0.0493827 &             0.678 \\
1996 - 1997 &        6 &        9 &                 0 &               0.0 &                       1 &                          6 &       3.000 &  0.0493827 &             0.678 \\
1997 - 1998 &        6 &        9 &                 0 &               0.0 &                       1 &                          6 &       3.000 &  0.0493827 &             0.678 \\
1998 - 1999 &        6 &        9 &                 0 &               0.0 &                       1 &                          6 &       3.000 &  0.0493827 &             0.678 \\
1999 - 2000 &        6 &       10 &                 0 &               0.0 &                       1 &                          6 &       3.333 &       0.02 &             0.833 \\
2000 - 2001 &        7 &       21 &                 0 &               0.0 &                       1 &                          7 &       6.000 &          0 &             1.000 \\
2001 - 2002 &        7 &       21 &                 0 &               0.0 &                       1 &                          7 &       6.000 &          0 &             1.000 \\
2002 - 2003 &        7 &       21 &                 0 &               0.0 &                       1 &                          7 &       6.000 &          0 &             1.000 \\
2003 - 2004 &       10 &       13 &                 0 &               0.0 &                       1 &                         10 &       2.600 &    0.37574 &             0.553 \\
2004 - 2005 &       19 &       28 &                 0 &               0.0 &                       1 &                         19 &       2.947 &   0.544005 &             0.730 \\
2005 - 2006 &       21 &       32 &                 0 &               0.0 &                       1 &                         21 &       3.048 &   0.530273 &             0.713 \\
2006 - 2007 &       24 &       36 &                 0 &               0.0 &                       1 &                         24 &       3.000 &   0.533179 &             0.678 \\
2007 - 2008 &       32 &       59 &                 0 &               0.0 &                       1 &                         32 &       3.688 &   0.627837 &             0.732 \\
2008 - 2009 &       56 &      102 &                 0 &               0.0 &                       1 &                         56 &       3.643 &   0.716792 &             0.699 \\
2009 - 2010 &       99 &      238 &                 0 &               0.0 &                       1 &                         99 &       4.808 &   0.781539 &             0.734 \\
2010 - 2011 &      121 &      288 &                 0 &               0.0 &                       1 &                        121 &       4.760 &   0.776054 &             0.713 \\
2011 - 2012 &      210 &      610 &                 0 &               0.0 &                       1 &                        210 &       5.810 &     0.7809 &             0.747 \\
2012 - 2013 &      330 &      908 &                 0 &               0.0 &                       1 &                        330 &       5.503 &    0.81959 &             0.753 \\
2013 - 2014 &      406 &     1125 &                 0 &               0.0 &                       1 &                        406 &       5.542 &    0.81494 &             0.749 \\
2014 - 2015 &      514 &     1390 &                 0 &               0.0 &                       1 &                        514 &       5.409 &   0.827358 &             0.757 \\
2015 - 2016 &      614 &     1682 &                 0 &               0.0 &                       1 &                        614 &       5.479 &   0.831149 &             0.765 \\
2016 - 2017 &      703 &     1925 &                 0 &               0.0 &                       1 &                        703 &       5.477 &   0.839165 &             0.774 \\
2017 - 2018 &      815 &     2300 &                 0 &               0.0 &                       1 &                        815 &       5.644 &   0.857237 &             0.775 \\
\bottomrule
\end{tabular}
}
    \caption{Collaborativeness metrics for cumulative graphs' main clusters.}\label{table:clusters_cumulative}
\end{adjustbox}
\end{table}

To better assess the change over time for each metric they have been plotted in
Figure~\ref{fig:cumulative_networks}. The number of nodes, connected components
and the size of largest component have been normalised such that we can compare
the trend between the three networks.

\begin{itemize}
    \item In Figure~\ref{fig:normalised_number_nodes} the normalised number of nodes,
    which is calculated by dividing by the total number of nodes in each respective network,
    is shown. A steep increase in the size of all three networks is spotted soon after
    2000. This could indicate that more data have been available in the sources
    used in this work following the year 2000. It is however, definitely not a effect
    of a single field, as it is true for all three sub fields considered here.
    The sudden increase following the year 2000, is also reported by the
    number of connected components and the size of the main cluster, Figures
    \ref{fig:normalised_number_connected_components}, \ref{fig:normalised_size_of_cc}.
    A connected components represents at least one publication which means that indeed
    more articles are being gathered from 2000 onwards.
    \item Auction games have been through out time less collaborative compared
    to the iterated prisoner's dilemma. The average degree (Figure~\ref{fig:average_degree})
    and the clustering coefficient (Figure~\ref{fig:clustering_coefficient}) of 
    the cumulative sub graphs have been lower than that of \(G_1\) throughout
    time. The only exception is during years 2001-2008. For these
    year auction games appear to have had a more collaborative environment.
    \item In the price of anarchy cumulative graphs a sharp increase since the 
    beginning of the field can be observed for all metrics. There are not
    many data points due to the recent development of the field, however
    these steep trends could be an indication that game theoretic and potentially
    all scientific research has over time been more collaborative. This could
    be due to logistic and techical solutions.
    \item The high values of modularity through out time is not true only for the
    network reported in~\cite{Liu2015} but also for all three networks of this
    field. This could indicate a limitation to the co authorship network. As the
    measure is likely to be skewed, as each paper is more likely a connected
    component on it's own.
\end{itemize}

\begin{figure}[!hbtp]
    \centering
    \begin{subfigure}{.45\textwidth}\centering
        \includegraphics[width=1.1\textwidth]{./assets/images/percentage_networks_nodes.pdf}
        \caption{\% Nodes.}\label{fig:normalised_number_nodes}
    \end{subfigure}
    \begin{subfigure}{.45\textwidth}\centering
        \includegraphics[width=1.1\textwidth]{./assets/images/degrees_over_time.pdf}
        \caption{Average Degree.}\label{fig:average_degree}
     \end{subfigure}

     \begin{subfigure}{.45\textwidth}\centering
        \includegraphics[width=1.1\textwidth]{./assets/images/connected_components_over_time.pdf}
        \caption{\% Connected components.}\label{fig:normalised_number_connected_components}
     \end{subfigure}
    \begin{subfigure}{.45\textwidth}\centering
        \includegraphics[width=1.1\textwidth]{./assets/images/size_of_largest_cc_over_time.pdf}
        \caption{\% Size of largest connected component.}\label{fig:normalised_size_of_cc}
    \end{subfigure}

    \begin{subfigure}{.45\textwidth}\centering
        \includegraphics[width=1.1\textwidth]{./assets/images/clustering_coeff_over_time.pdf}
        \caption{Clustering coefficient.}\label{fig:clustering_coefficient}
     \end{subfigure}
     \begin{subfigure}{.45\textwidth}\centering
        \includegraphics[width=1.1\textwidth]{./assets/images/modularity_over_time.pdf}
        \caption{Modularity.}\label{fig:modularity}
     \end{subfigure}
    \caption{Collaborative metrics over time for cumulative networks for \(G_1\),
    \(G_2\) and \(G_3\).}\label{fig:cumulative_networks}
\end{figure}

The metrics that change over time for the main clusters have also been plotted,
to further gain understanding in the change of the main cluster for all three networks
over the course of time.
Summary the results do not appear to change over the main cluster. The networks
have been always modular which could be a limitation to the co authorship
network. Over time auction gae have only a small amount of period where the
collaborativeness was more, and the historical data with the steep increases support
our hypothesis that the anarchy is not mature yet.

% \begin{figure}[!hbtp]
%     \centering
%     \begin{subfigure}{.45\textwidth}\centering
%         \includegraphics[width=\textwidth]{./assets/images/degrees_over_time_cluster.pdf}
%         \caption{Average degree.}\label{fig:average_degree_clusters}
%     \end{subfigure}
%     \begin{subfigure}{.45\textwidth}\centering
%         \includegraphics[width=\textwidth]{./assets/images/clustering_coeff_over_time_clusters.pdf}
%         \caption{Clustering coefficient.}\label{fig:g_three}
%      \end{subfigure}
%      \begin{subfigure}{.45\textwidth}\centering
%         \includegraphics[width=\textwidth]{./assets/images/modularity_over_time_cluster.pdf}
%         %\caption{\(G_3\) larger connected component.}\label{fig:g_three_cluster}
%      \end{subfigure}
%      \caption{Collaborative metrics over time for cumulative networks for \(G_1\)'s,
%      \(G_2\)'s and \(G_3\)'s main clusters.}\label{fig:cumulative_networks}
% \end{figure}

\newpage

The next results discussed here are on centrality measures. As a reminder,
two centrality measures are reported here, these are the closeness centrality
and the betweenness centrality. Closeness centrality is a measure of how
easy is for an author to contact others, and consequently affect them; influence them.
Thus closeness centrality here is a measure of influence. Betweenness centrality
is a measure of how many paths pass through a specific nodes, thus the amount
of information this person has access to. Betweenness centrality is used here
as a measure of how much an author gain from the field. All centrality measure
can have values ranging from \([0, 1]\).

For \(G_1\)
the most central author based on closeness and betweenness are given by Tables
\ref{table:central_authors_cc} and \ref{table:central_authors} respectively.
Centrality measures range between \([0, 1]\). The betweenness centrality of the
most central authors in \(G_1\) are rather low with the highest ranked author
being Matjaz Perc with a between centrality of 0.008, Table~\ref{table:central_authors}.
Matjaz Perc is also the first ranked author based on closeness centrality, with
a centrality of 0.04. Perc's work has been briefly discussed in Section, and the
centrality measure suggest that the network is very influenced by him. He is
connected to a total of 58 nodes and he has published to all five of the different
sources we are considering in the study. Though he also gains from his position
in the network, the gain is minor. An author who is not in the top influencers but
does indeed gain from his position in the network is Martin Nowak. An author
that his work has been discussed in Section~\ref{section:timeline}.

\begin{figure}[!hbtp]
    \centering
    \begin{minipage}{.45\textwidth}
        \centering
        \begin{tabular}{llr}
\toprule
{} &      Name &  Closeness \\
\midrule
1  &   L. Wang &   0.096421 \\
2  &   M. Perc &   0.095338 \\
3  &  Y. Zhang &   0.094736 \\
4  &   Z. Wang &   0.094260 \\
5  &   Y. Chen &   0.090542 \\
6  &   J. Wang &   0.089248 \\
7  &   X. Wang &   0.088720 \\
8  &    Y. Liu &   0.088546 \\
9  &  J. Zhang &   0.088181 \\
10 &  L. Zhang &   0.087923 \\
\bottomrule
\end{tabular}

        \caption{Ten most influenced authors in \(G_1\).}\label{table:central_authors_cc}
    \end{minipage}%
    \begin{minipage}{.45\textwidth}
        \centering
        \begin{tabular}{llr}
\toprule
{} &             Name &  Betweeness \\
\midrule
1  &      Matjaz Perc &    0.008331 \\
2  &        Zhen Wang &    0.006356 \\
3  &     Yamir Moreno &    0.004806 \\
4  &        Long Wang &    0.003538 \\
5  &     Martin Nowak &    0.003230 \\
6  &  Valerio Capraro &    0.002739 \\
7  &    Arne Traulsen &    0.002479 \\
8  &    Angel Sanchez &    0.002319 \\
9  &       Jianye Hao &    0.002188 \\
10 &   Franz Weissing &    0.002186 \\
\bottomrule
\end{tabular}

        \caption{Authors that gain the most influence in \(G_1\).}\label{table:central_authors}
    \end{minipage}
\end{figure}


From Tables~\ref{table:central_authors_cc} and \ref{table:central_authors} it
can be seen that authors in \(G_1\) are more likely to affect their field instead
of gaining from it. This can be better explored by considering the distributions
of the centralities and by comparing them to other fields.

Overall, the values of closeness centrality appear to be higher than those of
betweenness. These can be better explored by considering the centralities'
distributions for all three networks. The distributions for both centralities are
plotted in Figures~\ref{fig:betweenness_dist} and~\ref{fig:closeness_dist}, and
in Figures~\ref{fig:betweenness_dist_cluster} and~\ref{fig:closeness_dist_cluster}
for their respective main clusters.

Regarding gaining from your network. For all the distributions the values
are vely low and skewed to the left. That implies that
in all three networks, authors do not gain much from the influence of their
fields.

\begin{figure}[!hbtp]
    \centering
    \begin{subfigure}{\textwidth}\centering
        \includegraphics[width=\textwidth]{./assets/images/betweeness_distributions.pdf}
        \caption{Betweenness centrality distributions \(G_1, G_2, G_3\). The descriptive
        statistics for each of the distribution are for \(G_1\): mean\(=0.000019\),
        median\(=0.0\), std\(=0.000207\). For \(G_2\): mean\(=0.000086\), median\(=0.0\),
        std\(=0.000693\) and for \(G_3\): mean\(=0.000151\), median\(=0.0\), std\(=0.000931\).
        None of the three distributions is normally distributed. There is no
        statistical difference between the medians of \(G_1\) and \(G_3\). There
        is however, statistical difference between the median of \(G_2\).
        These have been tested using a Kruskal Wallis test.}\label{fig:betweenness_dist}
    \end{subfigure}
    \begin{subfigure}{\textwidth}\centering
        \includegraphics[width=\textwidth]{./assets/images/betweeness_distributions_clusters.pdf}
        \caption{Betweenness centrality distributions for \(G_1, G_2, G_3\) respective
        main clusters.The descriptive
        statistics for each of the distribution are for \(G_1\): mean\(=0.0054\),
        median\(=0.0\), std\(=0.022\). For \(G_2\): mean\(=0.0048\), median\(=0.0\),
        std\(=0.019\) and for \(G_3\): mean\(=0.02\), median\(=0.0\), std\(=0.055\).
        None of the three distributions is normally distributed. There is no
        statistical difference between the medians of \(G_1\) and \(G_3\). There
        is however, statistical difference between the median of \(G_2\).
        These have been tested using a Kruskal Wallis test.}\label{fig:betweenness_dist_cluster}
    \end{subfigure}%
\end{figure}
 

On the other hand, closeness distributions have more variation. The following
observations are made from the distributions:

\begin{itemize}
    \item  Neither are normally distributed and there is a significant difference
    between the medians of all three distributions, with \(G_3\) having a larger median.
    \item There are clusters from all three networks for which a number of authors
    have a closeness centrality greater than 0.02. The authors in these clusters
    were explored but not pattern was found behind their publications. The
    provenance and the year of publication were checked.
    \item The authors in these clusters, are the authors which are in the main
    clusters of their relative networks. Thus, the people that influence the
    field the most are the most central authors in the main cluster of the
    co authorship network of a field.
    \item Both \(G_2\) and \(G_3\) have more people influencing the field compared
    to \(G_1\).
\end{itemize}

\begin{figure}[!hbtp]
    \centering
    \begin{subfigure}{\textwidth}\centering
    \includegraphics[width=\textwidth]{./assets/images/closeness_distributions.pdf}
    \caption{Closeness centrality distributions \(G_1, G_2, G_3\). The descriptive
        statistics for each of the distribution are for \(G_1\): mean\(=0.0050\),
        median\(=0.00056\), std\(=0.010\). For \(G_2\): mean\(=0.000086\), median\(=0.00058\),
        std\(=0.000693\) and for \(G_3\): mean\(=0.000151\), median\(=0\), std\(=0.000931\).
        None of the three distributions is normally distributed. All medians
        are statistically different.}\label{fig:closeness_dist}
\end{subfigure}
\begin{subfigure}{\textwidth}\centering
    \centering
    \includegraphics[width=\textwidth]{./assets/images/closeness_distributions_clusters.pdf}
    \caption{Closeness centrality distributions for \(G_1, G_2, G_3\) respective
    main clusters. The descriptive statistics for each of the distribution are
    for \(G_1\): mean\(=0.19\),
    median\(=0.19\), std\(=0.035\). For \(G_2\): mean\(=0.14\), median\(=0.14\),
    std\(=0.026\) and for \(G_3\): mean\(=0.19\), median\(=0.19\), std\(=0.035\).
    None of the three distributions is normally distributed. All medians
    are statistically different.}\label{fig:closeness_dist_cluster}
    \end{subfigure}
\end{figure}

\subsection{Conclusion}
%
\section{Acknowledges}

A variety of software libraries have been used in this work:

\begin{itemize}
    \item Networkx~\cite{networkx}, library for analysing networks.
    \item Gephi~\cite{ICWSM09154} open source package for visualising networks.
    \item louvain, library for calculating the networks modularity.
\end{itemize}

\newpage
\bibliographystyle{plain}
\bibliography{bibliography.bib}
\end{document}
