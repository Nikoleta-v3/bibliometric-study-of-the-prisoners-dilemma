\documentclass{article}
% Package to manage page layout
\usepackage[margin=1.5cm, includefoot, footskip=30pt]{geometry}

\setlength\parindent{0pt}
\setlength{\parskip}{1em}

%%%%%%%PACKAGES HERE%%%%%%%
\usepackage{amsmath}
\usepackage{amsthm}
\usepackage{amssymb}
\usepackage{hyperref}
\usepackage{standalone}
\usepackage{subcaption}
\usepackage{tikz}
\usepackage{booktabs}
\usepackage{minted}
\usepackage{multicol,multirow,array}
\usepackage{algorithm,algorithmic}
\usetikzlibrary{er,positioning, calc, patterns}

\definecolor{background}{RGB}{5, 66, 81}
\usemintedstyle{tango}

\setcounter{secnumdepth}{4}
\setcounter{tocdepth}{4}

\usepackage{kbordermatrix}
\theoremstyle{definition}
\newtheorem{definition}{Definition}[section]

%%%%%%%%%%%%%%%%%%%%%%%%%%%%%%%PARAMETERS%%%%%%%%%%%%%%%%%%%%%%%%%%%%%%%%%%%%%%%
\newcommand{\totalarticles}{\input{assets/total_articles.txt}}
\newcommand{\uniquetitles}{\input{assets/unique_titles.txt}}
\newcommand{\numberofduplicates}{\input{assets/number_of_duplicates.txt}}
\newcommand{\manual}{\input{assets/prov_manual.txt}}
\newcommand{\authors}{\input{assets/authors.txt}}
\newcommand{\edges}{\input{assets/prisoners_edges.txt}}
\newcommand{\auctionauthors}{\input{assets/authors_auction.txt}}
\newcommand{\auctionedges}{\input{assets/edges_auction.txt}}
\newcommand{\priceauthors}{\input{assets/authors_price.txt}}
\newcommand{\priceedges}{\input{assets/edges_price.txt}}
\newcommand{\prisonerscon}{\input{assets/prisoners_connected_components.txt}}
\newcommand{\prisonerscc}{\input{assets/prisoners_clustering.txt}}
\newcommand{\pricecon}{\input{assets/price_connected_components.txt}}
\newcommand{\pricecc}{\input{assets/price_clustering.txt}}
\newcommand{\auctioncon}{\input{assets/auction_connected_components.txt}}
\newcommand{\auctioncc}{\input{assets/auction_clustering.txt}}
\newcommand{\prisonerisolated}{\input{assets/prisoners_isolated.txt}}
\newcommand{\auctionisolated}{\input{assets/auction_isolated.txt}}
\newcommand{\priceisolated}{\input{assets/price_isolated.txt}}
%%%%%%%%%%%%%%%%%%%%%%%%%%%%%%%%%%%%%%%%%%%%%%%%%%%%%%%%%%%%%%%%%%%%%%%%%%%%%%%%
%%%%%%%%%%%%%%%%%%%%%%%%%%%%%%%%%%%%%%%%%%%%%%%%%%%%%%%%%%%%%%%%%%%%%%%%%%%%%%%%
\title{A systematic literature review of the Prisoner's Dilemma; collaboration and influence.}
\author{Nikoleta E. Glynatsi, Vincent A. Knight}
\date{2016}

\begin{document}

\maketitle

\begin{abstract}
    The prisoner's dilemma is a well known game used ever since the 1950's as a framework
    for studying the emergence of cooperation; a topic of continuing interest
    for mathematical, social, biological and ecological sciences. The iterated version
    of the game attracted attention in 1980's after
    the publication of the ``The Evolution of Cooperation'' and has been a topic
    of pioneering research ever since. In this work we aim to provide a chronological
    literature review of the field. This is achieved by partitioning the timeline into six different
    time periods. Furthermore, a comprehensive data set of literature is analysed
    using network theoretic approaches in order to explore the collaborative
    behaviour and identify the influencers of the field.
\end{abstract}

\section{Introduction}\label{section:introduction}

\textit{To add a sentence about selfness and selfishness}. There is a simple
way of representing these behaviours/concepts. This is to use a
particular two player non-cooperative game called the prisoner's dilemma, originally
described in~\cite{Flood1958}.

Each player has two choices, to either be selfness and cooperate or to act in a
selfish manner and chose to defect. Each decision is made simultaneously and independently.
The fitness of each player is influenced by its own behaviour, and the behaviour
of the opponent. Both players do better if they choose to cooperate than if both
choose to defect. However, a player has the temptation to deviate as that player will
receive a higher payoff than that of a mutual cooperation.

A player's payoffs are generally represented by (\ref{eq:the_pd_payoffs}). Both
players receive a reward for mutual cooperation, \(R\), and a payoff \(P\) for
mutual defection. A player that defects while the other cooperates receives a payoff of
\(T\), whereas the cooperator receives \(S\). The dilemma exists due
to constrains (\ref{eq:constrain_one}) and (\ref{eq:constrain_two}).

\begin{equation} \label{eq:the_pd_payoffs}
    \begin{pmatrix}
    R & S \\ T & P
    \end{pmatrix}
\end{equation}

\begin{equation}\label{eq:constrain_one}
    T > R > P > S
\end{equation}

\begin{equation}\label{eq:constrain_two}
    2R > T + S
\end{equation}

Constrains (\ref{eq:constrain_one}), (\ref{eq:constrain_two}) and rational behaviour
guarantee that it never benefits a player to cooperate. It can been shown mathematically that
defecting is the dominant strategy for the one shot prisoner's dilemma. However,
when the game is studied in a manner where prior outcome matters,
defecting is no longer necessarily the dominant choice.

The repeated form of the game is called the iterated prisoner's dilemma and theoretical
works have shown that cooperation can emerge once players interact for more than
one time. The most important of these works has been R. Axelrod's
``The Evolution of Cooperation''~\cite{Axelrod1984}.
In his book Axelrod reports on a series of computer tournaments he organised of
a finite turns games of the iterated prisoner's dilemma. Participants
had to choose between cooperation and defection again and again while having
memory of their previous encounters. Academics from several fields were invited to
design computer strategies to compete. The pioneering work of Axelrod
showed that greedy strategies did very poorly in the long run whereas altruistic
strategies did better.

``The Evolution of Cooperation'' is considered a milestone in the field but it
is not the only one. On the contrary, the prisoner's dilemma has attracted much
attention ever since the game's origins. This is shown in Figure~\ref{fig:timeline},
which illustrates the number of publications on the prisoner's dilemma per year
from the following sources:

\begin{multicols}{3}
    \begin{itemize}
        \item arXiv;
        \item PLOS;
        \item IEEE;
        \item Nature;
        \item Springer.
    \end{itemize}
\end{multicols}

Each point of Figure~\ref{fig:timeline} marks the starting year of a time period.
Each of these time periods is reviewed and presented in Section~\ref{section:timeline},
as subsections of an extensive literature review.
Furthermore, in Section~\ref{section:analysis} a comprehensive data set of literature
regarding the prisoner's dilemma will be presented and analysed. This allow us to
review the amount of published academic articles as well as measure and explore
the collaborations within the field.

\begin{figure}[!htbp]
    \centering
    \includegraphics[width=\textwidth]{assets/images/timeline.pdf}
    \caption{\label{fig:timeline} A timeline of the prisoner's dilemma research.}
\end{figure}

\section{Timeline}\label{section:timeline}

In this section we review a large amount of literature regarding the prisoner's
dilemma. We start from the year the game was formulated all the way to today.

\subsection{Original research (1961-1972)}\label{section:origin}

The origin of the prisoner's dilemma goes back to the 1950s in early experiments
conducted at the RAND~\cite{Flood1958} to test the applicability of games
described in~\cite{VonNeumann1944}. The game received it's name later the same year.
According to~\cite{Tucker1983}, A. W. Tucker (the PhD supervisor of J. Nash~\cite{Nash1951}),
in an attempt to delivery the game with a story during a talk he described the players
as prisoners and the game has been known as the prisoner's dilemma ever since.

The study of the prisoner's dilemma has attracted people from various fields
across the years. An early figure within the field of game theory
is Prof A. Rapoport, a mathematical psychologist, whose work focused on
how to promote international and national cooperation. Rapoport sought to conceptualize
strategies that could promote international cooperation. It should come to no surprise
that in his teaching and research he used the prisoner's dilemma.
In~\cite{rapoport1965} Rapoport conducted experiments using a group of humans
to simulate rounds of the iterated prisoner's dilemma. He sought out to understand
the conditions under which altruist behaviour emerged, and
he was not the only one in his time.
Conditions explored were the gender~\cite{Evans1966, Lutzker1961, Mack1971} of
individuals, the representation of the game~\cite{Evans1966}, the distance between
players~\cite{Sensenig1972}, the initial effects~\cite{Tedeschi1968} and whether
the experimenter was biased~\cite{Gallo1968}. Several of the results presented
in these works argued about the importance of these factors.

The early experiments were very constrained. The only source of simulation was
through groups of humans and those came with disadvantages. Firstly, humans can
behave very randomly and secondly both the size and the background of the
individuals were different from experiment to experiment.

The robustness and efficiency of the early experiments was in question. In the
section we will introduce the pioneer computer tournaments of R. Axelrod, that
largely replaced experimental groups in the study of the iterated prisoner's dilemma.

\subsection{Axelrod's Tournaments (1981-1984)}\label{subsection:axelrods_tournament}

Before the 1980s a great deal of research was done in the field, as discussed
in Section~\ref{section:origin}. However, as described in his article
\cite{Axelrod2012}, the political scientist R. Axelrod believed
that there was no clear answer to the question of how to avoid conflict, or even
how an individual should play the game. Combining his interest in artificial
intelligence and political sciences Axelrod created a framework for exploring these
questions using computer tournaments. This section is dedicated to a series of
computer tournaments he performed in the early 1980s.

The first computer tournament was performed in 1980~\cite{Axelrod1980a}.
Several scientists were invited to submit their strategies, written in the
programming languages Fortran or Basic. There was a total of 13 submissions
made by the following researchers,

\begin{multicols}{2}
    \begin{enumerate}
        \item T Nicolaus Tideman and Paula Chieruzz;
        \item Rudy Nydegger;
        \item Bernard Grofman;
        \item Martin Shubik;
        \item Stein and Anatol Rapoport;
        \item James W Friedman;
        \item Morton Davis;
        \item Jim Graaskamp;
        \item Leslie Downing;
        \item Scott Feld;
        \item Johann Joss;
        \item Gordon Tullock;
        \item Name not given.
    \end{enumerate}
\end{multicols}

Each competed in a 200 turn match against all 12 opponents, itself and a player
that played randomly. This type of tournament is referred to as a round robin and
corresponds to a complete graph from a topological point of view. The tournament
was repeated \(5\) times to reduce variation in the results. Each participant knew
the exact length of the matches and had access to the full history of each match.
Furthermore, Axelrod performed an preliminary tournament and the results were known
to the participants. The payoff values used for equation~(\ref{eq:the_pd_payoffs}) where
\(R=3, P=1, T=5\) and \(S=0\). These values are commonly used in the literature
and unless specified will be the values used in the rest of the work described here.

The winner of the tournament was determined by the total average score and not by
the number of matches won. The strategy that was announced the winner was
submitted by Rapoport and was called \textbf{Tit For Tat}. Tit for Tat, is a
strategy that always cooperates on the first round and then mimics the opponent's
previous move.
The success of Tit for Tat came as a surprise. It was not only the simplest submitted
strategy but it had also won the tournament even though it could never do better
than the player it was interacting with.

In order to further test the results Axelrod performed a second
tournament~\cite{Axelrod1980b} later in 1980. The results of the first tournament
had been publicised and the second tournament received much more attention, with 62 entries
made by the following people,

\begin{multicols}{3}
    \begin{enumerate}
        \item Gail Grisell;
        \item Harold Rabbie;
        \item James W Friedman;
        \item Abraham Getzler;
        \item Roger Hotz;
        \item George Lefevre;
        \item Nelson Weiderman;
        \item Tom Almy;
        \item Robert Adams;
        \item Herb Weiner;
        \item Otto Borufsen;
        \item R D Anderson;
        \item William Adams;
        \item Michael F McGurrin;
        \item Graham J Eatherley;
        \item Richard Hufford;
        \item George Hufford;
        \item Rob Cave;
        \item Rik Smoody;
        \item John Willaim Colbert;
        \item David A Smith;
        \item Henry Nussbacher;
        \item William H Robertson;
        \item Steve Newman;
        \item Stanley F Quayle;
        \item Rudy Nydegger;
        \item Glen Rowsam;
        \item Leslie Downing;
        \item Jim Graaskamp and Ken Katzen;
        \item Danny C Champion;
        \item Howard R Hollander;
        \item George Duisman;
        \item Brian Yamachi;
        \item Mark F Batell;
        \item Ray Mikkelson;
        \item Craig Feathers;
        \item Fransois Leyvraz;
        \item Johann Joss;
        \item Robert Pebly;
        \item James E Hall;
        \item Edward C White Jr;
        \item George Zimmerman;
        \item Edward Friedland;
        \item X	Edward Friedland;
        \item Paul D Harrington;
        \item David Gladstein;
        \item Scott Feld;
        \item Fred Mauk;
        \item Dennis Ambuehl and Kevin Hickey;
        \item Robyn M Dawes and Mark Batell;
        \item Martyn Jones;
        \item Robert A Leyland;
        \item Paul E Black;
        \item T Nicolaus Tideman and Paula Chieruzz;
        \item Robert B Falk and James M Langsted;
        \item Bernard Grofman;
        \item E E H Schurmann;
        \item Scott Appold;
        \item Gene Snodgrass;
        \item John Maynard Smith;
        \item Jonathan Pinkley;
        \item Anatol Rapoport.
    \end{enumerate}
\end{multicols}

The new participants knew the results of the previous tournament. The rules
were similar with only one exception;
the number of turns was not specified instead a fixed probability (refereed to as
`shadow of the future'~\cite{Axelrod1988}) of the game ending on the next move
was used. The fixed probability was chosen to be 0.0036 so that the expected
median length of a match would be 200 turns. The topology was of a round robin
and each pair of players was matched 5 times. Each of the five matches had a
length of 63, 77, 151, 308 and 401.

Several entries tended to be variants of Tit for Tat, such as
\textbf{Tit for Two Tats} submitted by John Maynard Smith. Tit for Two
Tats defects only when the opponent has defected twice in a row. However
none of the variants managed to outperform the pure version and the winner was once
again Tit for Tat. The conclusions made from the first two tournaments were that the strong
performance of the strategy was due to:

\begin{itemize}
    \item The strategy would start of by cooperating.
    \item It would forgive it's opponent after a defection.
    \item It would always be provoked by a defection no matter the history.
    \item As soon as the opponents identified that they were playing Tit for Tat,
    they would choose to cooperate for the rest of the game.
\end{itemize}

However Axelrod wanted to further test the robustness of the strategy. In the later
sections of~\cite{Axelrod1980b}, he discusses about an ecological tournament 
he performed using the 62 strategy of the second tournament. The ecological
approach is a simulation of theoretical future rounds of the game where strategies
that do better are more likely to be included in future rounds than others.
The simulation of the process, as described in~\cite{Axelrod1980b}, is straightforward.
Let us consider an example. Let the four strategies Tit for Tat, Tit for Two Tat,
\textbf{Cooperator} and \textbf{Defector} compete in an ecological tournament.
Cooperator and Defector are two deterministic strategies that will always cooperate
and defect equivalently. The expected payoff matrix, when these four strategies
interact, is give by,

\[\begin{bmatrix}
    3.0,  & 3.0,  & 3.0, & 0.99 \\
    3.0,  & 3.0,  & 3.0, & 0.99 \\
    3.0,  & 3.0,  & 3.0, & 0.0  \\
    1.02, & 1.039,& 5.0, & 1.0 \\
\end{bmatrix}\]

Starting with proportions of each type in a given generation, their proportions
for the next generation needs to be calculated. This is achieved by calculating
the weighted average of the scores of a given strategy with all other players.

\begin{itemize}
    \item The weights are the numbers of the other strategies which
    exist in the current generation.
    \item The numbers of a given strategy in the next generation is then taken to
    be proportional to the product of its numbers in the current generation and
    its score in the current generation.
\end{itemize}

The process is then repeated for a given number of future tournaments.
Figure~\ref{fig:ecological_tournament} illustrates a simulation of our hypothetical
ecological tournament, as shown strategies that cooperate quickly kill off the Defector.
A similar result was presented by Axelrod. In his ecological tournament cooperative
strategies started to take over the population over time. On the other hand exploitative
strategies started to die off as weaker strategies were becoming extinct. In other
words they were dying because there was fewer and fewer prey for them to exploit.

\begin{figure}[!hbtp]
    \centering
    \includegraphics[width=.6\textwidth]{./assets/images/ecological_tournament.png}
    \caption{Results on an ecological tournament with Tit for Tat, Tit for Two Tats,
    Cooperator and Defector.}
    \label{fig:ecological_tournament}
\end{figure}

In 1981, Axelrod also studied the prisoner's dilemma in an evolutionary context based
on the evolutionary approaches of John Maynard Smith~\cite{Smith1973,
Smith1974, Smith1979}. John Smith is a well know evolutionary biologist as well
as an attendant of Axelord's second tournament. John Maynard Smith alongside George Price
are considered fundamental figures of evolutionary game theory. In~\cite{Smith1973}
they introduced the definition of an evolutionary stable strategy (ESS).

Imagine a population made up of individuals where everyone follows the
same strategy \(B\) and a single individual adopts a mutant strategy \(A\).
Strategy \(A\) is said to invade strategy \(B\) if the payoff of \(A\) against \(B\)
is greater than the expected payoff received by \(B\) against itself.
Since strategy \(B\) is in a population that interacts only with itself,
the concept of invasion is equivalent to a single mutant being able to outperform
the average population. Thus for a strategy to be ESS it must be able to resists
any invasion.

The work described in~\cite{axelrod1981}, studied the evolutionary stability of
Tit for Tat and although the strategy was likely to take over the population, its
stability was conditional on the importance of the future of the game. This is
represented by a discounting factor denoted as \(w\). Axelrod showed that if \(w\)
was sufficiently large, Tit for Tat could resist invasion by any other strategy.
Moreover, he showed how a small cluster of Tit for Tat players could invade a extortionate
environment.
In~\cite{Axelrod1984}, Axelrod decided to work on the biological applications of
his evolutionary results in collaboration with the biologist William Donald Hamilton.
According to Richard Dawkins he was the one to introduce Axelrod to Hamilton's work.
Their collaboration~\cite{Axelrod1984} won the pair the Newcomb-Cleveland prize
of the American Association for the Advancement of Science.

Axelrod's work was evolutionary and offered many insights to the field. Several
other strategies, apart from Tit for Tat, are still being
used in research today. Such as Tit for Two Tats and \textbf{Grudger}.
Grudger was originally submitted by James W. Friedman. Grudger is a strategy that
will cooperate as long as the opponent does not defect. The name Grudger was give
to the strategy in~\cite{Li2014}. Though the strategy goes by many names in the
literature such as, Spite~\cite{Beaufils1997}, Grim Trigger~\cite{Banks1990} and
Grim~\cite{Van2015}.

Axelrod was neither the first to demonstrate that cooperation is possible
in the iterated prisoner's dilemma, neither to provide proof that reciprocity can
be advantageous, however, he was the first to conduct these experiments
in a such a well designed framework. As describe in~\cite{Rapoport2015},
``Axelrod's “new approach” has been extremely successful and immensely
influential in casting light on the conflict between an individual and the collective
rationality reflected in the choices of a population whose members are unknown
and its size unspecified, thereby opening a new avenue of research.''
However, it is important to mention that the only source code available is the code
for the 62 strategies of the second tournament, found on Axelrod's personal
website~\cite{fortan_code}.

His work offered him more than 9000 citations and opened a new avenue of research.
In a collaboration~\cite{Axelrod1988} with a colleague, Douglas Dion, they summarized
a number of works inspired from the ``Evolution of Cooperation''.
Several of these are presented and discussed in the following section where
we cover the response of the academic community to the computer tournaments.

\subsection{Response to the computer tournaments (1984-1993)}
\label{section:responses_to_computer_tournament}

The pioneering work of the computer tournaments and the results of the reciprocal behaviour
in the prisoner's dilemma spread the knowledge of the game worldwide and across
disciplines. Several researchers responded immediately to Axelrod's tournaments
and the study of cooperation became of critical interest once again.
This section focuses on the research that was carried out after the initial
computer tournaments, over the time period between 1984 and 1993.

% One of the scientific disciplines that immediately employed Axelrod's work has
% been the ecological field. More specifically, the works of~\cite{Craig1984,
% Dugatkin1988, Godfray1992, Milinski1987, Wilkinson1984}.
% In~\cite{Dugatkin1988, Milinski1987} the behaviour of fish when confronting a
% potential predator was studied. Conflicts can arise within pairs of fish in these
% circumstances. In both works experiments were held using a system of mirrors
% where sticklebacks and guppies respectively, would be accompanied by a cooperating
% companion or a defecting one. In both cases the hypothesis that the fish would
% behave according to Tit for Tat and that cooperation would evolve was supported.
% The works of~\cite{Godfray1992, Wilkinson1984} looked at food sharing between
% vampire bats and explained behaviour based on known strategies.
% In 1988 publications from other disciplines were using the iterated prisoner's
% dilemma and Axelrod's work for teaching and social studies.
% In~\cite{Levitt1988} a version of the prisoner's dilemma which set the
% problem in an ordinary business context was used as a pedagogic instrument within
% graduate business students. The work of~\cite{Rabow1988} considered non zero sum
% games, specifically the prisoner's dilemma, and illustrated the impact they
% have on societal problems such as war.

One the biggest assumptions in Axelrod's tournaments had been that each player had a perfect
information of the opponent's actions. However, stochastic uncertainty severely undercuts
the effectiveness of reciprocating strategies,~\cite{Molander1985} proved that
in an environment where \textbf{noise} is introduced two strategies playing Tit
for Tat receive the same average payoff as two Random players.
Noise is a probability that a player's move will be
flipped. Hammerstein~\cite{Hammerstein1984}, pointed out another weakness of Tit
for Tat that if by mistake, one of two Tit for Tat players makes a wrong move,
this locks the two opponents into a hopeless sequence of alternating defections
and cooperations. A year later in 1986,~\cite{Donninger1986} ran a computer tournament with a
10 percent chance of noise and Tit for Tat finished sixth out of
21 strategies.

A second type of noise is misperception, where a player's action
is made correctly but it's recorder incorrectly by the opponent. In 1986,~\cite{Sugden2004}
introduced a strategy called
\textbf{Contrite Tit for Tat} that was more successful than Tit for Tat in such environments.
Contrite Tit for Tat has three states:
\textit{contrite, content, provoked}. It begins by cooperating and stays there unless
there is a unilateral defection. If it was the victim of a defection while content
the strategy becomes provoked and defects until the opponent cooperates, and 
causes it to become content. If it was the defector while content, it
becomes contrite and cooperates. When contrite it becomes content only after
there has been a mutual cooperation.

Another limitation of~\cite{axelrod1981} was that the
interactions have been between pairs of players, as argued by~\cite{Joshi1987}.
In several applications, however, interactions involve more
than two players. This can be modelled using the corresponding \(n\)-player prisoner's
dilemma (NPD), in which players make a choice (cooperate or defect) which they play
with all other players. In evolutionary settings~\cite{Joshi1987} found that if
individuals play a "hard" Tit For Tat, meaning that they will cooperate until one
player defects, and the number of individuals playing hard Tir Fot Tat passes a
certain threshold, then hard Tit For Tat can dominate a population of Defectors.
But this threshold rises as the number of individuals in the society increases.

The work of~\cite{May1987} pointed out that it is important `to take more account of intrinsic
stochasticities'. This suggested considering stochastic strategies and
\cite{nowak1989} studied such strategies. An iterated prisoner's dilemma strategy
was represented by using three parameters \((y, p_1, p_2)\), where \(y\) is the
probability to cooperate in the first move, and \(p_1\) and \(p_2\) the conditional probabilities
to cooperate, given that the opponent's last move was a cooperation or a defection.
These are a very specific set of strategies that only remember their
opponent's last move, not their own and they are called reactive strategies.
Using the above notation a strategy can now be defined by a triple. For example,

\begin{itemize}
    \item Defector: (0, 0, 0)
    \item Cooperator: (1, 1, 1)
    \item Tit for Tat: (1, 1, 0)
\end{itemize}

This framework was used in~\cite{nowak1989} to study game dynamical aspects of the
iterated prisoner's dilemma, the results of which will be presented in the next section
which is dedicated to such research. Another outcome of the framework came
the next year. In 1990,~\cite{Nowak1990} gave a formal definition of a memory one strategy.
Memory one strategies consider the entire history of the previous turn to make a
decision (thus reactive strategies are a subset of memory one).

If only a single turn of the game is taken into account and depending on the
simultaneous moves of two players there are only four possible states that
players could possibly be in. These are \(CC, CD, DC\) and \(DD\). A memory one
strategy is denoted by the probabilities of cooperating after each of these states,
\( p = (p_1, p_2, p_3, p_4) \in\mathbb{R}_{[0,1]}^{4} \).
A match between two memory one players \(p\) and \(q\) can be modelled as a
stochastic process, where the players move from state to state. More specifically,
it can be modelled by the use of a Markov chain~\cite{gamerman2006markov},
which is described by a matrix \(M\).

\begin{equation}\label{eq:markov_matrix}
    M =
\begin{bmatrix}
    p_{1} q_{1} & p_{1} (- q_{1} + 1) & q_{1} (- p_{1} + 1) & (- p_{1} + 1) (- q_{1} + 1)
    \\
    p_{2} q_{3} & p_{2} (- q_{3} + 1) & q_{3} (- p_{2} + 1) & (- p_{2} + 1) (- q_{3} + 1)
    \\
    p_{3} q_{2} & p_{3} (- q_{2} + 1) & q_{2} (- p_{3} + 1) & (- p_{3} + 1) (- q_{2} + 1)
    \\
    p_{4} q_{4} & p_{4} (- q_{4} + 1) & q_{4} (- p_{4} + 1) & (- p_{4} + 1) (- q_{4} + 1)
    \\
\end{bmatrix}
\end{equation}

The players are assumed to move from each state until the system reaches a state
steady, let the steady states vector be denoted as \(\bar{v}\).
The utility of a player can be given by multiplying the steady states of
\(M\) by the payoffs of equation~(\ref{eq:the_pd_payoffs}). Thus~\cite{Nowak1990}
offered a mathematical framework to calculate the utility of two players without
actually simulating the game. The payoff of a player \(p\) can be obtained by,

\[s_p = \bar{v} \times \begin{pmatrix} R \\ S \\ T \\ P \end{pmatrix}\]

In 1992 reactive strategies were used
to investigate which strategies would manage to take over
the population and would be ESS in an environment with noise. The results demonstrated that 
though a small fraction of Tit for Tat players have been essential for the emergence
of cooperation, more generous strategies took over the population. More specifically
the re-active strategy known as \textbf{Generous Tit for Tat} which is give by
the triplet \((1, 0, \frac{2}{3})\). Generous Tit for Tat was not the only strategy
to outperform Tit for Tat in a noisy environment, same conclusions were made
by~\cite{Godfray1992, Bendor1991}.

In~\cite{Bendor1991} a similar tournament to that of Axelrod's was performed,
but with noise. Bendor had invited researchers from several
departments across his university. Each match would last a random number of turns,
with a probability of 0.0067 of ending in the next turn. The results of his
tournament demonstrated that Tit for Tat performed rather poorly and the highest
ranked strategies were generous ones. The top ranked strategy was
\textbf{Nice and Forgiving}. Nice and Forgiving, differs in significant ways
from Tit for Tat. Initially, Nice's generosity takes the form of a benign indifference.
It will continue to play cooperation as long as its rival's cooperation level
exceeded 80\%. Secondly, although it will retaliate if its rival's observed cooperation fell
below 80, it is willing to revert to full cooperation before its partner does,
so long as the partner satisfies a certain thresholds of acceptable behaviour.

Mistakes, such as noise and misperception, are unlikely to occur in a computer
tournament, but have to be expected in real life situations. Actual biological
situations are fraught with errors and uncertainties~\cite{Milinski1987}.
As it has been discussed through various works in the section, the answer to the
opponents mistaken moves is to increase or decrease the readiness to cooperate.
The next section will focus on the evolutionary settings and dynamics that the
iterated prisoner's dilemma offers.

\subsection{Evolutionary Dynamics (1987-1992)}\label{section:evolutionary_dynamics}

The complex nature of iterated prisoner's dilemma strategies makes their evolutionary
stability more complex to study. In this section we will cover several works in
the 80's and early 90's that
studied the dynamics of the iterated prisoner's dilemma strategies.

In~\cite{Boyd1987} Boyd and Lorderbaum show that if \(w\), the importance of the
future of the game, is large enough then no deterministic strategy is ESS because
it can always be invaded by any pair of other strategies.
This was also independently proven by~\cite{Pudaite1987}.
Furthermore, Boyd and Lorderbaum, in~\cite{Boyd1989}, showed that cooperation can
be started without a population structure if the correct combination of strategies
is presented. This remark argued with Axelrod's result that only a cluster of
cooperative strategies would succeed. The example they provided was the following:

Suppose that a population consists of Tit for Tat, Suspicious Tit for Tat and
Tit for Two Tat. Suspicious Tit for Tat will end up cooperating after the first
move, Suspicious Tit for Tat and Tit for Tat will each continue cooperating while
the other defects (and vice versa), and Tit for Tat and Tit for Two Tat will cooperate
on every move. In such a situation, Tit for Two Tat will be able to invade both
Tit for Tat and Suspicious Tit for Tat, even without clustering.

Instead of using a given number of strategies, Nowak and Sigmund in~\cite{nowak1989},
studied the dynamics of the evolutionary
iterated prisoner's dilemma with a spectrum of stochastic strategies.
The strategies they considered, as discussed in the previous section, were the reactive
strategies. They managed to prove that there can be multiple fixed points that
there can be an evolutionary stable coexistence among multiple such strategies.

There is limitation to all of these  dynamic treatments. That is their inability
to develop new strategies. A way of overcoming  is to use a genetic approach.
An evolutionary process called the genetic algorithm was used to discover
effective strategies in~\cite{Axelrod1987}. A genetic algorithm is a search heuristic
that is inspired by the theory of natural selection. In a population of candidate
solutions, the fittest individuals are selected for reproduction. They
produce offsprings that will replace the weakest members of the population as long
as they do better than them.

In order to use a genetic algorithm~\cite{Axelrod1987} needed to represent strategies
in a format such that the algorithm could optimise. Axelord considered deterministic
strategies that took into account the last 3 turns of the game. For each turn
there are 4 possible outcomes \((CC, CD, DC, DD)\), thus for 3 turns there are a total
of \(4\times4\times4=64\) possible combinations. Axelord therefore used a list of 64 C's and D's
to represent different strategies. We refer to this representation structure as a
lookup table, which is a set of deterministic responses based on
the opponents \(m\) last moves; for example \cite{Axelrod1987} considered \(m=3\).
In later section we discuss some more recent work that has been done using lookup
tables.

Another dynamic approach that was considered in 1992 is that of using very simplistic
strategies but in more complex topological structures. An extension to the natural selection
where who meets whom is not random anymore. In~\cite{Nowak1992b}, a population
of two deterministic strategies, Defector and Cooperator, were placed onto a two
dimensional square array where the individuals could interact only with the immediate
neighbours. The number of immediate neighbours could be either, fourth, six or eight. As
shown in Figure~\ref{fig:topologies}, where each node represents a player and the
edges denote whether two players will interact.

Thus each cell of the lattice is occupied by a Cooperator or a Defector.
\begin{itemize}
    \item At each generation step each cell owner interacts with its immediate neighbours.
    \item The score of each player is calculated as the sum of all the scores the player
    achieved at each generation.
    \item At the start of the next generation, each lattice
    cell is occupied by the player with the highest score among the previous owner
    and the immediate neighbours.
\end{itemize}

This topology is refereed to as spatial topology. The population dynamics of these
experiments were studied as a function of the temptation (\(T\)) payoff.
More specifically the following payoff matrix was used, which is equivalent
to equation~(\ref{eq:the_pd_payoffs}):

\begin{equation}
    \begin{pmatrix}
    R & S \\ T & P
    \end{pmatrix}
    =
    \begin{pmatrix}
        1 & 0 \\ b & 0
    \end{pmatrix}
\end{equation}

where \((b>1)\).

It was shown that for different values of the temptation payoff \(b\), this purely
deterministic spatial version could generate chaotically change patterns in which
cooperators and defectors could persist together in a mix population. 
Though it was known that in unstructured populations natural selection would favour
defection,~\cite{Nowak1992} provided evidence that in structured populations
the results can be wildly different.
The authors claimed that the essential results remain true of all topologies; 
the results also hold whether self interactions are taken into account.

\begin{figure}[!hbtp]
\centering
    \begin{subfigure}{.25\textwidth}
        \includestandalone[width=\textwidth]{assets/tex/square_lattice}
    \end{subfigure}
    \begin{subfigure}{.25\textwidth}\centering
        \includestandalone[width=\textwidth]{assets/tex/square_lattice_eight}
     \end{subfigure}
     \begin{subfigure}{.25\textwidth}\centering
        \includestandalone[width=\textwidth]{assets/tex/hexagonal_lattice}
     \end{subfigure}
     \caption{Spatial neighbourhoods}\label{fig:topologies}
    \end{figure}


Later in 2003, the authors~\cite{MASUDA2003} decided to consider small world
networks instead of regular graphs. More specifically they used the well
known Watts and Strogatz~\cite{watts1998} graphs. The experiment starts with each node
having \(k/2\) nearest neighbours on each side
. Then a proportion \(\rho\) of
the total edges is rewired by removing \(\rho kn/2\) edges and creating
\(\rho kn/2\) new edges each of whose initial vertex is the initial vertex of a removed
edge and its terminal vertex is randomly chosen so that the generated graph does
not have multiple edges or once removed edges.

The authors argued that by applying these experimental rules we re closer to
capturing real life behaviour where we constantly change who we interact with.
The results of their work had been that:

\begin{itemize}
    \item For small b, it is not so tempting for players to exploit cooperators.
    Thus cooperators converge regardless the value of \(\rho\).
    \item The number of cooperators highly depends on \(\rho\) roughly for for higher
    values of the temptation payoff \(b\).
    \item Lastly once temptation if very strong, even cooperators happen to form
    tight clusters, they cannot survive once they face defectors. Finally, the 
    cooperators eventually extinguish.
\end{itemize}

In 2006, Ohtsuki studied regular \(k\) degree graphs and introduced a "rule"
regarding when is cooperation favoured in spatial tournaments.
Ohtsuki used the following game matrix,

\begin{equation} \label{eq:the_pd_payoffs}
    \begin{pmatrix}
        b - c & c \\ b & 0
    \end{pmatrix}
\end{equation}

where \(b\) represents the altruistic act and \(c\) represents the cost.
Ohtsuki etc all proved that natural selection favours cooperation if
the ration of  \(b/c\) exceeds the average number of neighbours.
Their result, however, holds for weak selection. That means that the fitness of
an individual is only proportional to their payoff.

Later the same year, Ohtsuki and Nowak~\cite{Ohtsuki2006replicator},
managed to approximate the dynamics of a population on graphs using an approximation of the
replicator equations. The replicator equation was introduced as the first closed
form differential equation to describe the dynamics of natural selected populations.
A lot of work has been done on evolutionary games on graphs. Though we have covered
a number of academic publications~\cite{szabo2007} have also conducted a comprehensive
review.

Another topic worth mentioning is that of coevolution on graphs. In~\cite{Perc2011}
studied graphs were a probability of rewiring ones connections was in place,
however, the rewire could be with any given node in the graphs and not just
with imitate neighbours. Perc etc all showed that ``making of new friends'' may
be an important activity for the successful evolution of cooperation,
but also that partners must be selected carefully and one should keep their number
limited.
% A player called \textbf{Handshake} was presented by~\cite{Robson1989} in 1989.
% Handshake is a strategy that starts with cooperation, defection. If the opponent
% plays in a similar way then it will cooperate forever, otherwise it will defect
% forever. Handshake has a property that will be revisited in this literature
% review which is it's recognition property.

\subsection{Modern approaches (1993-2017)}\label{section:modern_approaches}

% In the iterated prisoner's dilemma every player tries to find the best strategy
% in order to maximize their payoff in the long term run.
In this section we will cover several
research projects published between 1993 and 2017. The research reviewed here focuses
on computer tournaments and serves as an introduction to various strategies that
have made an impact in the literature.

Another protagonist in the literature is a strategy called \textbf{Pavlov},
introduced in 1993~\cite{Nowak1993}. Pavlov has the tolerance of Generous Tit for Tat
but also the capability of resisting and invading an all out cooperators population.
The strategy is based on the fundamental behavioural mechanism win-stay,
lose-shift. It starts off with a cooperation and then repeats it's previous move
only if it was awarder with a payoff of \(R\) or \(T\). A variant of Pavlov
was quickly introduced by~\cite{Wu1995}. That variant was Generous Pavlov, a variant
that cooperates 10\% of the times when it would either wise had defected.

An interesting approach of capturing promising strategies for the game
was written in 1996 by~\cite{Miller1996}. Strategies represented by finite automata
were learning to update their choices through a genetic algorithm.
The specific type of finite automata that were used were Moore machines~\cite{moore1956}.
Finite state machine consist of a set of internal states. One of these states
is the initial state of the machine. A machine also consists of transitions
arrows associated with the states. Each arrow is labelled with \(A/R\) where
\(A\) is the opponent's last action and \(R\) is the player's response.
For example let us consider a graphical representation of the famous Tit for Tat
given by a finite machine, Figure~\ref{fig:tit_for_tat_fsm}.

\begin{figure}[!hbtp]
    \centering
    \includestandalone[width=.3\textwidth]{./assets/tex/tit_for_tat_fsm}
    \caption{Finite state machine representation of Tit for Tat.}
    \label{fig:tit_for_tat_fsm}
\end{figure}

Miller used the genetic algorithm to train finite state machines in
environments with noise. His results showed that even a small difference in noise
(from 1\% to 3\%) significantly changed the characteristics of the
evolving strategies. Three machines described in his paper are the following:

\begin{itemize}
    \item \textbf{Punish Twice}: A strategy that punishes defection with 2 defections.
    \item \textbf{Punish Once for Two Tats}: A strategy which will defect only if the
    opponent has defected twice in a row.
    \item \textbf{Punish Twice and Wait}: A variant of Punish Twice which will
    answer defection with 2 defections and will cooperate if an only if the opponent
    cooperated.
\end{itemize}

% and their representation is given by Figure~\ref{}.

In 1997, \textbf{Gradual} another well performed strategy
was proposed by~\cite{Beaufils1997}. Gradual starts off by cooperating,
then after the first defection of the other player, it defects one time and cooperates
twice. After the second defection of the opponent, it defects two times and cooperates
twice. After the \(n^{th}\) defection it reacts with \(n\) consecutive defections 
and then two cooperations. In a tournament of 12 strategies~\cite{Beaufils1997},
Gradual had managed to outperform strategies such as Tit for Tat and Pavlov.

Following the success of Gradual the authors of~\cite{tzafestas-2000a} conducted
the same tournament, with now 13 strategies, designed to outperform Gradual. Their strategy
was the \textbf{Adaptive Tit for Tat} and the algorithm used by it is given by
Algorithm~\ref{alg:adaptive_tft}. Adaptive Tit for Tat ranked first in it's tournament
surpassing Gradual.

\begin{algorithm}
\begin{algorithmic}
    \IF{opponent played \textbf{C} in the last cycle}
     \STATE world = world + \(r(1-\text{world})\), \(r\) is the adaptation rate
    \ELSE
     \STATE work = world + \(r(0 - \text{world})\)
     \ENDIF
    \IF{world \(\geq\) 0.5}
        \STATE play \textbf{C}
    \ELSE
    \STATE play \textbf{D}
    \ENDIF
\end{algorithmic}
\caption{Adaptive Tit for Tat.}
\label{alg:adaptive_tft}
\end{algorithm}

In 2006, Slany and Kienreich, tried to deal with one of the problems of Tit for Tat.
This problem was described above and its that Tit for Tat players would go into
a chain of mutual defections if noise were to be added in the environment.
Furthermore, the also altered their strategy so that it had the ability to
to recognize and exploit the Random strategy in a way that after an opponent
strategy crosses a certain randomness threshold they conclude that the opponent is
a Random strategy and change the behaviour to act as a Defector.
They called their strategy OmegaTFT~\cite{Wolfgang2006}.

Similar to Miller's work, in~\cite{Ashlock2006b} the author presented two new
strategies that have been trained using a finite state machine representation.
These strategies are called,
\textbf{Fortress3} and \textbf{Fortress4}. Figure~\ref{fig:fortress3_and_4}
illustrates their diagrammatic representation.

\begin{figure}[!hbtp]
\centering
    \begin{subfigure}{.4\textwidth}
        \includestandalone[width=\textwidth]{assets/tex/fortress_3}
    \end{subfigure}
    \begin{subfigure}{.4\textwidth}\centering
        \includestandalone[width=\textwidth]{assets/tex/fortress_4}
     \end{subfigure}
     \caption{Representations of Fortress 3 and Fortress 4. Note that the
     strategy's first move, enters state 1, is defection for both strategies.}
     \label{fig:fortress3_and_4}
\end{figure}

During the experiments that introduced Fortress3 and Fortress4 a large diversity
of strategies were found. Differentiating between strategies represented as 
a finite machine is not an easy task.  It is not obvious looking at a finite
state diagram how a machine will behave, and many different machines can represent
the same strategy.
In order to distinguish
the strategies and assuring that they are indeed different~\cite{Ashlock2005}
introduced a method called fingerprinting.
The method of fingerprinting is a technique for generating a functional signature for a
strategy~\cite{Ashlock2008}. This is achieved by computing the score of a strategy
against a spectrum of opponents. The basic method is to play the strategy
against a probe strategy with varying noise parameters. In~\cite{Ashlock2005}
Tit for Tat is used as the probe strategy. Fingerprint functions
can then be compared to allow for easier identification of similar strategies.
In Figure~\ref{fig:fingerprinting} an example of Pavlov's fingerprint is given.
Fingerprinting has been studied in depth in~\cite{Ashlock2008, Ashlock2009,
Ashlock2010, Ashlock2006a}.

\begin{figure}[!hbtp]
    \centering
    \includegraphics[height=.3\textheight]{./assets/images/Win-Stay_Lose-Shift.png}
    \caption{Pavlov fingerprinting with Tit for Tat used as the probe strategy.
    Figure was generated using~\cite{axelrodproject}.}
    \label{fig:fingerprinting}
\end{figure}

In~\cite{Li2011} \textbf{APavlov},
which stands for adaptive Pavlov, made an appearance.  The strategy attempts to
classify the opponent as one of the following strategies, All Cooperator,
All Defector, Pavlov, Random or \textbf{PavlovD}. PavlovD, is just Pavlov
but it starts the game with a \textbf{D}. Once Adaptive Pavlov has classified
the opponent plays to maximize it's payoff.
In 2011 the authors of~\cite{Li2011} performed their own tournament where
several interesting strategies made an appearance.

\begin{itemize}
    \item \textbf{Periodic player CCD}, plays \textbf{C}, \textbf{C}, \textbf{D}
    periodically. Note that variations of a period player also make appearance
    in the article but will not be listed here.
    \item \textbf{Prober}, starts with the pattern \textbf{D}, \textbf{C}, \textbf{C}
     and then defects if the opponent has cooperated in the second and third move;
     otherwise, it play as Tit for Tat.
    \item \textbf{Reverse Pavlov}, a strategy that does the reverse of Pavlov.
\end{itemize}

In 2012 Press and Dyson~\cite{Press2012} studied the iterated prisoner's dilemma and presented
a new set of strategies called \textbf{zero determinant (ZD)}. The ZD strategies
are memory one strategies that manage to force a linear relationship between their
score and that of the opponent.
In Section~\ref{section:responses_to_computer_tournament} it was described how
the payoffs of two players could be retrieved by formulating their interactions
using a Markov chain. Let us denote the payoffs of players \(p\) and \(q\) as:

\begin{align*}
    s_p = v S_p \\
    s_q = v S_q
\end{align*}

where \(v\) is a vector of the steady states of matrix \(M\) and \(S_p\), \(S_q\)
are the equivalent payoff values of the players for each state \(CC, CD, DC, DD\).
Using linear algebra, Press and Dyson showed that the dot product of the stationary
distribution of \(v\) with any vector \(f\) can be expressed as a \(4\times 4\)
determinant. In which one column is \(f\), one column is entirely under the control
of player \(p\) and another column is entirely under the control of player \(q\).

This meant that either \(p\) or \(q\) could independently force the dot product
of \(v\) with some other chosen vector \(f\) to be zero by choosing their
strategy so as to make the column they control be proportional to \(f\).
In particular, by \( f = \alpha S_p + \beta S_q + \gamma\), any player can force
a given linear relation to hold between the long-run scores of both players.

Press and Dyson's results suggested that the best strategies were selfish ones
that led to extortion, not cooperation. Arguing with Axelrod's reports.
All the more, their work stated that in the iterated prisoner's dilemma, memory
is not advantageous.

The ZD strategies have attracted a lot of attention. It was stated that
``Press and Dyson have fundamentally changed the viewpoint on the Prisoner's
Dilemma''~\cite{Stewart2012}. In~\cite{Stewart2012}, they ran a variant of
Axelrod's tournament with 19 strategies to test the effectiveness of 
ZD strategies. While conducting their tournament they have implement several
strategies discussed by~\cite{Press2012} and revealed a set of generous ZDs
the~\textbf{Generous ZD}.

In~\cite{Lee2015}, the `memory of a strategy does not matter' statement was
questioned. A set of more complex strategies, strategies that take in account
the entire history set of the game, were trained and proven to be more stable than
ZD strategies. Complex strategies were also studied by~\cite{Knight2017,KnightHGC17}.
This was done using an open source package, called the Axelrod project~\cite{axelrodproject}
which launched on 2015.

The project is written in the programming language
Python, it is accessible and open source. To date the list of strategies implemented
within the library exceed the 200. The project has been used in several
publications including~\cite{Knight2017} and a paper describing it and
it's capabilities was published in 2016~\cite{Knight2016}.

The two paper using the Axelrod project~\cite{Knight2017, KnightHGC17} present
several powerful strategies created using reinforcement learning techniques.
Reinforcement learning refers to a collection of algorithms that trains a model by
exploring a space of actions and evaluating consequences of those actions. In these
papers the authors used genetic algorithms and particle swarm optimisation
algorithms~\cite{suganthan1999}. A number of strategy representations, referred as
archetypes, were used to train strategies. These included, lookerup tables,
finite state machines, artificial neural networks~\cite{yegnanarayana2009} and
hidden Markov models~\cite{eddy1996}.

Hidden Markov models, are a variant of a finite state machine that use probabilistic
transitions based on the prior round of play to other states and cooperate or
defect with various probabilities at each state. Finite state machines and hidden Markov models
based strategies are characterized by the number of states. Similarly, artificial
neural networks based players are characterized by the size of the hidden layer
and number of input features.

Additionally a variant of a look up table is also presented called the lookerup
archetype. The lookerup archetype responses based on the opponent's first \(n_1\)
moves, the opponent's last \(m_1\) moves, and the players last \(m_2\) moves.
Taking into account the initial move of the opponent can give many insights.
For it is the only move a strategy is truly itself without being affected by
the other player. As a reminder, Axelrod in his work highlighted the importance
of the initial move and believed that it was one of the secrets of success of the
strategy Tit for Tat. Finally, a new archetype called the Gambler is also introduced,
which is a stochastic variant of the lookerup archetype.

The training of these archetypes was done in two following settings:

\begin{itemize}
    \item A Moran process, which is an evolutionary model of invasion and resistance across
    time during which high performing individuals are more likely to be replicated.
    \item A round robin tournament.
\end{itemize}

The result of~\cite{KnightHGC17} show that the trained strategies evolve an ability
to recognise themselves by using a handshake. This characteristic of the strategies
was an important one because in a Moran process this recognition mechanism allowed these
strategies to resist invasion. In~\cite{Knight2017}, they performed a
a standard tournament with 200 turns but also a noisy tournament. For the standard
tournament the newly introduced trained strategies outperform the designed ones.
In the case of noise there is one particular strategy that has not seen much
attention in the literature called ``Desired Belief Strategy''~\cite{Au2006}.
These experiments are, to the authors knowledge, the biggest ones done in the
field in terms of different strategies.

\subsubsection{Software}

Though several of this tournament discussed so far were generated using computer
code not all of the source code was made available by the authors. Several
open projects were created and published through the year. The first one
discussed in this work, excluding the code for Axelrod's strategies, is 
PRISON~\cite{prison}. PRISON is written in the programming language Java and
preliminary version was launched on 1998. It was used by it's authors in several
publications, such as~\cite{Beaufils1997} which introduced Gradual and~\cite{Beaufils1988}.
The project includes a good number of strategies from the
literature but unfortunately the last update of the project dates back in 2004.

Other recent software projects include~\cite{pd_trust, pd_game}, both are education
platforms and not research tools. In~\cite{pd_trust}, several concepts such as
the iterated game, computer tournaments and evolutionary dynamics are introduced
through a user interface game. Project~\cite{pd_game} offers a big collection of
strategies and allows the user to try several matches and tournament configurations.

Finally, as described in Section~\ref{section:modern_approaches}, the open source
package Axelrod. Axelrod package is a software written following best practice
approaches and contains the larger to date data set of strategies. The strategy
list of the project has been cited by publications~\cite{Neumann2018} and
the package has not been used only by the contributors for academic research but
from several such as:~\cite{Latorre2018}.

% QUESTION: should add all the other publications? Even though I do not
% discuss them?

\subsection{Contemporary period (2017 - 2018)}\label{section:contemporary_period}

In recent years the study of the iterated prisoners' dilemma is still active
and papers are still being published. New strategies, new variants of the game
and new applications are being introduced every year. In this section we will
briefly review some articles that have been published between 2017 and 2018.

The iterated prisoner's dilemma serves as a model in a wide range of applications.
For example in~\cite{Kaznatchee2017} cancer cells and how they can resist treatment has been
modelled using evolutionary approaches described in Section~\ref{section:evolutionary_dynamics}.
Furthermore, in~\cite{Dridi2018} they explore whether learning in social situations
can be driven by rewards.

A lot of work has been done on evolutionary dynamics on structured populations.
This is mainly because the applications they can offer in real life prolem,
such as social interactions.
In~\cite{Liu2017} the authors consider the evolutionary spatial prisoner dilemma with
memory one strategies and their results indicate that a Pavlov like behaviour
is stable and dominant. But how does cooperation evolve in structure situations
and more specifically in situations where the number of neighbours can vary?
In~\cite{Allen2017} the authors tried to shed some light to this question.
However this was done for weak selection. Their results argue that, by considering
the coalescence times of random walks of any given graph they can approximate if
cooperation will emerge.

Several variations to the actual game are still being introduced. Ohtsuki in~\cite{Ohtsuki2018},
studies the NPD, briefly introduced in Section~\ref{section:responses_to_computer_tournament}.
Ohtsuki propose a model called coordinated cooperation. It's an NPD game
which starts with is a negotiation before an actual game is played. Each individual
can flexibly change their decision, either to cooperation or to defection,
according to the number of those who show the intention of cooperation/defection.
This NPD model was introduced in order to provide one explanation of why
people tend to take into account others' decisions even when doing so gives them
no payoff consequences at all.

\subsection{Conclusion}

In Section~\ref{section:timeline} we covered a literature review on the prisoner's
dilemma research. We reviewed articles from 1950 to 2018. This was achieved by
partitioning the timeline in six different sections. Here we will briefly give an
overview of the outcomes presented by each of these sections.

In Section~\ref{section:origin}, we covered the origins of the game and some early
research that was conducted back in the 60s-70s. This research was very limited
and constrained. The only experiments researchers could conduct were experiments
that used humans to simulate rounds of the game. Even such experiments are still
being conducted today, most researchers prefer the use of computer tournaments instead.

Computer tournaments were introduced in 1980s, and their introduction was
covered in Section~\ref{subsection:axelrods_tournament}. Axelrod performed several
tournaments in the years 1980 to 1988. All of these have been summarized and presented.
His work provided proof that reciprocity, more specifically Tit for Tat, was
a very successful and robust way of playing the game. However, his work did not
come without criticism.

The main criticism was that Axelrod did not account for uncertainty behaviour that
could occur in the interactions between strategies. Several works, discussed
in Section~\ref{section:responses_to_computer_tournament} proved that
Tit for Tat was not robust enough in such environments. On the contrary, the
most robust strategies were strategies that would  increase or decrease, based on
the type of uncertainty, the willingness to cooperate or to defect.

Even in evolutionary research Axelrod's results were questioned. Though there is
not a clear answer to which strategies are ESS in the iterated prisoner's dilemma,
several results were introduced in Section~\ref{section:evolutionary_dynamics}.
In unstructured populations, Boyd and Lorderbaum showed that if importance of the
future is large enough, then no deterministic strategy can be ESS. Structure
populations appear to allow cooperation to emerge more. For weak selection it has
been shown that we can approximate if cooperation will emerge just by looking
the degree of the graph.

In Section~\ref{section:modern_approaches}, we presented a number of different
tournaments that were performed and how the community still aims to identify a
dominant strategy or even factors that make a strategy successful. Axelrod spoke
about simplicity and kindness but through out research people had different
results to show. Zero determinant strategies, were extortionate strategies and
their success lead the field to believe that exploiting individual opponents can
be more beneficial than being nice. In more recent years, more complex strategies
using modern methods of both representation and training, are being introduced.
These strategies are not easy to described. However, we can observe the sort
of characteristics these strategies have developed, on their own, and gain
a better understanding of what makes a strategy successful.

In the final section of this literature review, Section~\ref{section:contemporary_period},
we very briefly discussed some recent research that was published in 2017-2018.
Though several other articles have been published in the past few months, the aim of the
section was to demonstrate that the field is still an active one. This will become
more evident in Section~\ref{section:analysis}.

\section{Analysing a large corpus of articles}\label{section:analysis}

% In this section we will focus on the analysis of the study of the prisoner's dilemma
% using a large dataset of articles. In Section~\ref{section:data_collection} the data
% will be described and analysed in Section~\ref{section:preliminary_analysis}. In Section
% \ref{section:co_authorship_analysis} the author relationships will be analysed graph
% theoretically to ascertain the level of collaborative nature of the field and identify
% influencer. This will be done relative to:

% \begin{itemize}
%     \item two other sub fields of game theory: auction games~\cite{menezes2005} and 
%     the price of anarchy~\cite{roughgarden2005}.
%     % \item A temporal analysis. Removing the temporal stuff for now.
% \end{itemize}

% \subsection{Data Collection}\label{section:data_collection}

% Academic articles are accessible through scholarly databases and collections of
% academic journals. Several databases and collections today offer access through
% an open application protocol interface (API). An API allows users to query
% directly a journal's database and bypass the user interface side of the journal.
% Interacting with an API has two phases: requesting and receiving.

% The request phase includes composing a url with the details of what is wanted. Figure~\ref{fig:request_message}
% presents an example of a request message. The first part of the request is the address
% of the API we are querying. In this example the address corresponds to the API of arXiv.
% The second part of the request contains the search arguments. In our example we
% are requesting for a single article that the word `prisoners dilemma' exists within
% it's title. The format of the request message is different from API to API.

% The receive phase includes receiving a number of raw metadata of articles that
% satisfied the request message. The raw metadata are commonly received in extensive markup
% language (xml) or Javascript object notation (json) format~\cite{nurseitov2009}.
% Similarly to the request message, the structure of the received data differs from journal
% to journal.

% \begin{figure}[!hbtp]
%     \centering
%     \begin{minted}
%         [
%         autogobble=true,
%         framesep=2mm,
%         fontsize=\normalsize,
%         ]
%         {xml}
%         http://export.arxiv.org/api/query?search_query=abs:prisoner's dilemma&max_results=1
%     \end{minted}
%     \caption{\label{fig:request_message} A request message for the arXiv API.}
% \end{figure}

% The data collection is crucial to this study. To ensure that this study can be
% reproduced all code used to query the different APIs has been packaged as a Python library and is
% available online~\cite{nikoleta_2017}. The software could be used for any type of
% projects similar to the one described here, documentation for it is available at:
% \url{http://arcas.readthedocs.io/en/latest/}.

% Project~\cite{nikoleta_2017} allow us to collect articles from a list of APIS by
% specifying just a single keyword. The following sources were used to collect data
% for this analysis:

% \begin{enumerate}
%         \item PLOS~\cite{plos_one};
%         \item Nature~\cite{nature};
%         \item IEEE~\cite{ieee};
%         \item Springer~\cite{springer}.
%         \item arXiv~\cite{mckiernan2000};
% \end{enumerate}

% These are four prominent journals in the field, as well as the pre print server arXiv~\cite{mckiernan2000}.
% In the case of an article being both in a journal and the arXiv, only the journal version
% was considered.

% For each article~\cite{nikoleta_2017} collects a list of the features, shown in Table~\ref{table:arcas_results}.
% Note that the plain text of the article is not collected, just the metadata. The
% data is archived and available at. %TODO: archive data
% In this work only the features of Table~\ref{table:result_set} are used.

% \begin{table}[!hbtp]
%     \begin{center}
%     \resizebox{0.9\linewidth}{!}{\arraycolsep=2.5pt%
%         \begin{tabular}{lll}
%             \toprule
%              & Result name & Explanation \\
%              \midrule
%              1 & Abstract & The abstract of the article.\\ 
%              2 & Author & A single entity of an author from the list of
%              authors of the respective article. Thus there are multiple entries for each article.\\ 
%              3 & Date & Year of publication.\\ 
%              4 & Journal & Journal of publication.\\ 
%              5 & Key & A generated key containing an authors name and
%              publication year (ex. Glynatsi2017).\\ 
%              6 & Keyword & A single entity of a keyword assigned to the article
%              by the given journal.\\ 
%              7 & Labels & A single entity of labels assigned to the article
%              manual by us.\\ 
%              8 & Pages & Pages of publication.\\ 
%              9 & Provenance & Scholarly database for where the article was
%              collected.\\ 
%              10 & Score & Score given to article by the given journal.\\ 
%              11 & Title & Title of article.\\ 
%              12 & Unique key &  A unique hash. \\ 
%             \bottomrule
%         \end{tabular}}
%     \end{center}
%     \caption{Metadata for each entry.}
%     \label{table:arcas_results}
% \end{table}

% \begin{table}[!hbtp]
%     \begin{center}
%         \begin{tabular}{lll}
%             \toprule
%              & Result name & Explanation \\
%              \midrule
%              1 & Abstract & The abstract of the article.\\ 
%              2 & Author & A single entity of an author from the list of
%              authors of the respective article.\\ 
%              3 & Date & Year of publication.\\ 
%              4 & Journal & Journal of publication.\\ 
%              5 & Provenance & Scholarly database for where the article was
%              collected.\\ 
%              6 & Title & Title of article.\\ 
%             \bottomrule
%         \end{tabular}
%     \end{center}
%     \caption{Structure of data set used for this work.}
%     \label{table:result_set}
% \end{table}

% A series of keywords were used to identify relevant articles. Articles for which
% any of these keywords existed within the title or the abstract are included in the
% analysis. The keywords used to collect the main data set were,

% \begin{itemize}
%     \item ``prisoner's dilemma'',
%     \item ``prisoners dilemma'',
%     \item ``tit-for-tat'',
%     \item ``tit for tat'',
%     \item ``zero determinant strategies''.
% \end{itemize}

% As will be described in Section~\ref{section:preliminary_analysis}, two other
% game theoretic subfields were also considered in this work, auction games and the
% price of anarchy. For collecting data on these subfields the following keywords were used:

% \begin{itemize}
%     \item key: ``auction game theory'';
%     \item key: ``price of anarchy''.
% \end{itemize}

% For both of these topics only a single keyword has been used. In comparison 5
% different keywords were used to search of articles on the prisoner's dilemma.
% The amount of articles collected from the key such as `tit for tat' and
% `zero determinant' had a small contribution to the size of the data set.

% \subsection{Preliminary Analysis}\label{section:preliminary_analysis}

% A total of three data sets are explored in this work. A summary of each data is
% presented in this section. The three data sets are:

% \begin{itemize}
%     \item The main data set which contains articles on the prisoner's dilemma.
%     \item A secondary data set which contains article on auction games.
%     \item A secondary data set which contains articles on the price of anarchy.
% \end{itemize}
% % TODO archive all data sets

% \subsubsection{The prisoner's dilemma data set}

% The main data set and the main focus of this analysis. This data set
% consists of \totalarticles articles, where \uniquetitles have unique titles.
% %TODO add reference to archived dataset
% This is because a total of \numberofduplicates articles have been collected from
% both a journal and arXiv. All duplicates from arXiv are dropped, thus hereupon
% we consider \uniquetitles unique article entries.

% Of these \totalarticles \manual articles that have not been collected from the
% aforementioned APIs. These articles were of specific interest and manually added to the
% dataset throughout
% the writing of Section~\ref{section:timeline}. A more detailed summary of the 
% articles' provenance is given in Table~\ref{table:provenance}.
% The larger number of articles were collected from arXiv, Springer and IEEE. Both
% Nature and PLOS have a small contribution to the size of the data set. The oldest
% article was published in 1944 and the most recent one in 2017. Note
% that the latest data collection was on December 2017.
% % TODO Ensure this stay accurate

% \begin{table}[!hbtp]
%     \begin{center}
%     \begin{tabular}{lrr}
\toprule
{} &  \# of Articles &  Percentage \\
provenance &                &             \\
\midrule
Manual     &             89 &        2.88 \\
IEEE       &            295 &        9.55 \\
PLOS       &            482 &       15.60 \\
Springer   &            572 &       18.52 \\
Nature     &            673 &       21.79 \\
arXiv      &           1056 &       34.19 \\
\bottomrule
\end{tabular}

%     \end{center}
%     \caption{Articles' provenance for main data set.} % Use citation when archived
%     \label{table:provenance}
% \end{table}

% Not all journal have existed for the same ammount og time, so thus calculate the average
% publication over time. This is done for the overall data set and for each journal
% individually. This is denoted as,

% \[ \mu_P = \frac{N_A}{N_Y},\]

% where \(N_A\) is the total number of articles and \(N_Y\) is the years of publication.
% The years of publication is calculated as the range between 2017 and the first published
% article within the data.

% Table~\ref{table:publication_rates} summarises these averages. Overall an average of
% 21 articles are published per year on the topic. The most significant contribution
% to this appears to be from arXiv with 8 articles per year, followed by Springer
% with 5 articles per year.

% \begin{table}[!hbtp]
%     \begin{center}
%     \begin{tabular}{lr}
\toprule
{} &  Average Yearly publication \\
\midrule
IEEE     &                         5.0 \\
PLOS     &                         8.0 \\
Springer &                         9.0 \\
Nature   &                        11.0 \\
arXiv    &                        16.0 \\
Overall  &                        49.0 \\
\bottomrule
\end{tabular}

%     \end{center}
%     \caption{Average publication for main data set.} % Use citation when archived
%     \label{table:publication_rates}
% \end{table}

% Though the average publication offers insights about the publications of the
% fields, it is still a constant number. The data we are handling here is a time
% series which appears to have a trend. This is shown
% by calculating the rolling average which is plotted in Figure~\ref{fig:timeseries}.
% The rolling average of each time
% point is calculated as the average of the points on either side of it.

% The rolling average indicates that the time series has an increasing trend.
% Even so there seems to be a small decrease by the last time point. In order
% to offer some insights as to what expected of the field in the next years
% we conduct a forecast for the next time periods.

% Initially we test for stationarity using an Augmented Dickey-Fuller test
% \cite{HARRIS1992381}. A time series data is said to be stationary if its
% statistical properties such as mean and variance remain constant over time.
% The results show that our data are not significantly stationary at the 0.005 level
% (\(p > 0.005\)).

% For projecting the behaviour of the field of the new years we are using an ARIMA~\cite{brockwell2013}
% model which is able to handel non stationary data. The parameters
% of the model have been fitted using the Akaike Information Criterion value.
% The model used was ARIMA \((1, 1, 1)\) at the forecast for the next 10 time
% periods are given by Table~\ref{table:forecast}.

% \begin{figure}[!hbtp]
%     \centering
%     \includegraphics[width=0.6\textwidth]{./assets/images/timeseries.pdf}
%     \caption{Time series of main data set and the rolling average.}\label{fig:timeseries}
% \end{figure}

% \begin{table}[!hbtp]
%     \begin{center}
%     \begin{tabular}{lr}
\toprule
{} &  Forecast \\
\midrule
2017 &     371.0 \\
2018 &     423.0 \\
2019 &     483.0 \\
2020 &     550.0 \\
2021 &     627.0 \\
\bottomrule
\end{tabular}

%     \end{center}
%     \caption{Forecasting the number of publications over the next 10 years.}
%     \label{table:forecast}
% \end{table}

% Thought the time series has indicated a slight decrease we can see that the
% model forecasts an increase over the next years.

% \subsubsection{Auction games and the price of anarchy data sets}

% Two subfields of game theory are chosen for this work; auction game and the
% price of anarchy. A summary of both data sets collected on this two topics %TODO: cite
% is given by Table~\ref{table:summary_other_topics}.

% A total of 2103 articles with 3860 unique authors are examined for auction games.
% Auction games is well studied topic with the earliest entry going
% back to 1974. In comparison, 296 unique articles have been collected on price of
% anarchy. The earliest entry being in 2003 and a total of 668 unique authors have
% written about the topic.

% In Figure~\ref{fig:timeplots_other_topics} a time plot for each topic is
% displayed and is exhibited that both topics have had an increasing trend over
% the years. Though price of anarchy is clearly a new topic compared to auction games.

% The frequency of the prisoner's dilemma, for both articles and authors, lies
% between the frequencies of these two topics.

% \begin{table}[!hbtp]
%     \begin{center}
%     \begin{tabular}{llr}
%         \toprule
%          &            Price of anarchy &  Auction games \\
%         \midrule
%         Unique articles      & 296  & 2103 \\
%         Unique authors       & 668  & 3860 \\
%         Min publication year & 2003 & 1974 \\
%         Max publication year & 2017 & 2017 \\
%         \bottomrule
%     \end{tabular}
%     \end{center}
%     \caption{Secondary data sets summaries.}
%     \label{table:summary_other_topics}
% \end{table}

% \begin{center}
% \begin{figure}[!hbtp]
%     \begin{subfigure}{0.5\textwidth}
%         \includegraphics[width=\textwidth]{./assets/images/anarchy_timeline.pdf}
%     \end{subfigure}
%     \begin{subfigure}{0.5\textwidth}
%         \includegraphics[width=\textwidth]{./assets/images/auction_timeline.pdf}
%     \end{subfigure}
% \caption{Time plots for secondary data sets.} % TODO cite
% \label{fig:timeplots_other_topics}
% \end{figure}
% \end{center}

% The provenance of the articles is given by Table~\ref{table:provenance_other_topics}.
% Almost 1500 article for auction games have been collected from Springer, that is
% more than three times the articles that have been collected from other sources.
% PLOS and Nature have a minor contribution and PLOS and Nature had no
% articles on the price of anarchy.

% \begin{table}[!hbtp]
%     \begin{center}
%     \begin{tabular}{lcc}
%         \toprule
%         \textbf{Provenance} & \textbf{Total articles} & \textbf{Total articles}\\
%                    & (auction games) & (price of anarchy)\\
%         \midrule
%         Springer   &                1429 &                78 \\
%         arXiv      &                 436 &                108 \\
%         IEEE       &                 301 &                131 \\
%         PLOS       &                  15 &                 -  \\
%         Nature     &                   1 &                 -   \\
%         \bottomrule
%     \end{tabular}
%     \end{center}
%     \caption{Articles' provenance for secondary data sets.}
%     \label{table:provenance_other_topics}
% \end{table}

% \begin{table}[!hbtp]
%     \begin{center}
%     \begin{tabular}{lcc}
%         \toprule
%         \textbf{Provenance} & \textbf{Av. publication}  & \textbf{Av. publication}\\
%                             & (auction games) & (price of anarchy)\\
%         \midrule
%         Overall             &          58.973 & 19.812 \\
%         Springer            &          38.622 & 4.875  \\
%         arXiv               &          11.784 & 6.750   \\
%         IEEE                &           8.135 & 8.188    \\
%         PLOS                &           0.405 &   -       \\
%         Nature              &           0.027 &   -        \\
%         \bottomrule
%     \end{tabular}
%     \end{center}
%     \caption{Average publication for auction games and the price of anarchy.}
%     \label{table:other_topics_publication_rates}
% \end{table}

% The overall average publication for auction games and the price of anarchy are 59 and 20 articles
% respectively. It appears that auction games publication is largely different
% for both the prisoner's dilemma and the price of anarchy. These two topics have
% the same average publication. Note that the significance of each journal differs
% from topic to topic. Though this analysis will not focus on individual sources
% from hereupon.

% % \paragraph{Temporal analysis}
% % \mbox{ }\\

% % For comparison reasons in the following subsections the analysis will also be
% % held relative to a temporal analysis. The main data set~\cite{}
% % is divided into time period according to the subsections of Section~\ref{section:timeline}.
% % The respective measures of unique titles and unique authors for each period is
% % given by Table~\ref{table:summary_temporal}.

% % \begin{table}[!hbtp]
% %     \begin{center}
% %     \begin{tabular}{lrr}
\toprule
{} &  Unique articles &  Unique authors \\
\midrule
period 1 &               21 &              38 \\
period 2 &                5 &               6 \\
period 3 &               64 &              70 \\
period 4 &              121 &             169 \\
period 5 &              926 &            1730 \\
period 6 &              453 &            1008 \\
period 7 &              180 &             466 \\
\bottomrule
\end{tabular}

% %     \end{center}
% %     \caption{Periods and their respective measures.}
% %     \label{table:summary_temporal}
% % \end{table}

% In this section we have described the three data sets that we are going to use
% in the following sections in order to identify collaborative behaviour and influence.
% Two data sets of different topics are used for comparison reasons. The frequency of
% articles and authors differs within the three data sets which is ideal.
% % Finally a temporal analysis of the data sets~\cite{} will also assists us in
% % obtaining more insights. The period have also be presented here.


% \subsection{Co authorship Analysis}\label{section:co_authorship_analysis}

% Most academic research is undertaken in the form of
% collaborative effort. As discussed in~\cite{Kyvik2017}, it is rationale that two
% or more people have the potential to do better as a group than individually. Academic collaborations
% have many different forms. Researchers might have immediately collaborated and
% written together. Others might have collaborated through a common co author.

% Collaboration in groups has a long tradition in experimental sciences and it has
% be proven to be productive according to~\cite{Etzkowitz1992}. Even so, the number of collaborations
% can be very different between research fields and measuring collaboration is
% not always an easy task. Another aspect of collaborative behaviour is influence. For
% example academics can influence through workshops, talks or by collaborating with people
% in our environment.

% Several studies tend to consider academic citations as a measure for these things.
% As discussed in~\cite{nature_blog}, depending on citations can often be misleading.
% This is because:

% \begin{itemize}
%     \item The true number of citations can not be known. Citations can be missed
%     due to data entry errors.
%     \item Academics are influenced by many more papers than they actually cite.
%     \item Several citations are superficial.
% \end{itemize}

% We suggest an alternative measure of collaboration and influence by looking at the co
% authorship network. A co authorship network,
% is a network where academics that have written and published together are connected.

% Using graph theoretic concepts this network will be analysed to undestand:

% \begin{itemize}
%     \item Collaborativeness; for example the number of connections an author has
%     as well as more sophisticated measures of closeness.
%     \item Influence; how many connections are made possible because of an author.
% \end{itemize}

% We introduced several network measures that we will be using such as:

% \begin{itemize}
%     \item Number of connected components.
%     \item Clustering coefficient.
%     \item Degree distributions.
%     \item Centrality.
% \end{itemize}

% \subsubsection{Constructing a co authorship network}

% To construct a co authorship network we need to consider all the unique authors.
% The issue with retrieving the unique authors is that authors names can be written
% in different ways in different sources. For example consider the author of this
% paper:

% \begin{itemize}
%     \item Nikoleta Glynatsi
%     \item Nikoleta E. Glynatsi
%     \item Nikoleta Evdokia Glynatsi
% \end{itemize}

% Consequently, several different entries of the same author existed within the data
% set. Thus we wanted to figure out when two author names were the same in real life.
% Though identifying if two string correspond to the same author is human possible
% the data sets consited of more than 1000 authors, thus we wanted to automate the
% procedure.

% This was done by using the Levenshtein Distance~\cite{miller2009}. The Levenshtein
% Distance is a metric for measuring the difference between two sequences. It is
% based upon the number of actions one has to take to transform one string into
% the other. These actions include:

% \begin{enumerate}
%     \item Insertion;
%     \item Deletion;
%     \item Substitution of a single character.
% \end{enumerate}

% Let us consider an example where we are trying to calculate the distance between
% the two strings. These are `Wang' and `Yang'. To compute the distance in a non-recursive
% way, we use a matrix \(D\) containing the distances between all the prefixes of the
% two strings. The first row and column are indexed by empty strings. The rest of
% the rows and columns are index by the prefixes of the two strings.

% The matrix is filled from left to right. The first row is filled as follows:

% \begin{enumerate}
%     \item To go from an empty string to an empty string zero actions are needed.
%     Thus the \(D_{\text{e}, \text{e}}\) is 0.
%     \item To go from an empty string to `W', or the other way around, 1 action
%     is needed. Thus the \(D_{\text{e}, \text{W}}\) is 1.
%     \item For every new letter we have to take another action \((+1)\).
% \end{enumerate}

% Similarly, this is done for the first columns. For rest of the elements
% we follow a similar approach, but this time the previous distances are also
% taken in account. For example, \(D_{\text{Y}, \text{W}}\). For the letter `Y'
% to go to `W' a single action is required. Note that now 1 is added to the minimum
% distance between of \(D_{\text{e}, \text{e}}, D_{\text{e}, \text{W}}\) and
% \(D_{\text{e}, \text{Y}}\). 

% Similarly we fill the rest of matrix. The last value computed, bottom right,
% is the Levenshtein Distance of the two strings. In our example it is calculated
% to be 1.

% \[ D = \kbordermatrix{
%      & \text{e} & \text{W} & \text{A} & \text{N} & \text{G} \\
%    \text{e} & 0 & 1 & 2 & 3 & 4 \\
%     \text{Y} & 1 & 1 & 2 & 3 & 4 \\
%     \text{A} & 2 & 2 & 1 & 2 & 3 \\
%     \text{N} & 3 & 3 & 2 & 1 & 2 \\
%     \text{G} & 4 & 4 & 3 & 2 & 1
%   }\]

% In this work we calculate the ratio of two string matching for all possible
% pairs of authors in the data sets. The matching ratio is calculated as,

% \[(1 - \frac{\text{lev}}{m}) \times 100,\]

% where \(\text{lev}\) is the distance and \(m\) is the length of the longest of
% the two words. If the ratio of a pair was between \(85\) and \(99\) both entries
% were highlighted. The highlighted entries were manually checked to assure that
% there were indeed the same author and then one of them was replaced by the other.

% For example all entries with author name written as example ``Y. Moreno'' were
% replaced by ``Yamir Moreno''.

% The manual check is performed because not all highlighted entries are indeed the
% same. For example:

% \begin{enumerate}
%     \item Zhen Yang and
%     \item Zhen Wang
% \end{enumerate}

% are two different authors. Once the name entries have been cleaned the
% co authorship networks can be defined. The definition of a co authorship network
% is given by:

% \begin{definition}{Co authorship network.}
%     A co authorship network is an undirected network \(G\) of vertices \(V\) and
%     edges \(E\) where vertices representing each unique author and an edge
%     connects two authors if and only if those authors have written together.
%     No weight has been applied to the edges nor the nodes.
% \end{definition}

% %Define what a network is and then G_1, G_2, G_3
% The three networks considered:

% \begin{itemize}
%     \item \(G_1\), the prisoner's dilemma network, where \(V(G_1)=\) \authors and
%     \(E(G_1)=\) \edges.
%     \item \(G_2\), the auction games network, where \(V(G_2)=\) \auctionauthors and
%     \(E(G_2)=\) \auctionedges.
%     \item \(G_3\), the auction games network, where \(V(G_3)=\) \priceauthors and
%     \(E(G_3)=\) \priceedges.
% \end{itemize}

% The respective illustrations of \(G_1, G_2\) and \(G_3\) are given by Figures
% \ref{fig:authors_network} and \ref{fig:co-authorship-other-topics}.

% \begin{figure}[!hbtp]
%     \centering
%     \includegraphics[width=0.8\textwidth]{./assets/images/co-authors-network.pdf}
%     \caption{Co authorship network for the prisoner's dilemma.}\label{fig:authors_network}
% \end{figure}

% \begin{center}
%     \begin{figure}[!hbtp]
%         \begin{subfigure}{0.5\textwidth}
%             \includegraphics[width=\textwidth]{./assets/images/co-authors-network-auction.pdf}
%             \caption{Co authorship network for auction games.}
%         \end{subfigure}
%         \begin{subfigure}{0.5\textwidth}
%             \includegraphics[width=\textwidth]{./assets/images/co-authors-network-price.pdf}
%             \caption{Co authorship network for the price of anarchy.}
%         \end{subfigure}
%     \caption{Co authorship network for secondary data sets.}
%     \label{fig:co-authorship-other-topics}
%     \end{figure}
%     \end{center}

% \subsubsection{Measures of collaboration}

% In this section we ascertain the level of collaborative nature of the field.
% This is measured as the connections authors can
% have within their groups. Moreover, how strongly connected these groups are.
% Several connectivity measures will be used to explain such behaviour which are
% introduced through various examples.

% The first measure introduced is the \textbf{number of connected components}. A connected
% component of an undirected graph is a maximal set of nodes such that each pair
% of nodes is connected by a path. Two examples are illustrated in Figure
% \ref{fig:connected_components}. These are two different sub graphs of \(G_1\)
% with a number of connected components of 1 and 5.

% \begin{center}
% \begin{figure}[!hbtp]
%     \begin{subfigure}{0.5\textwidth}
%         \includegraphics[width=\textwidth]{./assets/images/connected_example_one.pdf}
%         \caption{A sub graph of \(G_1\) with 1 connected component.}
%     \end{subfigure}
%     \begin{subfigure}{0.5\textwidth}
%         \includegraphics[width=\textwidth]{./assets/images/connected_example_two.pdf}
%         \caption{A sub graph of \(G_1\) with 5 connected component.}
%     \end{subfigure}
% \caption{Connected components examples.}
% \label{fig:connected_components}
% \end{figure}
% \end{center}

% Note that a vertex with no incident edges is itself a connected component.
% The number of connected components gives a naive measure of how disjoint the network is.
% In essence the number of groups in the field.

% The second measure considered is the \textbf{degree}. The degree of a node express
% the number of connections a node has. We will consider the degree distribution
% of a network. It will allow us to understand the mean connection that authors
% have in the network's groups.

% The final measure is the \textbf{clustering coefficient}. The clustering coefficient is a
% measure of the degree to which nodes in a graph tend to cluster together. There
% are two types of this measure; the local and the global coefficients. The local
% coefficient is the clustering coefficient of a single vertex. It is calculated as,

% \[C_u = \frac{2 \times L_u}{(k_u (k_u - 1)},\]

% where \(k_u\) is the degree of vertex \(u\) and \(L_u\) is the number of edges
% between \(k_u\) neighbours of vertex \(u\).

% The global coefficient, \(\bar{C}\), is calculated by averaging all the local
% coefficients of the graph. The values of the measure can range between \(0\) and
% \(1\). Figure~\ref{fig:clustering_coefficients} illustrates several sub graphs
% with different \(\bar{C}\) values.

% \begin{center}
%     \begin{figure}[!hbtp]
%         \begin{subfigure}{0.33\textwidth}
%             \includegraphics[width=\textwidth]{./assets/images/clustering_example_one.pdf}
%             \caption{A sub graph of \(G_1\) with a global clustering coefficient of 0.}
%         \end{subfigure}
%         \begin{subfigure}{0.33\textwidth}
%             \includegraphics[width=\textwidth]{./assets/images/clustering_example_two.pdf}
%             \caption{A sub graph of \(G_1\) with a global clustering coefficient of 0.23.}
%         \end{subfigure}
%         \begin{subfigure}{0.33\textwidth}
%             \includegraphics[width=\textwidth]{./assets/images/clustering_example_three.pdf}
%             \caption{A sub graph of \(G_1\) with a global clustering coefficient of 1.}
%         \end{subfigure}
%     \caption{Clustering coefficients examples.}
%     \label{fig:clustering_coefficients}
%     \end{figure}
%     \end{center}

% A clustering coefficient of \(1\) indicates that a graph is a
% complete graph. On the contrary, a coefficient of \(0\) indicates that authors write
% with just a single co author.
% % A high clustering coefficient indicates that people within groups have several
% % connections. There are not there due to a single publication with a random researcher
% % but on the contrary that group is said to be collaborative.

% \paragraph{Analysis}
% \mbox{ }\\

% All connectivity measures are calculated for \(G_1, G_2\) and \(G_3\). This is
% done using the open source package networkx~\cite{networkx}.

% We are aware that all three of the networks are disjoint. This is also verified
% by the number of connected components. More specifically there are \prisonerscon,
% \auctioncon and \pricecon for graphs \(G_1, G_2\) and \(G_3\) respectively.
% % Though the number of connected components for \(G_1\) is between \(G_2\) and \(G_3\)
% % we believe that this could be due the size of the data sets.
% % The of authors that have written by themselves was also calculated.
% A total of \prisonerisolated, \auctionisolated, \priceisolated authors have
% written by themselves, for graphs \(G_1, G_2\) and \(G_3\) respectively.

% The normalised degree distributions of all three networks are shown in Figure~\ref{fig:degrees_dist}.
% They have been normalised such that the frequencies sum to one. 
% None of the distributions is normally distributed thus the non parametric test
% Kruskal-Wallis is used~\cite{mckight2010}. Kruskal-Wallis allow us to compare the
% medians of two or more distributions. The test returns a \(p\)-value of 0.29.
% Thus there is significant difference at the level of 0.005.

% \begin{figure}[!hbtp]
%     \centering
%     \includegraphics[width=\textwidth]{./assets/images/degrees_histrograms.pdf}
%     \caption{Degree distributions for all three networks.}\label{fig:degrees_dist}
% \end{figure}

% The global clustering coefficients of all three networks are presented in Table
% \ref{table:clustering}. The price of anarchy has the largest clustering coefficient
% followed by \(G_1\) and \(G_2\).


% \begin{table}[!hbtp]
%     \begin{center}
%     \begin{tabular}{lcc}
%         \toprule
%                   & \textbf{\(\bar{C}\)}\\
%         \midrule 
%         \(G_1\) & \prisonerscc\\
%         \(G_2\) & \auctioncc\\
%         \(G_3\) & \pricecc\\
%         \bottomrule
%     \end{tabular}
%     \end{center}
%     \caption{Global clustering coefficient for all three networks.}
%     \label{table:clustering}
% \end{table}

% % \paragraph{Temporal comparison}
% % \mbox{ }\\

% % The collaborative behaviour of \(G_1\) is also studied over time. This is achieved
% % by calculating the number of connected components, the degrees distribution
% % and the clustering coefficients for each time period. Table~\ref{table:cc_over_time}
% % summarises the results.

% % The number of connected components indicate that after period 3 to 7 the 
% % number of author is increasing. Period 2 is a very poor period of publication
% % in our sources. For periods 3 to 7 it is shown in Figure~\ref{fig:dist_over_time}
% % that the degrees are stabilising over 3.Thus in the study of the prisoners dilemma
% % according to our data, papers with 3 authors seems to be favoured.

% % Furthermore, the clustering coefficients are also increasing over these periods.
% % This analysis indicates that not only more people are were attracted to the
% % the topic over the years but also that the field was becoming more collaborative.

% % \begin{table}[!hbtp]
% %     \begin{center}
% %     \begin{tabular}{lcc}
% %         \toprule
% %                   & \textbf{Connected components} & \textbf{Clustering coefficient}\\
% %         \midrule
% %         period 1  & 15                            & 0.5 \\
% %         period 2  & 5                             & 0.0 \\
% %         period 3  & 49                            & 0.14 \\
% %         period 4  & 96                            & 0.3 \\
% %         period 5  & 534                           & 0.64 \\
% %         period 6  & 281                           & 0.74 \\
% %         period 7  & 134                           & 0.76 \\
% %         \bottomrule
% %     \end{tabular}
% %     \end{center}
% %     \caption{Collaborative behaviour measures over time periods.}
% %     \label{table:cc_over_time}
% % \end{table}

% % \begin{figure}[!hbtp]
% %     \centering
% %     \includegraphics[width=0.8\textwidth]{./assets/images/degrees_histrograms_temporal.pdf}
% %     \caption{Degrees distribution over time.}\label{fig:dist_over_time}
% % \end{figure}

% \subsubsection{Measures of Influence}

% Network centrality is used in network theory to study which nodes of a graph are
% the most important. There are several centrality measures used to explain different
% behaviours of the nodes. Centrality will be used here to explain influence. 
% The two centrality which are used are:

% \begin{itemize}
%     \item Closeness centrality \(C_C\).
%     \item Betweenness centrality \(C_B\).
% \end{itemize}

% Both network measures are explained with an example. The definitions for both
% centralities are given by Definition~\ref{def:closeness} and~\ref{def:betweenness}.

% \begin{definition}{Closeness.}\label{def:closeness}
%     Closenesse centrality of a node \(u\) is the reciprocal of the average
%     shortest path distance to \(u\) over all \(n-1\) reachable nodes. It is denoted as,
    
%     \[C_C(u)= \frac{n - 1}{\displaystyle \Sigma_{v=1}^{n-1}d(v, u)}\]
    
%     where \(d(v, u)\) is the shortest-path distance between \(v\) and \(u\), and \(n\)
%     is the number of nodes that can reach \(u\). The mornalised centrality
%     is \(C_C\) normalised by the number of nodes in the connected part of the
%     graph.
% \end{definition}

% \begin{definition}{Betweenness.}\label{def:betweenness}
% Betweenness centrality of a node \(u\) is the sum of the fraction of all-pairs of
% shortest paths that pass through \(u\). It is denoted as,

% \[ C_B(u)=\Sigma_{s,t \in V} \frac{\sigma (s,t|u)}{\sigma(s,t)}\]

% where \(V\) is the set of nodes, \(\sigma(s,t)\) is the number of shortest \((s,t)\)-paths,
% and \(\sigma(s,t|v)\) is the number of those paths passing through some node
% \(u\) other than \(s,t\). If \(s=t\), \(\sigma(s,t)=1\), and if \(u \in s,t, \sigma(s,t|u)=0\).
% Normalised \(C_B\) is normalized by \(\frac{2}{((n-1)(n-2))}\)
% \end{definition}

% Closeness is a measure that shows how well a node connects other nodes. Equivalently,
% how well an author is connected to other authors and contributes to them collaborating. 
% On the other hand betweenness is about how connected a
% node is, thus how much influence an author can gain from their environment.

% As an example consider a sub graph of \(G_1\) which is illustrated in
% Figure~\ref{fig:subgraph_t}. Note that nodes 1, 2 and 3 are connected to three
% authors. Thus we expect their betweenness centrality to be the same. However, this
% is not true for closeness centrality. Node 3 is the connecting link between at least
% 4 people. Thus node 3 is the person in the sub graph that influences most authors.
% Node 3 also gains influence due ot its rule in the team, but node 2 achieves the
% same without connecting people as much as node 3.

% \begin{figure}[!hbtp]
%     \centering
%     \includegraphics[width=0.8\textwidth]{./assets/images/centrality_example.pdf}
%     \caption{A sub graph of \(G_1\).}\label{fig:subgraph_t}
% \end{figure}

% The centrality for all three networks are calculated using~\cite{networkx}. Table
% \ref{table:central_authors_pd} summarises the most important authors of network
% \(G_1\) based on the two centralities. The people that influence the field the
% most are Matjaz Perc, Yamir Moreno, Luo-Luo Jiang, Arne Traulsen and Martin A.
% Nowak. Their work have been discussed in Section~\ref{section:timeline}.
% %Examples of their works.
% Though Matjaz Perc and Yamir Moreno appear to both influence and gain from
% the networks influence, it does not hold for the rest of the three authors.

% \begin{table}[!hbtp]
%     \begin{center}
%     \scalebox{0.8}{
%     \begin{tabular}{llr}
\toprule
{} &      Author name &  Betweeness \\
\midrule
0 &      Matjaz Perc &    0.010584 \\
1 &     Yamir Moreno &    0.008786 \\
2 &    Luo-Luo Jiang &    0.004319 \\
3 &    Arne Traulsen &    0.003920 \\
4 &  Martin A. Nowak &    0.003832 \\
\bottomrule
\end{tabular}

%     \begin{tabular}{llr}
\toprule
\(R_2\) &            Author name &  Closeness \\
\midrule
0 &            Matjaz Perc &   0.044360 \\
1 &        Attila Szolnoki &   0.043494 \\
2 &         Daniel Ashlock &   0.038851 \\
3 &          Angel Sánchez &   0.037902 \\
4 &              Long Wang &   0.037542 \\
5 &           Yamir Moreno &   0.037385 \\
6 &              Zhen Wang &   0.036959 \\
7 &           Gyorgy Szabo &   0.036768 \\
8 &  Krishnendu Chatterjee &   0.036245 \\
9 &        Martin A. Nowak &   0.035881 \\
\bottomrule
\end{tabular}
}
%     \caption{Top 5 ranked authors of \(G_1\) based on different centrality measures.}
%     \label{table:central_authors_pd}
%     \end{center}
% \end{table}

% \paragraph{Analysis}
% \mbox{ }\\

% Influence for a given network is calculated easily. However, we want to assert
% the power of the influence of \(G_1\) by comparing the results to those of
% \(G_2\) and \(G_3\). Using the distribution of the centralities will we
% statistically tests whether a difference does exist between them.

% For \(C_C\) all distributions are not normally distributed so we will use the
% non parametric test Kruskal Wallis. The test returns a \(p\) value of 0.0 thus 
% it can be stated with 95\% confidence of that the distributions are statistically
% different. These are plotted as violin plots in Figure~\ref{fig:closeness_dist}.

% Note that the centrality values range between 0 and 1. For all three graphs \(C_C\)
% have low values. Both \(G_1\) and \(G_2\) appear to have a similar distributions.
% Most authors have have a coefficient of zero and a few authors by the tails
% appear to have a larger coefficient. This means that for the two topics, the highest
% frequency of authors have a very small influence in their field. Though there 
% are people with a high influence they are only a few.

% The coefficients of \(G_3\) are different from the other topics. Overall it
% can be seen that authors' centrality is more spread through the different values.

% \begin{figure}[!hbtp]
%     \centering
%     \includegraphics[width=.8\textwidth]{./assets/images/Closeness_histrograms.pdf}
%     \caption{Closeness distributions for all three networks.}\label{fig:closeness_dist}
% \end{figure}

% Figure~\ref{fig:closeness_dist} shows that all coefficients are clustered around the value of zero.

% \begin{center}
% \begin{figure}[!hbtp]
%     \centering
%     \includegraphics[width=.8\textwidth]{./assets/images/Betweenness_histrograms.pdf}
%     \caption{Betweenness distributions for all three networks.}\label{fig:closeness_dist}
% \end{figure}
% \end{center}

% % \paragraph{Temporal comparison}
% % \mbox{ }\\

% % The two centralities are also studied over time. Table~\ref{table:cc_over_time} higlights
% % the authors with the most influnce at each time period. Equivalently, Table~\ref{table:bc_over_time}
% % highlights the authors that gained more from the networks' influence at each
% % time period.

% % For periods 1, 3, 4, 5 and 6 the authors with the highest closeness centrality
% % are the authors with the highest betweenness centrality as well. Several of these
% % authors have been shown to also have a strong influence to the entire network
% % as well.

% % For period 7 though March Harper is the person that influences most authors,
% % Chen Yi Xia is the person that gains the most. For period 2 we have a similar
% % case. Robert Axelrod appears to be the most influenced author but not gaining 
% % as much from it.

% % Note that the centrality values are decreasing over the time periods. This could
% % be an effect due the size of the network which is increasing. In smaller networks
% % authors have more motivation to collaborate with the few people on their fields.

% % \begin{table}[!hbtp]
% %     \begin{center}
% %     \scalebox{0.8}{
% %     \begin{tabular}{llr}
\toprule
{} &      Author name &  Closeness \\
\midrule
1 &  Svenn Lindskold &   0.108108 \\
2 &       R. Axelrod &   0.200000 \\
3 &  Martin A. Nowak &   0.043478 \\
4 &  Lee A. Dugatkin &   0.029762 \\
5 &      Matjaz Perc &   0.043443 \\
6 &     Yamir Moreno &   0.039311 \\
7 &      Marc Harper &   0.049462 \\
\bottomrule
\end{tabular}
}
% %     \caption{Authors with the most influnce at each time period.}
% %     \label{table:cc_over_time}
% %     \end{center}
% % \end{table}

% % \begin{table}[!hbtp]
% %     \begin{center}
% %     \scalebox{0.8}{
% %     \begin{tabular}{llr}
\toprule
{} &      Author name &  Betweeness \\
\midrule
1 &  Svenn Lindskold &    0.006006 \\
2 &     A. W. Tucker &    0.000000 \\
3 &  Martin A. Nowak &    0.001279 \\
4 &  Lee A. Dugatkin &    0.000499 \\
5 &      Matjaz Perc &    0.010689 \\
6 &     Yamir Moreno &    0.005468 \\
7 &     Cheng-Yi Xia &    0.001242 \\
\bottomrule
\end{tabular}
}
% %     \caption{Authors that gained more from the networks influence at each
% %     time period.}
% %     \label{table:bc_over_time}
% %     \end{center}
% % \end{table}

% \subsubsection{Conclusion}

% In this section we have conducted an investigation of the literature based on a
% data analysis. More specifically, this was mainly done using network theory.

% Initially, we gave a summary on the data collection. An open source project which was
% developed for the purpose of this work was used~\cite{nikoleta_2017}. The project
% takes advantage of the API system several academic journals offer today. The
% procedure, the sources as well as the keywords used in the process of collection
% have been clearly specified making the process reproducible.

% Three data sets have been composed for three different topics of game theory.
% These are:

% \begin{itemize}
%     \item The prisoner's dilemma. The main focus of this paper.
%     \item Auction games. A sub field of game theory used for comparison reasons.
%     \item The price of anarchy. A sub field of game theory used for comparison reasons.
% \end{itemize}

% We conducted a brief preliminary analysis on these data sets. Mainly to understand
% the sizes, provenance and trends of each topic. The main data set was also partitioned
% into time periods such that an temporal comparison could be conducted.

% The main focus of the analysis has been to explain collaborative behaviour and
% influence. Both terms have been defined and we have explained how network measures
% of the co authorship network were used to quantify them. The co authorship
% network is a network representing all the unique authors of a topic. An edge
% exists within two authors if they have written together. Co authorship was
% decided to be used as be believe other measures, such as citations, perform less
% well.

% % The findings of this analysis are presented in two parts. The comparison of the
% % collaborative behaviour and influence of the prisoner's dilemma field based on
% % other topics and how these measures change for the field over time.

% All three networks have been disjointed with a large number of connected components.
% The collaborative behaviour was based on the nature of these connected components.
% The median connection of an author has been the same for all three networks.
% However, the price of anarchy had a smaller number of authors that prefer to write
% on their own and based on the clustering coefficient the collaboration of the
% field authors appears to be stronger. The collaboration for both the prisoner's
% dilemma and auction games it's similar. Note though that the price of anarchy
% is a new topic with less authors. It makes more sense for a few people that
% work on the topic to be more collaborative which each other.

% Similarly influence was studied using two centrality measures. For \(G_1\) and
% \(G_2\) we conclude that there are only a few authors that have power of influence
% on the network. For research this is not ideal. It could mean that the research
% is only driven from the work of specific people. It could also indicate a hostile
% environment for new authors. In comparison, \(G_3\) has several authors that have
% different influence on their neighbours. A wider spread of influence could indicate
% a nice flow of knowledge across the field from different people. This could
% help the growth of a field and accelerate findings. The influence that authors
% gain from the respective networks was also explored. The results argued that
% the gain was very low for all three networks.

% Collaborations and influence of the field have changed over time. As argued
% by~\cite{Etzkowitz1992} over time the amount of research groups was increasing
% for the topic. However, as stated earlier, as the number of researchers increases
% the amount of influence and collaboration decreases. This is because researchers
% can not have full knowledge of their entire network and the more people the network
% has the more people we actually do not know. The results from the temporal
% analysis imply that this is true for the field of the prisoner's dilemma.


\newpage
\bibliographystyle{plain}
\bibliography{bibliography.bib}
\end{document}
