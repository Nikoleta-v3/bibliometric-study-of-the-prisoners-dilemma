\documentclass{article}
% Package to manage page layout
\usepackage[margin=2cm, includefoot, footskip=30pt]{geometry}

\setlength\parindent{0pt}
\setlength{\parskip}{1em}

%%%%%%%PACKAGES HERE%%%%%%%
\usepackage{amsmath}
\usepackage{amsthm}
\usepackage{amssymb}
\usepackage{hyperref}
\usepackage{standalone}
\usepackage{subcaption}
\usepackage{adjustbox}
\usepackage{tikz}
\usepackage{booktabs}
\usepackage{multicol,multirow,array}
\usepackage{graphicx}
\usepackage{algorithm,algorithmic}
\usepackage{authblk}
\usepackage{tabularx}
\usepackage{color, colortbl}
\usetikzlibrary{er,positioning, calc, patterns}
\usetikzlibrary{decorations.pathreplacing}

\definecolor{Gray}{gray}{0.92}
\definecolor{background}{RGB}{5, 66, 81}

\setcounter{secnumdepth}{4}
\setcounter{tocdepth}{4}

\theoremstyle{definition}
\newtheorem{definition}{Definition}[section]

%%%%%%%%%%%%%%%%%%%%%%%%%%%%%%%PARAMETERS%%%%%%%%%%%%%%%%%%%%%%%%%%%%%%%%%%%%%%%
\newcommand{\totalarticles}{\input{assets/total_articles.txt}}
\newcommand{\manual}{\input{assets/prov_manual.txt}}
\newcommand{\authors}{\input{assets/number_of_authors.txt}}
\newcommand{\edges}{\input{assets/num_Edges.txt}}
\newcommand{\isolated}{\input{assets/num_Isolated_nodes.txt}}
\newcommand{\isolatedpercentage}{\input{assets/perce_Isolated_nodes.txt}}
\newcommand{\connectedcomponents}{\input{assets/num_Connected_components.txt}}
\newcommand{\communities}{\input{assets/num_Communities.txt}}
\newcommand{\largestcc}{\input{assets/Size_of_largest_component.txt}}
\newcommand{\clustering}{\input{assets/Clustering_coeff.txt}}
%%%%%%%%%%%%%%%%%%%%%%%%%%%%%%%%%%%%%%%%%%%%%%%%%%%%%%%%%%%%%%%%%%%%%%%%%%%%%%%%
%%%%%%%%%%%%%%%%%%%%%%%%%%%%%%%%%%%%%%%%%%%%%%%%%%%%%%%%%%%%%%%%%%%%%%%%%%%%%%%%
\title{A bibliometric study of collaboration and influence in the field of
the Iterated Prisoner's Dilemma}

\author[1]{Nikoleta E. Glynatsi}
\author[1]{Vincent A. Knight}

\affil[1]{Cardiff University, School of Mathematics, Cardiff, United Kingdom}
\date{}
\setcounter{Maxaffil}{0}
\renewcommand\Affilfont{\itshape\small}
\begin{document}

\maketitle

\begin{abstract}
This manuscript explores the collaborative behaviour of authors in the field of
the Prisoner's Dilemma using bibliometric analysis, topic modeling, and a graph
theoretic analysis of the co-authorship network. The results demonstrate that
the Prisoner's Dilemma is a field of continued interest and a collaborative
field. Authors are very like to cooperate with a collaborator's co-author and
single author publications are rare. The co-authorship network however suggests
that authors are collaborative with authors from their communities, and now many
connections across the communities exist, moreover, authors do not influence or
gain much information by their connections, unless they are connected to a
``main'' group of authors. The topic modeling approach identified five different
research topics within the field, these were human subject research, biological
studies, strategies, evolutionary dynamics on networks and modeling problems
as a Prisoner's Dilemma game.
The articles
collected for this work are classified based on the language of their
abstracts in five different topics, and the results are additionally verified by
the topics' networks.
\end{abstract}

\section{Introduction}\label{section:introduction}

The Prisoner's Dilemma (PD) is a well known game used since its introduction in
the 1950's~\cite{Flood1958} as a framework for studying the emergence of
cooperation; a topic of continued interest for mathematical,
social~\cite{Perc2008}, biological~\cite{Turner1999} and
ecological~\cite{Wu2011} sciences. This manuscript presents a bibliometric
analysis of 2,420 published articles on the Prisoner's Dilemma between 1951 and
2018. It presents the dominant topics in the PD publications, which have been
identified using Latent Dirichlet Allocation, and it explores the changes in the
dominant topics over time. The collaborative behaviour of the field is explored
using the co-authorship network, and furthermore, the Latent Dirichlet
Allocation topics are combined with the co-authorship network analysis to assess
the relative influence of authors in these topics. Assessing the collaborative
behaviour of the field of collaboration itself is the main aim of this work.

As discussed in~\cite{youngblood2018}, bibliometrics (the statistical analysis
of published works originally described by~\cite{pritchard1969}) has been used
to support historical assumptions about the development of fields
\cite{raina1998}, identify connections between scientific growth and policy
changes \cite{das2016}, develop a quantitative understanding of author
order~\cite{sekara2018}, and investigate the collaborative structure of an
interdisciplinary field~\cite{Liu2015}. Most academic research is undertaken in
the form of collaborative effort and as~\cite{Kyvik2017} points out, it is
rational that two or more people have the potential to do better as a group
than individually. Collaboration in groups has a long tradition in experimental
sciences and it has be proven to be productive according
to~\cite{Etzkowitz1992}. The number of collaborations can be very different
between research fields and understanding how collaborative a field is not
always an easy task. Several studies tend to consider academic citations as a
measure for these things. A blog post published by Nature~\cite{nature_blog}
argues that depending on citations can often be misleading because the true
number of citations can not be known. Citations can be missed due to data entry
errors, academics are influenced by many more papers than they actually cite and
several of the citations are superficial.

A more recent approach to measuring collaborative behaviour, and to studying the
development of a field is to use the co-authorship network, as described
in~\cite{Liu2015}. The co-authorship network has many advantages as several
graph theoretic measures can be used as proxies to explain author relationships.
In~\cite{Liu2015}, the approach was applied to analyse the development of the field
``evolution of cooperation'', and in~\cite{youngblood2018} to identify the
subdisciplines of the interdisciplinary field of ``cultural evolution'' and
investigate trends in collaboration and productivity between these subdisciplines.
Moreover, \cite{Li2019} examined the
long-term impact of co-authorship with established, highly-cited scientists on
the careers of junior researchers. This paper builds upon the work done
by~\cite{Liu2015} and ~\cite{youngblood2018}, and extends their methodology.
Though in \cite{Liu2015, youngblood2018}, they considered a data set from a
single source, Web of Science, the data set described here, archived
at~\cite{pd_data_2018}, has been collected from five sources.

Latent Dirichlet Allocation (LDA) is a topic modeling technique proposed
in~\cite{Blei2003} as a generative probabilistic model for discovering
underlying topics in collections of data.
Applications of the technique include detection in image data~\cite{Agarwal2008,
Coelho2010} and detection in video~\cite{Niebles2008, Wang2008}. Nevertheless,
LDA has been applied by several works on publication data for identifying the
topic structure of a subject area. In~\cite{Inglis2018}, it was applied to the
publications on mathematical education of the journals ``Educational Studies in
Mathematics'' and ``Journal for Research in Mathematics Education'' to
identify the dominant topics that each journal was publishing on. The topics of
the North American library and Information Science dissertations were 
studied chronologically~\cite{Sugimoto2011}, and LDA was expanded in~\cite{Blei2007}
to include the correlation between topics. Another extension to the approach
is to combine LDA with a co-authorship network analysis~\cite{Bergmann2018}.
In~\cite{Bergmann2018} the scientific content presented at EvoLang conferences
were initially quantified using LDA and the clustered, using a clustering
algorithm, to reveal different domains. The work of~\cite{Bergmann2018} only
considered the main network and no analysis was done in the individual sub graphs
which corresponded to the clusters. Compared to that this manuscript applies
LDA to identify topics and analysis the networks corresponding to each topic
as well.

The methodology
used in this manuscript, which includes the data collection, is covered in
Section~\ref{section:methodology}. The results regarding the amount of
publications and the topics of the field are described in
Section~\ref{section:results}. In Section~\ref{section:co_authorship}, graph
theoretical methods are used to ascertain the level of collaborative nature of
the field and identify influence. Finally, the results are summarised in
Section~\ref{section:conclusion}.

\section{Methodology}\label{section:methodology}

Academic articles are accessible through scholarly databases. Several databases
and collections today offer access through an open application protocol
interface (API). An API allows users to query directly a journal's database and
bypass the graphical user interface. Interacting with an API has two phases:
requesting and receiving. The request phase includes composing a url with the
details of the request. For example,
\url{http://export.arxiv.org/api/query?search_query=abs:prisoner's
dilemma&max_results=1} represents a request message. The first part of the
request is the address of the API. In this example the address corresponds to
the API of arXiv. The second part of the request contains the search arguments.
In this example it is requested that the word `prisoners dilemma' exists within
the article's title. The format of the request message is different from API to
API. The receive phase includes receiving a number of raw metadata of articles
that satisfies the request message. The raw metadata are commonly received in
extensive markup language (xml) or Javascript object notation (json)
formats~\cite{nurseitov2009}. Similarly to the request message, the structure of
the received data differs from journal
to journal.

The data collection is crucial to this study. To ensure that this study can be
reproduced all code used to query the different APIs has been packaged as a
Python library and is available online~\cite{nikoleta_2017}. The software could
be used for any type of projects similar to the one described here,
documentation for it is available at:
\url{http://arcas.readthedocs.io/en/latest/}. Project~\cite{nikoleta_2017} allow
users to collect articles from a list of APIs by specifying just a single
keyword. Articles for which any of the terms ``prisoner's dilemma'',
``prisoners dilemma'', ``prisoner dilemma'', ``prisoners evolution'', ``prisoner
game theory'' existed within the title, the abstract or the text are included
in the analysis. Four prominent journals in the field and a preprint server
were used as sources to collect data for this analysis:

\begin{multicols}{2}
    \begin{itemize}
        \item arXiv~\cite{mckiernan2000}; a repository of electronic preprints.
        It consists of scientific
        papers in the fields of mathematics, physics, astronomy, electrical engineering,
        computer science, quantitative biology, statistics, and quantitative finance,
        which all can be accessed online.
        \item PLOS~\cite{plos}; a library of open access journals and other scientific literature
        under an open content license. It launched its first journal, PLOS Biology,
        in October 2003 and publishes seven journals, as of October 2015.
        \item IEEE Xplore Digital Library (IEEE)~\cite{ieee}; a research database for discovery
        and access to journal articles, conference proceedings, technical standards,
        and related materials on computer science, electrical engineering and electronics,
        and allied fields. It contains material published mainly by the Institute of
        Electrical and Electronics Engineers and other partner publishers. 
        \item Nature~\cite{nature}; a British multidisciplinary scientific journal,
        first published on 4 November 1869. It was ranked the world's most cited
        scientific journal by the Science Edition of the 2010 Journal Citation Reports
        and is ascribed an impact factor of 40.137, making it one of the world's
        top academic journals.
        \item Springer~\cite{springer}; a leading global scientific publisher of
        books and journals. It publishes close to 500 academic and professional
        society journals.
    \end{itemize}
\end{multicols}

The data set has been archived and available at~\cite{pd_data_2018}. Note that the latest data
collection was performed on the \(30^{\text{th}}\) November 2018.
% TODO Ensure this stay accurate

The relationship between the authors within a field will be modeled as a graph
\(G = (V_G, E_G)\) where \(V_G\) is the set of nodes and \(E_G\)  is the set of
edges. The set \(V_G\) represents the authors and an edge connects two authors
if and only if those authors have written together. The co-authorship network is
constructed using the main data set~\cite{pd_data_2018} and the open source package
\cite{networkx}. The PD network is denoted as \(G\) where the
number of unique authors \(|V(G)|\) is \authors and \(|E(G)|\) is \edges.
All authors' names were formatted as their last name and first name (i.e.
Martin A. Nowak to Martin Nowak). This was done to avoid errors such as Martin
A. Nowak and Martin Nowak being treated as a different person. There are some
authors for which only their first initial was found. These entries are left as
such.

The collaborativeness of the authors will be analysed using measures such as, isolated nodes,
connected components, clustering coefficient, communities, modularity and average degree.
These measures show the number of connections authors can have
and how strongly connected these people are. The number of isolated nodes is the
number of nodes that are not connected to another node, thus the
number of authors that have published alone. The average degree denotes the average
number of neighbours for each nodes, i.e. the average number of collaborations
between the authors.
A connected component is a maximal set of nodes such that each pair of nodes is
connected by a path~\cite{Easley2010}. The number of connected components as well as the size of the
largest connected component in the network are reported.
The size of the largest connected component represents the scale of the central cluster
of the entire network, as will discussed in the analysis section.
Clustering coefficient and modularity are also calculated. The clustering
coefficient, defined as 3 times the number of triangles on the graph divided
by the number of connected triples of nodes, is a local measure of the degree to
which nodes in a graph tend to cluster together
in a clique~\cite{Easley2010}. It is precisely the probability that the collaborators
of an author also write together.
In comparison, modularity is a global measure designed to measure the strength of
division of a network into communities. The number of communities will be reported
using the Clauset-Newman-Moore method~\cite{clauset2004}. Also the modularity index
is calculated using the Louvain method described in~\cite{Blondel2008}. The value
of the modularity index can vary between \([-1, 1]\), a high value of modularity
corresponds to a structure where there are dense connections between the nodes within
communities but sparse connections between nodes in different communities.
That means that there are many sub communities of authors that write together
but not across communities.
Two further points are aimed to be explored in this work, (1) which people control the flow
of information;
as in which people influence the field the most and (2) which are the authors that
gain the most from the influence of the field. To measure these concepts
centrality measures are going to be used.
Centrality measures are often used to understand different
aspects of social networks~\cite{Landherr2010}. Two centrality measures have been
chosen for this paper and these are closeness and betweenness centrality.

\begin{enumerate}
    \item In networks some nodes have a short distance to a lot of nodes and
    consequently are able to spread information on the network very effectively.
    A representative of this idea is \textbf{closeness centrality}, where a node
    is seen as centrally involved in the network if it requires only few
    intermediaries to contact others and thus is structurally relatively
    independent. Here, this is interpreted as a influence. Authors with a high
    value of closeness centrality, are the authors that spread scientific
    knowledge easier on the network and they have high influence.
    \item Another centrality measure is the \textbf{betweenness centrality},
    where the determination of an author's centrality is based on the quotient
    of the number of all shortest paths between nodes in the network that
    include the node in question and the number of all shortest paths in the
    network. In betweenness centrality the position of the node matters. Nodes
    with a higher value of betweenness centrality are located in positions that
    a lot of information pass through, this is interpreted as the gain from
    the influence, thus these authors gain the most from their networks.
\end{enumerate}

The articles contained in~\cite{pd_data_2018} will be classified
into research topics using LDA an unsupervised machine learning technique
designed to summarize large collections of documents by a small number of
conceptually connected topics or themes~\cite{Blei2003, Grimmer2013}. The
documents are the articles' abstracts and LDA was carried out using~\cite{rehurek_lrec}.
In LDA, each document/abstract is represented by a distribution over topics, 
and the topics themselves are represented by a distribution over words. A
document's dominant topic is based on which topic has the highest percentage
contribution. For example let the words that contribute to a topic be:

\begin{itemize}
    \item Topic A: \(0.039 \times\)``cooperation'', \(0.028 \times\)``study'' and \(0.026 \times\)``human''.
    \item Topic B: \(0.020 \times\)``cooperation'', \(0.028 \times\)``agents'' and
    \(0.026 \times\)``strategies''.
\end{itemize}

A document with abstract ``The study of cooperation in humans'' has a percentage
contribution \(0.039 + 0.028 + 0.026 = 0.093\) for Topic A and 
\(0.020 + 0.0 + 0.0 = 0.020\) for Topic B. For the given document the dominant
topic is Topic A.
LAD requires that the number of topics is specified in advance before running
the algorithm. The number of topics can be chosen using the coherence
value~\cite{Roder2015} or through subjective minimisation of the overlapping
keywords between two topics. Both these approaches will be used in this work.

Though several of the approaches described in this section have previously been
carried out~\cite{Bergmann2018}, ~\cite{Liu2015},~\cite{Sugimoto2011}
and~\cite{youngblood2018}, the novelty here is combining the approaches and applying them
to a new data set. The manuscript presents the identified topics in the field of the PD,
and studies them over time. Moreover, the collaborativeness of the authors is
explored using the co-authorship network, for the network as a whole and
individually for each topic. Similarly, the influence of the authors is explored
for the field as a whole and for individual topics. The results are presented in
the following sections.

\section{Preliminary Analysis}\label{section:results}

The data set~\cite{pd_data_2018} consists of \totalarticles articles with unique
titles. In case of duplicates the preprint version of an article (collected from
arXiv) was dropped. Similarly to~\cite{Liu2015}, \manual articles have not been
collected from the aforementioned APIs but have been manually added because they
are of interest to the authors. Examples of such papers include~\cite{Flood1958}
the first publication on the PD,~\cite{Ohtsuki2006, Stewart2012} two well cited
articles in the field, and a series of works from Robert Axelrod
~\cite{Axelrod1980, Axelrod1980more, Axelrod1987, Axelrod1981, Riolo2001} a
leading author of the field. A more detailed summary of the articles' provenance
is given by Table~\ref{table:preliminary_table}. Only 3\% of the data set consists of
articles that were manually added and 27\% of the articles were collected from
arXiv. The average number of publications is also included in
Table~\ref{table:preliminary_table}. Overall an average of 43 articles are published
per year on the topic. The most significant contribution to this appears to be
from arXiv with 11 articles per year, followed by Springer with 9 and PLOS with
8.

\begin{table}[!hbtp]
    \begin{center}
    \resizebox{.8\textwidth}{!}{
    \begin{tabular}{lrrrr}
\toprule
{} &  Number of Articles &  Percentage \% &  Year of first publication &  Average number of publications per year\\
\midrule
IEEE     &               294 &       12.14\% &                    1973 &                             5\\
Manual   &                76 &        3.14\% &                    1951 &                             1\\
Nature   &               436 &       18.00\% &                    1959 &                             8\\
PLOS     &               477 &       19.69\% &                    2005 &                             8\\
Springer &               533 &       22.01\% &                    1966 &                             9\\
arXiv    &               654 &       27.00\% &                    1993 &                            11\\
Overall  &              2470 &      100.00\% &                    1951 &                            43\\
\bottomrule
\end{tabular}
}
    \end{center}
    \caption{Articles' provenance for the main data set~\cite{pd_data_2018}.}
    \label{table:preliminary_table}
\end{table}

The data handled  here is in fact a time series starting from the 1950s and the formulation
of the game, until 2018 (Figure~\ref{fig:timeseries}). Two observations can be
made from Figure~\ref{fig:timeseries}.

\begin{enumerate}
    \item There is a steady increase to the number of publications since the
    1980s and the introduction of computer tournaments~\cite{Axelrod1981}
    (a work by Robert Axelrod).
    \item There is a decrease in 2017-2018. This is due to our data set being
    incomplete. Articles that have been written in 2017-2018 have either not
    being published or were not retrievable by the APIs at the time of the last
    data collection.
\end{enumerate}

These observations can be confirmed by studying the time series.
Using~\cite{scipy}, an exponential distribution is fitted to the data
(Figure~\ref{fig:fitting}). The fitted model can be used to forecast the
behaviour of the field for the next 5 years (Figure~\ref{fig:forecasting}). Even
though the time series has indicated a slight decrease, the model forecasts that
the number of publications will keep increasing, thus demonstrating that the
field of the PD continues to attract academic attention.

\begin{figure}[!hbtp]
\begin{minipage}{.45\textwidth}
    \centering
    \includegraphics[width=.85\textwidth]{./assets/images/timeline.pdf}
    \caption{Line plot; number of articles published on the PD 1951-2019 (on a log scale).}\label{fig:timeseries}
\end{minipage}
\begin{minipage}{.45\textwidth}
    \centering
    \includegraphics[width=.85\textwidth]{./assets/images/fitting.pdf}
    \caption{Scatter plot; number of articles published on the PD 1951-2019 (on a log scale).}\label{fig:fitting}
\end{minipage}
\end{figure}

\begin{figure}[!hbtp]
    \centering
    \includegraphics[width=.40\textwidth]{./assets/images/forecasting.pdf}
    \caption{Forecast for 2017-2022 (on a log scale).}\label{fig:forecasting}
\end{figure}

The overall collaboration index (CI) or the average number of authors on
multi-authored papers is 3.2, thus on average a non single author publication in
the PD has 3 authors. This appears to quite standard compared to other fields
such as cultural evolution~\cite{youngblood2018}, Astronomy and Astrophysics,
Genetics and Heredity, Nuclear and Particle Physics as reported
by~\cite{nature_author_blog}. This indicates the PD a collaborative field, but
perhaps not more collaborative than other fields are.

There are only a total of 545 publications with a single author, which
corresponds to the 22\% of the papers. It appears that academic publications
tend to be undertaken in the form of collaborative effort, which is in line
with the claim of~\cite{Kyvik2017}. From
Figure~\ref{fig:ci_over_time} the trend of CI over the years is given. There are
some peaks in the early years 1969 and 1980, however, a steady increase appears
to happen after 2004. This could be an effect of better communication tools
being introduced around that time which enabled more collaborations between
researchers. The collaborative behaviour will be explored again in the
following section using the co-authorship network.

\begin{figure}[!hbtp]
    \centering
    \includegraphics[width=.55\textwidth]{./assets/images/collaborative_index.pdf}
    \caption{Collaboration index over time.}\label{fig:ci_over_time}
\end{figure}

The authors in~\cite{pd_data_2018} have had multiple publications collected from
the data collection process. The highest number of articles collected for an
author is 83 publications for Matjaz Perc. The distribution of the number of
papers per author is given by Figure~\ref{fig:num_papers_per_author}, and it can
be seen that Matjaz Perc is an outlier. More specifically, most authors have
1 to 6 publications in the data set.

\begin{figure}[!hbtp]
    \centering
    \includegraphics[width=.50\textwidth]{./assets/images/papers_per_author.pdf}
    \caption{Distribution of number of papers per author (on a log scale).}
    \label{fig:num_papers_per_author}
\end{figure}

\textbf{What are the topics of the PD?}

In order to identify the topics which are being discussed in the field of
the PD, a LDA algorithm implemented in~\cite{rehurek_lrec} is applied to the
abstract's of the
data set. As mentioned before, the number of topics needs to be specified before
running the algorithm. To chose the appropriate number of topics the coherence
value is used. Figure~\ref{fig:coherence_value_over_number_of_topcis}
gives the coherence values over 18 different models, and it can be seen than
the most appropriate number of topics is 6 with a coherence value
of 0.418.

\begin{figure}[!hbtp]
    \centering
    \includegraphics[width=.5\textwidth]{./assets/images/coherence_values.pdf}
    \caption{Coherence for LDA models over the number of topics.}
    \label{fig:coherence_value_over_number_of_topcis}
\end{figure}

A visualisation of the LDA model with 6 number of topics is given in Figure
\ref{fig:lda_visualisation}, this has been generated using~\cite{Sievert2014}
an open source designed for visualising LDA models. Each bubble of Figure
\ref{fig:lda_visualisation} represents a topic. The bubbles are labelled 1 to
6 and the size of a bubble corresponds to the prevalence of the topic.
From the visualisation it can be seen that two topics overlap (the keywords
of these topics overlap). Even though several number of topics with higher
values of coherence exist (apart from 6), a number of 5 topics is chosen
to carry out the analysis of this work. That is because 5 minimises the
number of overlapping topics. Five has a coherence value 0.406 which is close
to 0.418.

\begin{figure}[!hbtp]
    \centering
    \includegraphics[width=.43\textwidth]{./assets/images/topic_clustering.png}
    \caption{Visualisation of LDA model with 6 number of topics.}
    \label{fig:lda_visualisation}
\end{figure}

Based on 5 number of topics the articles have been clustered and assigned
to the topic with the highest percentage contribution. The keywords associated
with a topic, the most representative article of the given topic based on the percentage
contribution and its academic reference are given by
Table~\ref{table:topics_and_articles}. The topics are labelled as A, B, C, D and
E, and more specifically:

\begin{itemize}
    \item Based on the keywords associated with Topic A, and the most
    representative article, Topic A appears to be about \textbf{human subject
    research}. Several publications assigned to the topic study the PD by
    setting experiments and having human participants simulated the game,
    instead of computer simulations. These articles include~\cite{Matsumoto2016}
    which showed that prosocial behavior increased with the age of the
    participants, ~\cite{Li2014} which studied the difference in cooperation
    between high-functioning autistic and typically developing
    children,~\cite{Molina2013} explored the gender effect in highschool
    students and~\cite{Bell2017} explored the effect of facial expressions of
    individuals.
    \item Though it is not immediate from the keywords associated with
    Topic B, investigating the papers assigned to the topic indicate that it
    is focused on \textbf{biological studies}. Papers assigned to the topic include
    papers which apply the PD to genetics~\cite{Santorelli2008, Sistrom2015}, to
    the study of tumours~\cite{archetti2013evolutionary, sartakhti2017} and
    viruses~\cite{turner1999prisoner}. Other works include how phenotype affinity
    can affect the emergence of cooperation~\cite{wu2019phenotype} and modeling
    bacterial communities as a spatial structured social dilemma.
    \item Based on the keywords and the most representative article Topic
    C appears to include publications on PD \textbf{strategies}. Publications
    in the topic include the introduction of new strategies~\cite{stewart2013extortion},
    the search of optimality in strategies~\cite{banerjee2007reaching} and the
    training of strategies~\cite{ishibuchi2011evolution} with different
    representation methods. Moreover, publications which studied the evolutionary
    stability of strategies~\cite{adami2013evolutionary} and introduced methods
    of differentiating between them~\cite{ashlock2008fingerprinting} are
    also assigned to the topic.
    \item The keywords associated with Topic D clearly show that the topic
    is focused on \textbf{evolutionary dynamics on networks}. Publications include
    \cite{ichinose2013robustness} which explored the robstuness of cooperation
    on networks,~\cite{wang2012spatial} which studied the effect of a strategy's neighbourhood
    on the emergence of cooperation and~\cite{chen2016fixation} which explored
    the fixation probabilisteis of any two strategies is spatial
    structures.
    \item The publication assigned to Topic E are on \textbf{modeling problems
    as a PD game}. The problems formulated as a PD include decision making in
    operational research~\cite{ormerod2010or}, information sharing among members
    in a virtual team~\cite{feng2008trilateral}, the measurement of influence
    in articles based on citations~\cite{hutchins2016relative} and the price
    spikes in electric power markets~\cite{Guan2002}.
\end{itemize}

\begin{table}[!hbtp]
    \begin{center}
    \resizebox{\textwidth}{!}{
    \begin{tabularx}{1.5\textwidth}{lXXl|cc}
\toprule
Dominant Topic &                                                                                                 Topic Keywords &                                                                                                                                    Most Representative Article Title &        Reference &  \# Documents &  \% Documents \\
\midrule
A &                 social, behavior, human, study, experiment, cooperative, cooperation, suggest, find, behaviour &                                                                                      Facing Aggression: Cues Differ for Female versus Male Faces &  \cite{Geniole2012} &                496.0 &                   0.2008 \\
B &                               individual, group, good, show, high, increase, punishment, cost, result, benefit &  Genomic and Gene-Expression Comparisons among Phage-Resistant Type-IV Pilus Mutants of Pseudomonas syringae pathovar phaseolicola &  \cite{Sistrom2015} &                309.0 &                   0.1251 \\
C &                             game, strategy, player, agent, dilemma, play, payoff, state, prisoner, equilibrium &                                                            Fingerprinting: Visualization and Automatic Analysis of Prisoner's Dilemma Strategies &  \cite{Sistrom2015} &                561.0 &                   0.2271 \\
D &  cooperation, network, population, evolutionary, evolution, interaction, dynamic, structure, cooperator, study &                                                   Influence of initial distributions on robust cooperation in evolutionary  Prisoner's Dilemma &     \cite{Chen2007} &                556.0 &                   0.2251 \\
E &                           model, theory, base, system, problem, paper, propose, information, provide, approach &                                                                          Gaming and price spikes in electric power markets and possible remedies &     \cite{Guan2002} &                548.0 &                   0.2219 \\
\bottomrule
\end{tabularx}
}
    \end{center}
    \caption{Keywords for each topic and the document with the most representative article for each topic.}
    \label{table:topics_and_articles}
\end{table}

Note that the whilst for the choice of 5 topics the actual clustering is not
subjective (the algorithm is determining the output) the interpretation is.

The five topics in the PD publications are human subject research, biological
studies, strategies, evolutionary dynamics on networks and  modeling problems as
a PD. These 5 topics nicely summarise the PD research. They highlight the
interdisciplinarity of the field; how it brings together applied modeling of
real world situations (Topic B and E) and more theoretical notions such as
evolutionary dynamics and optimality of strategies.

\textbf{Is one topic worth to start investigating?}

Figure~\ref{fig:number_of_articles_per_topic} gives the number of articles
per topic over time. The topics appear to have had a similar trend over the years,
with topics B and D having a later start. Following the introduction of a topic
the publications in that topic have been increasing. There is no decreasing
trend in any of the topics. All the topics have been publishing for years and
they still attract the interest of academics. Thus, any of the topics is
worth to start investigating.

\begin{figure}[!hbtp]
    \centering
    \includegraphics[width=\textwidth]{./assets/images/papers_per_topic_over_time.pdf}
    \caption{Number of articles per topic over the years (on a logged scale).}\label{fig:number_of_articles_per_topic}
\end{figure}

\textbf{How do the topics of the PD change over time?}

To gain a better understanding regarding the change in the topics over the years,
LDA is applied to the cumulative data set over 8 time periods. These periods are
1951-1965, 1951-1973, 1951-1980, 1951-1988, 1951-1995, 1951-2003, 1951-2010,
1951-2018. The number of topics for each cumulative subset is chosen based on
the coherence value and no objective approach is used. As a result, the period
1951-2018 has been assigned 6 number of topics which had the highest coherence
value instead of 5. The chosen models for each period including the
number of topics, their keywords and number of articles assigned to them are
given by Table \ref{table:topics_per_year}.

\begin{table}[!hbtp]
    \begin{center}
    \resizebox{\textwidth}{!}{
    \begin{tabular}{llccc}
\toprule
    Period &  Topic &                                                                                                Topic Keywords & Num of Documents & Percentage of Documents \\
\midrule
 1951-1965 &               1 &                 problem, technology, divert, euler, subsystem, requirement, trace, technique, system, untried &                3 &                   0.375 \\
 1951-1965 &               2 &            interpret, requirement, programme, evolution, article, increase, policy, system, trace, technology &                2 &                    0.25 \\
 1951-1965 &               3 &          equipment, agency, conjecture, development, untried, programme, trend, technology, weapon, technique &                1 &                   0.125 \\
 1951-1965 &               4 &                 variation, celebrated, trend, untried, change, involve, month, technique, subsystem, research &                1 &                   0.125 \\
 1951-1965 &               5 &                           give, good, modern, trace, technique, ambiguity, problem, trend, technology, system &                1 &                   0.125 \\
 \midrule
 1951-1973 &               1 &                           study, shock, cooperative, money, part, vary, investigate, good, receive, equipment &               12 &                  0.3243 \\
 1951-1973 &               2 &          cooperation, level, significantly, sequence, reward, provoke, descriptive, principal, display, argue &                4 &                  0.1081 \\
 1951-1973 &               3 &               player, make, effect, triad, experimental, motivation, dominate, hypothesis, instruction, trend &                3 &                  0.0811 \\
 1951-1973 &               4 &                                           ss, sex, male, female, dyad, design, suggest, college, factor, tend &                3 &                  0.0811 \\
 1951-1973 &               5 &               result, research, format, change, operational, analysis, relate, understanding, decision, money &                2 &                  0.0541 \\
 1951-1973 &               6 &                          condition, give, high, treatment, conflict, cc, real, original, replication, promote &                2 &                  0.0541 \\
 1951-1973 &               7 &              group, competitive, show, interpret, scale, compete, escalation, free, variable, individualistic &                2 &                  0.0541 \\
 1951-1973 &               8 &                        outcome, strategy, choice, type, pdg, difference, dummy, conclude, compare, consistent &                2 &                  0.0541 \\
 1951-1973 &               9 &                   game, difference, pair, approach, behavior, person, weapon, occur, advantaged, differential &                2 &                  0.0541 \\
 1951-1973 &              10 &                    response, present, dilemma, influence, cooperate, bias, point, amount, participate, factor &                2 &                  0.0541 \\
 1951-1973 &              11 &                       trial, problem, previous, involve, prisoner, experiment, follow, tit, increase, initial &                1 &                   0.027 \\
 1951-1973 &              12 &                           matrix, behavior, rational, black, model, research, broad, distance, complex, trace &                1 &                   0.027 \\
 1951-1973 &              13 &                    play, finding, individual, noncooperative, white, nature, race, ratio, represent, prisoner &                1 &                   0.027 \\
 \midrule
 1951-1980 &               1 &                                      play, trial, group, follow, white, interpret, scale, black, trend, small &               14 &                    0.25 \\
 1951-1980 &               2 &                              outcome, level, effect, type, dyad, vary, pdg, participate, understanding, arise &                9 &                  0.1607 \\
 1951-1980 &               3 &         game, strategy, cooperation, significant, difference, sentence, text, occur, differential, hypothesis &                4 &                  0.0714 \\
 1951-1980 &               4 &                        male, female, find, result, sex, subject, experimental, situation, treatment, computer &                4 &                  0.0714 \\
 1951-1980 &               5 &                         research, problem, influence, matrix, format, model, analysis, year, crime, equipment &                4 &                  0.0714 \\
 1951-1980 &               6 &                                    condition, dilemma, bias, free, attempt, book, year, dummy, prison, design &                4 &                  0.0714 \\
 1951-1980 &               7 &                    variable, result, factor, individual, ability, triad, half, migration, change, investigate &                3 &                  0.0536 \\
 1951-1980 &               8 &                 show, present, suggest, rational, compete, approach, characteristic, examine, person, conduct &                3 &                  0.0536 \\
 1951-1980 &               9 &                         behavior, high, finding, relate, obtain, assistance, ratio, good, weapon, competition &                3 &                  0.0536 \\
 1951-1980 &              10 &                               ss, shock, money, competitive, part, difference, pair, amount, man, information &                3 &                  0.0536 \\
 1951-1980 &              11 &             player, conflict, theory, decision, determine, produce, maker, cooperate, specialist, programming &                2 &                  0.0357 \\
 1951-1980 &              12 &            study, prisoner, make, response, experiment, noncooperative, standard, separate, conclude, initial &                2 &                  0.0357 \\
 1951-1980 &              13 &                       give, cooperative, choice, cognitive, real, operational, set, subject, ascribe, concern &                1 &                  0.0179 \\
 \midrule
 1951-1988 &               1 &                     trial, difference, find, choice, significant, competitive, effect, triad, interact, occur &               24 &                  0.2553 \\
 1951-1988 &               2 &                                            ss, shock, money, pair, response, part, high, tit, receive, amount &               13 &                  0.1383 \\
 1951-1988 &               3 &                         suggest, paper, case, debate, view, achieve, framework, natural, assumption, finitely &               10 &                  0.1064 \\
 1951-1988 &               4 &                     prisoner, dilemma, behavior, model, present, involve, person, increase, trust, experiment &                8 &                  0.0851 \\
 1951-1988 &               5 &                                   game, player, show, approach, repeat, previous, move, tat, related, include &                8 &                  0.0851 \\
 1951-1988 &               6 &                cooperation, level, mutual, equilibrium, standard, provide, information, human, real, question &                6 &                  0.0638 \\
 1951-1988 &               7 &                      play, result, male, subject, female, cooperative, sex, experimental, treatment, computer &                5 &                  0.0532 \\
 1951-1988 &               8 &                        research, study, variable, ability, factor, conflict, matrix, year, student, interpret &                4 &                  0.0426 \\
 1951-1988 &               9 &                                         problem, group, small, scale, social, issue, large, base, bias, party &                4 &                  0.0426 \\
 1951-1988 &              10 &                          game, strategy, outcome, type, cooperate, ethical, pdg, explain, dependent, separate &                4 &                  0.0426 \\
 1951-1988 &              11 &              give, condition, individual, major, dyad, behaviour, produce, conflict, assistance, collectively &                3 &                  0.0319 \\
 1951-1988 &              12 &                        situation, iterate, statement, rational, card, side, paradox, true, consequence, front &                2 &                  0.0213 \\
 1951-1988 &              13 &                               inflation, hypothesis, rate, run, change, demand, nominal, cost, output, growth &                2 &                  0.0213 \\
 1951-1988 &              14 &                                     theory, make, analysis, decision, system, examine, work, soft, lead, hard &                1 &                  0.0106 \\
 \midrule
 1951-1995 &               1 &                            strategy, population, evolution, iterate, tit, opponent, evolve, dynamic, set, tat &               31 &                  0.1732 \\
 1951-1995 &               2 &                 game, repeat, assumption, rule, person, equilibrium, general, finitely, indefinitely, analyze &               24 &                  0.1341 \\
 1951-1995 &               3 &                            inflation, long, rate, hypothesis, run, policy, cost, nominal, demand, programming &               20 &                  0.1117 \\
 1951-1995 &               4 &            condition, outcome, trial, find, difference, cooperation, experiment, level, significant, response &               15 &                  0.0838 \\
 1951-1995 &               5 &                     rational, result, receive, statement, money, paradox, shock, iterate, consequence, common &               14 &                  0.0782 \\
 1951-1995 &               6 &             cooperation, show, competitive, high, probability, conflict, simulation, altruism, yield, natural &               14 &                  0.0782 \\
 1951-1995 &               7 &                           prisoner, dilemma, give, point, defect, form, cooperator, increase, relate, ethical &               10 &                  0.0559 \\
 1951-1995 &               8 &                       player, give, decision, provide, cooperative, game, previous, pair, determine, interact &                9 &                  0.0503 \\
 1951-1995 &               9 &                          play, cooperate, result, male, subject, female, time, relationship, suggest, student &                8 &                  0.0447 \\
 1951-1995 &              10 &                                   problem, group, theory, good, approach, society, large, scale, issue, level &                8 &                  0.0447 \\
 1951-1995 &              11 &            study, situation, behaviour, computer, argue, change, implication, characteristic, real, associate &                8 &                  0.0447 \\
 1951-1995 &              12 &                        model, paper, behavior, examine, present, mutual, expectation, develop, type, variable &                7 &                  0.0391 \\
 1951-1995 &              13 &                                   make, research, system, analysis, choice, work, base, relation, world, wide &                6 &                  0.0335 \\
 1951-1995 &              14 &               individual, social, behavior, standard, choose, evolutionary, partner, payoff, defection, small &                5 &                  0.0279 \\
 \midrule
 1951-2003 &               1 &                                    game, player, dilemma, prisoner, theory, give, paper, make, group, problem &              151 &                  0.4266 \\
 1951-2003 &               2 &                         cooperation, result, play, show, cooperate, condition, cooperative, high, level, time &              106 &                  0.2994 \\
 1951-2003 &               3 &                  strategy, model, agent, study, behavior, individual, population, evolutionary, state, player &               97 &                   0.274 \\
 \midrule
 1951-2010 &               1 &                                  model, theory, paper, base, make, present, problem, provide, human, decision &              325 &                  0.3454 \\
 1951-2010 &               2 &                                   game, strategy, player, agent, play, dilemma, system, behavior, show, state &              322 &                  0.3422 \\
 1951-2010 &               3 &  cooperation, network, study, population, individual, evolutionary, social, evolution, interaction, structure &              294 &                  0.3124 \\
 \midrule
 1951-2018 &               1 &                              model, theory, system, base, paper, problem, propose, present, approach, provide &              556 &                  0.2251 \\
 1951-2018 &               2 &                        behavior, social, human, decision, study, experiment, make, suggest, result, behaviour &              482 &                  0.1951 \\
 1951-2018 &               3 &                     individual, group, good, social, punishment, level, cost, mechanism, dilemma, cooperative &              428 &                  0.1733 \\
 1951-2018 &               4 &                            game, strategy, player, agent, play, dilemma, state, prisoner, payoff, equilibrium &              380 &                  0.1538 \\
 1951-2018 &               5 &                 population, evolutionary, dynamic, model, selection, result, evolution, evolve, show, process &              351 &                  0.1421 \\
 1951-2018 &               6 &       cooperation, network, interaction, structure, study, evolution, find, behavior, cooperative, simulation &              273 &                  0.1105 \\
\bottomrule
\end{tabular}
}
    \end{center}
    \caption{Topic modeling result for the cumulative data set over the periods
    }\label{table:topics_per_year}
\end{table}

But how do the five topics discussed earlier changed over time? The LDA models
with a number of 5, 6 and optimal number of topics per period are used to
calculate the contribution percentages for each document in the data of a given
period. The highest contribution percentage is chosen for each document and the
median of the highest contribution percentages for each model are given by
Figure~\ref{fig:median_percentage_contribution_over_time}.

The model with 5 number of topics has always had a better median contribution
that the model with 6 number of topics. Compared to the optimal models 
(Table~\ref{table:topics_per_year}) for periods 1951-1973 to 1951-1995
has a higher median. For periods 1951-2003 to 1951-2010 has a lower median, and
both models contribute that same at the period 1951-1965. Overall the 5 topics
appear to have fitted the publications over time well. The median highest
contribution has been increasing over time. This result suggests that the
topics have always been relevant in the PD publications.

\begin{figure}[!hbtp]
    \centering
    \includegraphics[width=.45\textwidth]{./assets/images/contribution_over_time.pdf}
    \caption{Medians of maximum percentage contributions over the time periods.}
    \label{fig:median_percentage_contribution_over_time}
\end{figure}

In the following section the collaborative behaviour of authors in the field,
and within the field's topics as were presented in this section, are explored
using a network theoretic approach.

\section{Analysis of co-authorship network}\label{section:co_authorship}

In this section the collaborative behaviour of authors in the field of the IPD
is assessed using the co-authorship network, which as mentioned in
Section~\ref{section:methodology} is denoted as \(G\). There are a total of
\connectedcomponents connected components in \(G\) and the largest component has
a size of \largestcc nodes. The largest connected component is going to be
refereed to as the main cluster of the network and is denoted as \(\bar{G}\). A
graphical representation of both networks is shown in
Figure~\ref{fig:graphical_representation_graphs} and a metrics summary is given by
Table~\ref{table:network_comparison.tex}.

\begin{figure}[!hbtp]\vspace{-2cm}
    \begin{subfigure}{.95\textwidth}\centering
        \includegraphics[width=.5\textwidth]{./assets/images/pd_network.pdf}
        \caption{\(G\) the co-authorship network for the IPD.}
    \end{subfigure}\hfill
    \begin{subfigure}{.95\textwidth}\centering
        \includegraphics[width=.5\textwidth]{./assets/images/pd_network_cluster.pdf}
        \caption{\(\bar{G}\) the largest connected component of \(G\).}
     \end{subfigure}
     \caption{A graphical representation of \(G\) and \(\bar{G}\)}\label{fig:graphical_representation_graphs}
\end{figure}

\textbf{Is the PD a collaborative field based on the co-authorship network?}

Based on Table~\ref{table:network_comparison.tex} an author in \(G\) has on
average 4 collaborators
and a 70\% probability of collaborating with a collaborator's co-author. An
author of \(\bar{G}\) on average is 7\% more likely to write with a
collaborator's co-author and on average has 2 more collaborators. Moreover, there
are only \isolatedpercentage\% of authors in the PD that has not connections to
any other author. However, both networks have a high modularity and a large
number of communities. A high modularity implies that authors create their own
publishing communities but not many publications from authors from different
communities occur. Thus, author tends to collaborate with authors in their
communities but not many efforts are made to create new connections to other
communities and spread the knowledge of the field across academic teams. The results
confirm the results of Section~\ref{section:results},
that the PD is indeed a collaborative field but perhaps it
is not no more collaborative than other fields because there is not effort
from the authors to write with people outside their community.

\begin{table}[!hbtp]
    \centering
    \resizebox{\textwidth}{!}{
    \begin{tabular}{lrrrrrrrrrr}
\toprule
{} &  \# Nodes &  \# Edges &  \# Isolated nodes &  \% Isolated nodes &  \# Connected components &  Size of largest component &  Av. degree &  \# Communities &  Modularity &  Clustering coeff \\
\midrule
Prisoner's Dilemma &     5394 &    10397 &               176 &               3.3 &                    1356 &                        815 &       3.855 &           1369 &       0.977 &             0.708 \\
Auction Games      &     5165 &     7861 &               256 &               5.0 &                    1272 &                       1348 &       3.044 &           1294 &       0.958 &             0.622 \\
Price of Anarchy   &     1155 &     1953 &                 4 &               0.3 &                     245 &                        222 &       3.382 &            253 &       0.965 &             0.712 \\
\bottomrule
\end{tabular}
}
    \caption{Network metrics for \(G\) and \(\bar{G}\) respectively.}
    \label{table:network_comparison.tex}
\end{table}

The evolution of the networks was also explored over time by constructing the
network cumulatively over 51 periods. Except from the first period 1951-1966 the
rest of the periods have a yearly interval (data for the years 1975 and 1982
were not collected by the collection data process). The metrics of each sub
network are given in the Appendix~\ref{appendix:tables}. The results, similarly
to the results of~\cite{Liu2015}, confirm that the networks grow over time and
that the networks always had a high modularity. Thus, the trend of authors
writing with people from their communities was not an effect of a specific
period.

\textbf{Are some topics more collaborative than other?}

The networks corresponding to the topics of Section~\ref{section:results} have
also been generated similarly to \(G\). Note that authors with publications in
more than one topic exist, and these authors are included in all the corresponding
networks. A metrics' summary for all five topic networks is given by Table
\ref{table:topics_networks}.

Topic B is the smallest network with the highest average degree, followed by
Topic A which has the second highest degree and is the largest network of all
five. In both B and A the number of isolated nodes is very small \((<0.2)\). In
comparison, in topics C and E authors are more likely to have no collaborators. Thus, in
topics such as ``human subject research'' and ``biological studies'' author tend to be
more collaborative than topics such as strategies (average degree 2.5) and
authors in A and B are less likely to have written single author publications
compared to modeling problems as a PD topic. Topic D is the topic associated
with evolutionary dynamics on networks and in fact the network of D is a
sub graph of \(\bar{G}\). In the main cluster it was shown that the authors have
on average 2 more collaborations and a higher likelihood of collaborating with a
collaborator's co-author and as it will be demonstrated in the following
section gain more from the influence of the network.

\begin{table}[H]
    \centering
    \resizebox{\textwidth}{!}{
    \begin{tabular}{lrrrrrrrrrr}
\toprule
{} &  \# Nodes &  \# Edges &  \# Isolated nodes &  \% Isolated nodes &  \# Connected components &  Size of largest component &  Av. degree &  \# Communities &  Modularity &  Clustering coeff \\
\midrule
Topic A &     1193 &     2137 &                84 &               7.0 &                     333 &                         56 &       3.583 &            334 &       0.983 &             0.715 \\
Topic B &      727 &     1382 &                45 &               6.2 &                     189 &                         80 &       3.802 &            190 &       0.950 &             0.739 \\
Topic C &      931 &     1141 &                72 &               7.7 &                     312 &                         29 &       2.451 &            312 &       0.981 &             0.615 \\
Topic D &      891 &     1509 &                28 &               3.1 &                     185 &                        312 &       3.387 &            193 &       0.917 &             0.692 \\
Topic E &     1152 &     1964 &               166 &              14.4 &                     461 &                         31 &       3.410 &            461 &       0.926 &             0.602 \\
\bottomrule
\end{tabular}
}
    \caption{Network metrics for topic networks.}\label{table:topics_networks}
\end{table}


\textbf{Are there authors in a more favourable position on the network? and
how does this corresponds to the networks}

There are two centrality measures reported in this work, closeness and
betweenness centrality. Closeness centrality is a measure of how easy it is for
an author to contact others, and consequently affect them; influence them. Thus
closeness centrality here is a measure of influence. Betweenness centrality is a
measure of how many paths pass through a specific node, thus the amount of
information this person has access to. Betweenness centrality is used here as a
measure of how much an author gains from the field. All centrality measure can
have values ranging from 0 to 1. The influence and the amount of information
an author has access to will be explored to assess whether there are authors
that benefit from the position.

For \(G\) and \(\bar{G}\) the most central authors based on closeness and
betweenness centralities are given by Table~\ref{table:central_authors}.
The most central authors in \(\bar{G}\) are the same people as in \(G\). This
implies that
the results on centrality heavily rely of the main cluster.
Matjaz
Perc is an author with 58 publications in the data set and the most central
authors in \(G\). The most central authors are fairly similar between the two
measures. The author that appear to be central based on one measure and not the
other are Martin Nowak, Franz Weissing, Jianye Hao, Angel Sanchez and Valerio
Capraro are authors that have access to information due to their positioning but
do not influence the network as much and the opposite is true for the authors
Attila Szolnoki, Luo-Luo Jiang Sandro Meloni, Cheng-Yi Xia and Xiaojie Chen.


The distributions of both centrality measures for both networks are given in
the Appendix~\ref{appendix:distributions}.

From the distributions for G it is obvious that the values of centralities are overall low,
and thus in the PD an author does not gain as much information or influence
the field. This is not true for the authors in G. The centrality values are
larger.

\begin{table}[!hbtp]
    \begin{center}
    \resizebox{.9\textwidth}{!}{\begin{tabular}{llrlrlrlr}
\toprule
& \multicolumn{4}{c}{$G$} & \multicolumn{4}{c}{$\bar{G}$} \\
\midrule
{} &             Name &  Betweenness &             Name &  Closeness &             Name &  Betweenness &             Name &  Closeness \\
\midrule
1  &      Matjaz Perc &        0.013 &      Matjaz Perc &      0.062 &      Matjaz Perc &        0.373 &      Matjaz Perc &      0.330 \\
2  &        Zhen Wang &        0.010 &        Long Wang &      0.057 &        Zhen Wang &        0.279 &        Long Wang &      0.301 \\
3  &        Long Wang &        0.006 &     Yamir Moreno &      0.056 &        Long Wang &        0.170 &     Yamir Moreno &      0.299 \\
4  &     Martin Nowak &        0.006 &  Attila Szolnoki &      0.056 &     Martin Nowak &        0.159 &  Attila Szolnoki &      0.297 \\
5  &    Angel Sanchez &        0.004 &        Zhen Wang &      0.056 &    Angel Sanchez &        0.114 &        Zhen Wang &      0.296 \\
6  &     Yamir Moreno &        0.004 &    Arne Traulsen &      0.053 &     Yamir Moreno &        0.110 &    Arne Traulsen &      0.281 \\
7  &    Arne Traulsen &        0.004 &    Luo-Luo Jiang &      0.053 &    Arne Traulsen &        0.107 &    Luo-Luo Jiang &      0.280 \\
8  &   Franz Weissing &        0.004 &    Sandro Meloni &      0.052 &   Franz Weissing &        0.101 &    Sandro Meloni &      0.278 \\
9  &       Jianye Hao &        0.003 &     Cheng-Yi Xia &      0.052 &       Jianye Hao &        0.094 &     Cheng-Yi Xia &      0.276 \\
10 &  Valerio Capraro &        0.003 &     Xiaojie Chen &      0.052 &  Valerio Capraro &        0.093 &     Xiaojie Chen &      0.276 \\
\bottomrule
\end{tabular}
}
\end{center}
\caption{10 most central authors based on betweenness and closeness centralities
for \(G\) and \(\bar{G}\).}\label{table:central_authors}
\end{table}

\newcolumntype{g}{>{\columncolor{Gray}}l}
\begin{table}[!hbtp]
    \begin{center}
    \resizebox{.9\textwidth}{!}{\begin{tabular}{lggllggllgg}
\toprule
& \multicolumn{10}{c}{\textbf{Betweenness centrality}}\\
\midrule
& \multicolumn{2}{g}{Topic A} & \multicolumn{2}{c}{Topic B} & \multicolumn{2}{g}{Topic C} & \multicolumn{2}{c}{Topic D} & \multicolumn{2}{g}{Topic E}\\
\midrule
{} &                 Name &  Betweeness &             Name &  Betweeness &             Name &  Betweeness &              Name &  Betweeness &                  Name &  Betweeness \\
\midrule
1  &           David Rand &       0.002 &        Long Wang &       0.006 &   Daniel Ashlock &       0.001 &       Matjaz Perc &       0.064 &             Zengru Di &         0.0 \\
2  &      Valerio Capraro &       0.001 &    Luo-Luo Jiang &       0.005 &      Matjaz Perc &       0.000 &     Luo-Luo Jiang &       0.037 &             Jian Yang &         0.0 \\
3  &        Angel Sanchez &       0.001 &     Martin Nowak &       0.004 &       Karl Tuyls &       0.000 &      Yamir Moreno &       0.031 &  Yevgeniy Vorobeychik &         0.0 \\
4  &              Feng Fu &       0.001 &      Matjaz Perc &       0.003 &  Philip Hingston &       0.000 &  Christoph Hauert &       0.027 &       Otavio Teixeira &         0.0 \\
5  &         Martin Nowak &       0.000 &  Attila Szolnoki &       0.003 &     Eun-Youn Kim &       0.000 &         Long Wang &       0.024 &      Roberto Oliveira &         0.0 \\
6  &  Nicholas Christakis &       0.000 &  Christian Hilbe &       0.002 &    Wendy Ashlock &       0.000 &         Zhen Wang &       0.024 &              M. Nowak &         0.0 \\
7  &   Pablo Branas-Garza &       0.000 &     Yamir Moreno &       0.002 &  Attila Szolnoki &       0.000 &      Han-Xin Yang &       0.023 &             M. Harper &         0.0 \\
8  &     Toshio Yamagishi &       0.000 &     Xiaojie Chen &       0.002 &       Seung Baek &       0.000 &      Martin Nowak &       0.020 &              Xiao Han &         0.0 \\
9  &         James Fowler &       0.000 &    Arne Traulsen &       0.002 &     Martin Nowak &       0.000 &     Angel Sanchez &       0.017 &            Zhesi Shen &         0.0 \\
10 &            Long Wang &       0.000 &        Zhen Wang &       0.002 &    Thore Graepel &       0.000 &       Zhihai Rong &       0.016 &           Wen-Xu Wang &         0.0 \\
\bottomrule
& \multicolumn{10}{c}{\textbf{Closeness centrality}}\\
\midrule
{} &                 Name &  Closeness &               Name &  Closeness &                 Name &  Closeness &             Name &  Closeness &             Name &  Closeness \\
\midrule
1  &           David Rand &      0.027 &          Long Wang &      0.043 &           Karl Tuyls &      0.022 &      Matjaz Perc &      0.123 &  Stefanie Widder &      0.029 \\
2  &      Valerio Capraro &      0.023 &        Matjaz Perc &      0.041 &        Thore Graepel &      0.019 &        Zhen Wang &      0.109 &   Rosalind Allen &      0.029 \\
3  &       Jillian Jordan &      0.022 &    Attila Szolnoki &      0.040 &           Joel Leibo &      0.018 &        Long Wang &      0.107 &  Thomas Pfeiffer &      0.029 \\
4  &  Nicholas Christakis &      0.021 &       Martin Nowak &      0.040 &        Edward Hughes &      0.017 &     Yamir Moreno &      0.105 &    Thomas Curtis &      0.029 \\
5  &         James Fowler &      0.020 &  Olivier Tenaillon &      0.038 &     Matthew Phillips &      0.017 &    Luo-Luo Jiang &      0.104 &     Carsten Wiuf &      0.029 \\
6  &         Martin Nowak &      0.020 &       Xiaojie Chen &      0.038 &  Edgar Duenez-Guzman &      0.017 &  Attila Szolnoki &      0.103 &    William Sloan &      0.029 \\
7  &        Angel Sanchez &      0.019 &             Bin Wu &      0.038 &    Antonio Castaneda &      0.017 &     Gyorgy Szabo &      0.102 &     Otto Cordero &      0.029 \\
8  &    Gordon Kraft-Todd &      0.019 &      Yanling Zhang &      0.037 &         Iain Dunning &      0.017 &     Xiaojie Chen &      0.102 &        Sam Brown &      0.029 \\
9  &        Akihiro Nishi &      0.019 &            Feng Fu &      0.037 &             Tina Zhu &      0.017 &    Guangming Xie &      0.101 &     Babak Momeni &      0.029 \\
10 &        Anthony Evans &      0.019 &         David Rand &      0.037 &          Kevin Mckee &      0.017 &     Lucas Wardil &      0.101 &     Wenying Shou &      0.029 \\
\bottomrule
\end{tabular}
}
\end{center}
\caption{10 most central authors based on betweenness centrality
for topics' networks.}\label{table:central_authors_bc_topics}
\end{table}

\newcolumntype{g}{>{\columncolor{Gray}}l}
\begin{table}[!hbtp]
    \begin{center}
    \resizebox{.9\textwidth}{!}{\begin{tabular}{lggllggllgg}
\toprule
& \multicolumn{2}{g}{Topic A} & \multicolumn{2}{c}{Topic B} & \multicolumn{2}{g}{Topic C} & \multicolumn{2}{c}{Topic D} & \\
\midrule
{} &                 Name &  Closeness &               Name &  Closeness &                 Name &  Closeness &             Name &  Closeness &             Name &  Closeness \\
\midrule
1  &           David Rand &      0.026 &          Long Wang &      0.042 &           Karl Tuyls &      0.021 &      Matjaz Perc &      0.122 &  Stefanie Widder &      0.026 \\
2  &      Valerio Capraro &      0.022 &        Matjaz Perc &      0.039 &        Thore Graepel &      0.019 &        Zhen Wang &      0.107 &   Rosalind Allen &      0.026 \\
3  &       Jillian Jordan &      0.021 &    Attila Szolnoki &      0.039 &           Joel Leibo &      0.018 &        Long Wang &      0.105 &  Thomas Pfeiffer &      0.026 \\
4  &  Nicholas Christakis &      0.020 &       Martin Nowak &      0.038 &        Edward Hughes &      0.017 &     Yamir Moreno &      0.103 &    Thomas Curtis &      0.026 \\
5  &         James Fowler &      0.019 &  Olivier Tenaillon &      0.037 &     Matthew Phillips &      0.017 &    Luo-Luo Jiang &      0.102 &     Carsten Wiuf &      0.026 \\
6  &         Martin Nowak &      0.019 &       Xiaojie Chen &      0.036 &  Edgar Duenez-Guzman &      0.017 &  Attila Szolnoki &      0.102 &    William Sloan &      0.026 \\
7  &        Angel Sanchez &      0.018 &             Bin Wu &      0.036 &    Antonio Castaneda &      0.017 &     Gyorgy Szabo &      0.101 &     Otto Cordero &      0.026 \\
8  &      Samuel Arbesman &      0.018 &      Yanling Zhang &      0.035 &         Iain Dunning &      0.017 &     Xiaojie Chen &      0.100 &        Sam Brown &      0.026 \\
9  &    Gordon Kraft-Todd &      0.018 &            Feng Fu &      0.035 &             Tina Zhu &      0.017 &    Guangming Xie &      0.100 &     Babak Momeni &      0.026 \\
10 &        Akihiro Nishi &      0.018 &         David Rand &      0.035 &          Kevin Mckee &      0.017 &     Lucas Wardil &      0.100 &     Wenying Shou &      0.026 \\
\bottomrule
\end{tabular}
}
\end{center}
\caption{10 most central authors based on closeness centrality
for topics' networks.}\label{table:central_authors_cc_topics}
\end{table}

\section{Conclusion}\label{section:conclusion}

This manuscript has explored the number of publications, the authors'
collaborative behaviour and their influence in the research topic of the Iterated
Prisoner's Dilemma. This was achieved by applying network theoretic approaches and
a bibliometric analysis in a data containing more than 2000 publications.
The data set was automatically
collected from five different sources using a bespoke piece of software written
for this purpose~\cite{nikoleta_2017}.

The data collection and an introduction to the methodology used in this work
were covered in Section~\ref{section:methodology}.
The data set contains a total of \totalarticles articles, it has been archive
and it available in~\cite{pd_data_2018} for further analysis.
Section~\ref{section:results} covered an initial analysis of the data set,
and applied a topic modeling algorithm which classified the articles into
topics. The initial analysis demonstrated that the field of the Prisoner's
Dilemma remains a prominent field with several papers being published in journals.
The topic modeling analysis identified five different topics which appeared
to be human subject research, biological studies, strategies and agent based simulations,
evolutionary dynamics on networks and modeling problems as a PD.
A temporal topic modeling analysis showed that over time not only
the number of topic changed, but also the scientific language and the most important
the topics authors were publishing on.

Following Section~\ref{section:results}, the collaborative behaviour of the field was explored
in Section~\ref{section:co_authorship}. It was concluded that the field
of the Iterated Prisoner's Dilemma is a collaborative field where authors
are likely to write with a collaborator's co-authors and on average an author
has 4 co-authors. The results on collaborativeness were verified when studying
the co-authorship network for each of the five topics defined in Section~\ref{section:results}
as well. Exploring the influence of authors and their gain from being in the
network of the field demonstrated that authors do not gain much, and the authors
with influence are the ones connected to the main cluster, to a "main" group of authors.

The study of the Prisoner's Dilemma is the study of cooperation and investigating
the cooperative behaviours of authors is what this work has aimed to achieve.
Interesting areas of future work would include extending this analysis to more
game theoretic sub fields, to evaluate whether the results remain the same.

\section{Acknowledgements}

A variety of software have been used in this work:

\begin{itemize}
    \item The Matplotlib library for visualisation~\cite{hunter2007matplotlib}.
    \item The Numpy library for data manipulation~\cite{walt2011numpy}.
    \item The Networkx~\cite{networkx} package for analysing networks.
    \item Gephi~\cite{ICWSM09154} open source package for visualising networks.
    \item The Gensim library for the topic modeling~\cite{rehurek_lrec}.
    \item The louvain library for calculating the networks modularity \url{https://github.com/taynaud/python-louvain}.
\end{itemize}

\bibliographystyle{plain}
\bibliography{bibliography.bib}

\appendix

\section{Cumulative Networks Metrics}\label{appendix:tables}

\subsection{Collaborativeness metrics for cumulative graphs, \(\tilde{G} \subseteq G\)}
\begin{table}[!hbtp]
    \centering
    \resizebox{.8\textwidth}{!}{
    \begin{tabular}{lrrrrrrl}
\toprule
{} &  \# Connected Components &  \# Isolated &  \% Isolated &  Av. Degree &  Clustering &  Largest cc & Modularity \\
\midrule
Period 0  &                       3 &           3 &        1.00 &        0.00 &        0.00 &           1 &          - \\
Period 1  &                       2 &           2 &        1.00 &        0.00 &        0.00 &           1 &          - \\
Period 2  &                       3 &           3 &        1.00 &        0.00 &        0.00 &           1 &          - \\
Period 3  &                       4 &           4 &        1.00 &        0.00 &        0.00 &           1 &          - \\
Period 4  &                       6 &           6 &        1.00 &        0.00 &        0.00 &           1 &          - \\
Period 5  &                       7 &           7 &        1.00 &        0.00 &        0.00 &           1 &          - \\
Period 6  &                       7 &           7 &        1.00 &        0.00 &        0.00 &           1 &          - \\
Period 7  &                       8 &           8 &        1.00 &        0.00 &        0.00 &           1 &          - \\
Period 8  &                       9 &           9 &        1.00 &        0.00 &        0.00 &           1 &          - \\
Period 9  &                      10 &          10 &        1.00 &        0.00 &        0.00 &           1 &          - \\
Period 10 &                      12 &          10 &        0.71 &        0.29 &        0.00 &           2 &        0.5 \\
Period 11 &                      15 &          12 &        0.67 &        0.33 &        0.00 &           2 &   0.666667 \\
Period 12 &                      19 &          13 &        0.46 &        0.93 &        0.16 &           5 &   0.591716 \\
Period 13 &                      21 &          15 &        0.48 &        0.97 &        0.16 &           6 &   0.533333 \\
Period 14 &                      23 &          16 &        0.47 &        0.94 &        0.15 &           6 &   0.585938 \\
Period 15 &                      26 &          16 &        0.38 &        1.10 &        0.26 &           6 &   0.763705 \\
Period 16 &                      27 &          16 &        0.36 &        1.16 &        0.31 &           6 &   0.801775 \\
Period 17 &                      29 &          17 &        0.35 &        1.12 &        0.29 &           6 &   0.814815 \\
Period 18 &                      29 &          17 &        0.35 &        1.12 &        0.29 &           6 &   0.814815 \\
Period 19 &                      30 &          18 &        0.37 &        1.10 &        0.29 &           6 &   0.814815 \\
Period 20 &                      30 &          18 &        0.37 &        1.10 &        0.29 &           6 &   0.814815 \\
Period 21 &                      33 &          19 &        0.35 &        1.07 &        0.26 &           6 &   0.837099 \\
Period 22 &                      34 &          19 &        0.34 &        1.07 &        0.25 &           6 &   0.846667 \\
Period 23 &                      36 &          21 &        0.34 &        1.11 &        0.25 &           6 &   0.854671 \\
Period 24 &                      37 &          22 &        0.34 &        1.16 &        0.28 &           6 &   0.866326 \\
Period 25 &                      37 &          22 &        0.33 &        1.27 &        0.29 &           6 &    0.85941 \\
Period 26 &                      40 &          24 &        0.34 &        1.27 &        0.32 &           6 &   0.873086 \\
Period 27 &                      43 &          26 &        0.35 &        1.23 &        0.30 &           6 &   0.878072 \\
Period 28 &                      46 &          28 &        0.35 &        1.19 &        0.28 &           6 &   0.882752 \\
Period 29 &                      46 &          28 &        0.35 &        1.19 &        0.28 &           6 &   0.882752 \\
Period 30 &                      54 &          35 &        0.40 &        1.09 &        0.26 &           6 &   0.887153 \\
Period 31 &                      58 &          38 &        0.41 &        1.05 &        0.24 &           6 &   0.891295 \\
Period 32 &                      61 &          39 &        0.39 &        1.07 &        0.26 &           6 &   0.903524 \\
Period 33 &                      70 &          43 &        0.37 &        1.10 &        0.26 &           6 &   0.921643 \\
Period 34 &                      75 &          45 &        0.36 &        1.10 &        0.26 &           6 &   0.930363 \\
Period 35 &                      84 &          50 &        0.36 &        1.07 &        0.26 &           6 &   0.939007 \\
Period 36 &                      87 &          52 &        0.36 &        1.07 &        0.25 &           6 &   0.942486 \\
Period 37 &                      95 &          54 &        0.34 &        1.12 &        0.26 &           6 &   0.951852 \\
Period 38 &                     110 &          62 &        0.32 &        1.24 &        0.33 &           6 &    0.95996 \\
Period 39 &                     118 &          64 &        0.31 &        1.23 &        0.31 &           6 &   0.964111 \\
Period 40 &                     127 &          68 &        0.30 &        1.30 &        0.32 &           6 &   0.966357 \\
Period 41 &                     137 &          71 &        0.29 &        1.31 &        0.34 &           6 &   0.970812 \\
Period 42 &                     138 &          55 &        0.20 &        1.42 &        0.36 &           6 &   0.976778 \\
Period 43 &                     139 &          49 &        0.17 &        1.50 &        0.38 &           6 &   0.979126 \\
Period 44 &                     156 &          54 &        0.16 &        1.53 &        0.40 &           6 &   0.981765 \\
Period 45 &                     157 &          42 &        0.12 &        1.67 &        0.42 &           9 &   0.980204 \\
Period 46 &                     178 &          48 &        0.12 &        1.70 &        0.43 &           9 &   0.982271 \\
Period 47 &                     208 &          54 &        0.11 &        1.66 &        0.43 &           9 &   0.985315 \\
Period 48 &                     230 &          54 &        0.10 &        1.73 &        0.44 &          10 &   0.986425 \\
Period 49 &                     256 &          54 &        0.09 &        1.82 &        0.47 &          20 &   0.985488 \\
Period 50 &                     309 &          62 &        0.08 &        1.94 &        0.50 &          22 &   0.988459 \\
Period 51 &                     361 &          72 &        0.07 &        2.18 &        0.53 &          26 &   0.986283 \\
Period 52 &                     404 &          82 &        0.07 &        2.40 &        0.55 &          40 &   0.984604 \\
Period 53 &                     451 &          94 &        0.07 &        2.50 &        0.55 &          70 &   0.979801 \\
Period 54 &                     503 &         106 &        0.07 &        2.81 &        0.56 &         195 &   0.965391 \\
Period 55 &                     570 &         116 &        0.06 &        3.00 &        0.59 &         259 &   0.964951 \\
Period 56 &                     637 &         120 &        0.05 &        3.16 &        0.62 &         377 &   0.960718 \\
Period 57 &                     700 &         131 &        0.05 &        3.30 &        0.63 &         498 &   0.956249 \\
Period 58 &                     769 &         139 &        0.05 &        3.35 &        0.64 &         651 &   0.948894 \\
Period 59 &                     857 &         148 &        0.04 &        3.52 &        0.65 &         845 &   0.943237 \\
Period 60 &                     935 &         155 &        0.04 &        3.98 &        0.67 &        1116 &   0.939336 \\
Period 61 &                     978 &         157 &        0.04 &        4.04 &        0.68 &        1253 &   0.938227 \\
Period 62 &                    1029 &         157 &        0.03 &        4.19 &        0.68 &        1456 &   0.929681 \\
\bottomrule
\end{tabular}
}
\end{table}

\newpage

\subsection{Collaborativeness metrics for cumulative graphs' main clusters, \(\tilde{G} \subseteq \bar{G}\)}
\begin{table}[!hbtp]
    \centering
    \resizebox{.8\textwidth}{!}{
    \begin{tabular}{lrrrrrrrlr}
\toprule
{} &  \# Nodes &  \# Edges &  \# Isolated nodes &  \% Isolated nodes &  \# Connected components &  Size of largest component &  Av. degree & Modularity &  Clustering coeff \\
\midrule
1954 - 1950 &        1 &        0 &                 1 &             100.0 &                       1 &                          1 &       0.000 &          - &             0.000 \\
1954 - 1955 &        1 &        0 &                 1 &             100.0 &                       1 &                          1 &       0.000 &          - &             0.000 \\
1955 - 1956 &        1 &        0 &                 1 &             100.0 &                       1 &                          1 &       0.000 &          - &             0.000 \\
1956 - 1957 &        1 &        0 &                 1 &             100.0 &                       1 &                          1 &       0.000 &          - &             0.000 \\
1957 - 1958 &        1 &        0 &                 1 &             100.0 &                       1 &                          1 &       0.000 &          - &             0.000 \\
1958 - 1959 &        1 &        0 &                 1 &             100.0 &                       1 &                          1 &       0.000 &          - &             0.000 \\
1959 - 1961 &        1 &        0 &                 1 &             100.0 &                       1 &                          1 &       0.000 &          - &             0.000 \\
1961 - 1962 &        1 &        0 &                 1 &             100.0 &                       1 &                          1 &       0.000 &          - &             0.000 \\
1962 - 1964 &        1 &        0 &                 1 &             100.0 &                       1 &                          1 &       0.000 &          - &             0.000 \\
1964 - 1965 &        1 &        0 &                 1 &             100.0 &                       1 &                          1 &       0.000 &          - &             0.000 \\
1965 - 1966 &        2 &        1 &                 0 &               0.0 &                       1 &                          2 &       1.000 &          0 &             0.000 \\
1966 - 1967 &        2 &        1 &                 0 &               0.0 &                       1 &                          2 &       1.000 &          0 &             0.000 \\
1967 - 1968 &        5 &        8 &                 0 &               0.0 &                       1 &                          5 &       3.200 &          0 &             0.867 \\
1968 - 1969 &        6 &       10 &                 0 &               0.0 &                       1 &                          6 &       3.333 &       0.02 &             0.833 \\
1969 - 1970 &        6 &       10 &                 0 &               0.0 &                       1 &                          6 &       3.333 &       0.02 &             0.833 \\
1970 - 1971 &        6 &       10 &                 0 &               0.0 &                       1 &                          6 &       3.333 &       0.02 &             0.833 \\
1971 - 1972 &        6 &       10 &                 0 &               0.0 &                       1 &                          6 &       3.333 &       0.02 &             0.833 \\
1972 - 1973 &        6 &       10 &                 0 &               0.0 &                       1 &                          6 &       3.333 &       0.02 &             0.833 \\
1973 - 1974 &        6 &       10 &                 0 &               0.0 &                       1 &                          6 &       3.333 &       0.02 &             0.833 \\
1974 - 1975 &        6 &       10 &                 0 &               0.0 &                       1 &                          6 &       3.333 &       0.02 &             0.833 \\
1975 - 1976 &        6 &       10 &                 0 &               0.0 &                       1 &                          6 &       3.333 &       0.02 &             0.833 \\
1976 - 1977 &        6 &       10 &                 0 &               0.0 &                       1 &                          6 &       3.333 &       0.02 &             0.833 \\
1977 - 1978 &        6 &       10 &                 0 &               0.0 &                       1 &                          6 &       3.333 &       0.02 &             0.833 \\
1978 - 1979 &        6 &       10 &                 0 &               0.0 &                       1 &                          6 &       3.333 &       0.02 &             0.833 \\
1979 - 1980 &        6 &       10 &                 0 &               0.0 &                       1 &                          6 &       3.333 &       0.02 &             0.833 \\
1980 - 1981 &        6 &        9 &                 0 &               0.0 &                       1 &                          6 &       3.000 &  0.0493827 &             0.678 \\
1981 - 1982 &        6 &        9 &                 0 &               0.0 &                       1 &                          6 &       3.000 &  0.0493827 &             0.678 \\
1982 - 1983 &        6 &       10 &                 0 &               0.0 &                       1 &                          6 &       3.333 &       0.02 &             0.833 \\
1983 - 1984 &        6 &        9 &                 0 &               0.0 &                       1 &                          6 &       3.000 &  0.0493827 &             0.678 \\
1984 - 1985 &        6 &        9 &                 0 &               0.0 &                       1 &                          6 &       3.000 &  0.0493827 &             0.678 \\
1985 - 1986 &        6 &       10 &                 0 &               0.0 &                       1 &                          6 &       3.333 &       0.02 &             0.833 \\
1986 - 1987 &        6 &        9 &                 0 &               0.0 &                       1 &                          6 &       3.000 &  0.0493827 &             0.678 \\
1987 - 1988 &        6 &       10 &                 0 &               0.0 &                       1 &                          6 &       3.333 &       0.02 &             0.833 \\
1988 - 1989 &        6 &       10 &                 0 &               0.0 &                       1 &                          6 &       3.333 &       0.02 &             0.833 \\
1989 - 1990 &        6 &        9 &                 0 &               0.0 &                       1 &                          6 &       3.000 &  0.0493827 &             0.678 \\
1990 - 1991 &        6 &        9 &                 0 &               0.0 &                       1 &                          6 &       3.000 &  0.0493827 &             0.678 \\
1991 - 1992 &        6 &        9 &                 0 &               0.0 &                       1 &                          6 &       3.000 &  0.0493827 &             0.678 \\
1992 - 1993 &        6 &        9 &                 0 &               0.0 &                       1 &                          6 &       3.000 &  0.0493827 &             0.678 \\
1993 - 1994 &        6 &        9 &                 0 &               0.0 &                       1 &                          6 &       3.000 &  0.0493827 &             0.678 \\
1994 - 1995 &        6 &        9 &                 0 &               0.0 &                       1 &                          6 &       3.000 &  0.0493827 &             0.678 \\
1995 - 1996 &        6 &        9 &                 0 &               0.0 &                       1 &                          6 &       3.000 &  0.0493827 &             0.678 \\
1996 - 1997 &        6 &        9 &                 0 &               0.0 &                       1 &                          6 &       3.000 &  0.0493827 &             0.678 \\
1997 - 1998 &        6 &        9 &                 0 &               0.0 &                       1 &                          6 &       3.000 &  0.0493827 &             0.678 \\
1998 - 1999 &        6 &        9 &                 0 &               0.0 &                       1 &                          6 &       3.000 &  0.0493827 &             0.678 \\
1999 - 2000 &        6 &       10 &                 0 &               0.0 &                       1 &                          6 &       3.333 &       0.02 &             0.833 \\
2000 - 2001 &        7 &       21 &                 0 &               0.0 &                       1 &                          7 &       6.000 &          0 &             1.000 \\
2001 - 2002 &        7 &       21 &                 0 &               0.0 &                       1 &                          7 &       6.000 &          0 &             1.000 \\
2002 - 2003 &        7 &       21 &                 0 &               0.0 &                       1 &                          7 &       6.000 &          0 &             1.000 \\
2003 - 2004 &       10 &       13 &                 0 &               0.0 &                       1 &                         10 &       2.600 &    0.37574 &             0.553 \\
2004 - 2005 &       19 &       28 &                 0 &               0.0 &                       1 &                         19 &       2.947 &   0.544005 &             0.730 \\
2005 - 2006 &       21 &       32 &                 0 &               0.0 &                       1 &                         21 &       3.048 &   0.530273 &             0.713 \\
2006 - 2007 &       24 &       36 &                 0 &               0.0 &                       1 &                         24 &       3.000 &   0.533179 &             0.678 \\
2007 - 2008 &       32 &       59 &                 0 &               0.0 &                       1 &                         32 &       3.688 &   0.627837 &             0.732 \\
2008 - 2009 &       56 &      102 &                 0 &               0.0 &                       1 &                         56 &       3.643 &   0.716792 &             0.699 \\
2009 - 2010 &       99 &      238 &                 0 &               0.0 &                       1 &                         99 &       4.808 &   0.781539 &             0.734 \\
2010 - 2011 &      121 &      288 &                 0 &               0.0 &                       1 &                        121 &       4.760 &   0.776054 &             0.713 \\
2011 - 2012 &      210 &      610 &                 0 &               0.0 &                       1 &                        210 &       5.810 &     0.7809 &             0.747 \\
2012 - 2013 &      330 &      908 &                 0 &               0.0 &                       1 &                        330 &       5.503 &    0.81959 &             0.753 \\
2013 - 2014 &      406 &     1125 &                 0 &               0.0 &                       1 &                        406 &       5.542 &    0.81494 &             0.749 \\
2014 - 2015 &      514 &     1390 &                 0 &               0.0 &                       1 &                        514 &       5.409 &   0.827358 &             0.757 \\
2015 - 2016 &      614 &     1682 &                 0 &               0.0 &                       1 &                        614 &       5.479 &   0.831149 &             0.765 \\
2016 - 2017 &      703 &     1925 &                 0 &               0.0 &                       1 &                        703 &       5.477 &   0.839165 &             0.774 \\
2017 - 2018 &      815 &     2300 &                 0 &               0.0 &                       1 &                        815 &       5.644 &   0.857237 &             0.775 \\
\bottomrule
\end{tabular}
}
\end{table}


\section{Centrality Measures Distributions}

\subsection{Distrubutions for \(G\) and \(\bar{G}\)}

\begin{figure}[!hbtp]
    \centering
    \includegraphics[width=.8\textwidth]{./assets/images/pd_betweeness_centralities.pdf}
    \caption{Distributions of betweenness centrality in \(G\) and \(\bar{G}\)}
    \label{fig:bc_distributions}
\end{figure}

\begin{figure}[!hbtp]
    \centering
    \includegraphics[width=.8\textwidth]{./assets/images/pd_closeness_centralities.pdf}
    \caption{Distributions of closeness centrality in \(G\) and \(\bar{G}\)}
    \label{fig:cc_distributions}
\end{figure}

\subsection{Distrubutions for Topic Networks}\label{appendix:distributions}

\begin{figure}[!hbtp]
    \centering
    \includegraphics[width=\textwidth]{./assets/images/topics_betweeness_distributions.pdf}
    \caption{Distributions of betweenness centrality in topics' networks.}
    \label{fig:bc_distributions_topics}
\end{figure}

\begin{figure}[!hbtp]
    \centering
    \includegraphics[width=\textwidth]{./assets/images/topics_closeness_distributions.pdf}
    \caption{Distributions of closeness centrality in topics' networks.}
    \label{fig:cc_distributions_topics}
\end{figure}

\end{document}